\documentclass[12pt]{article}
\usepackage[latin1]{inputenc}
\usepackage{graphicx,subfigure}
\topmargin -1mm
\oddsidemargin -1mm
\evensidemargin -1mm
\textwidth 165mm
\textheight 220mm
\parskip 2mm
\parindent 3mm
%\pagestyle{empty}

\begin{document}
\begin{center}
{\Large \bf Help and Documentation}

Bo Sundman, \today

\end{center}

After a successful installation you can find further information how
to use OC below.
\begin{itemize}
\item In the documentation directory there are several PDF files with
  different kinds of information in addition to the installation
  guides.

  The ``Getting\_started'' file shows how to operate the program.
  Also look at the ``OC-macro'' documentation which describes several
  macro files which you have on a directory ``macro/ocv4''.  These are
  useful to test the software and gives some ideas how to use it.
  There is also a ``news-OC4'' for a summary of the features of OC.

  The documentation of the source code is also in this directory
  ``gtp3'' for the model package and ``hms2'' for the minimizer.  The
  other software documentation, ``smp2'' for the step/map/plot
  routines, ``assess'' for the assessment module and ``ocasi'' for the
  application software interface are very preliminary.

\item In the manual directory there is a preliminary user guide
  ``ochelp'' The user guide is also available in the file
  ``ochelp.hlp'' in LaTeX format because it is used for the on line
  help.  The user guide is still very primitive as many commands are
  changing or not yet implemented.

\item The source code is in the directories ``minimizer, models,
  numlib, stepmapplot, userif'' and ``utilities''.

\item The TQ4lib directory has a few examples for using the software
  interface for Fortran and C/C++.  An attempt to implement isoC
  binding to C++ has been made.

\item Contributions of new and improved source code are welcome.  You
  can do this using the github repository.  Contact Bo Sundman if you
  want to know more.

\item the command line interface has a ``VAX/VMS'' flavor which
  reflects the age of the developer.  It means the commands are
  ``verbs'' like {\em set, list, calculate, enter} etc.  After the
  verb several objects are usually possible like {\em set~condition}
  or {set~status} etc.

  Each command and subcommand can be abbreviated, usually 3 characters
  are sufficient.

  If you want a graphical user interface you are welcome to develop
  it.
\end{itemize}

You are welcome to help providing a better installation guide also!

\bigskip

{\large \bf Have fun and help make OC useful!}

\end{document}

