\documentclass[12pt]{article}
\usepackage[latin1]{inputenc}
\usepackage{graphicx,subfigure}
\topmargin -1mm
\oddsidemargin -1mm
\evensidemargin -1mm
\textwidth 165mm
\textheight 220mm
\parskip 2mm
\parindent 3mm
%\pagestyle{empty}

\begin{document}
\begin{center}
{\Large \bf Installation of OpenCalphad on Windows using MinGW}

Bo Sundman, \today

\end{center}

There is no automatic installation routine for OC, you must download
and compile the software yourself.  You may also have to install
Fortran compilers and the GNUPLOT software if you do not already have
them.  The OC development team cannot offer you any help for that,
please ask some local experts if you need help.

The descrcipton below applies when installing OC on Windows using
MinGW, the guides available are:
\begin{itemize}
\item Install-OC-Windows-MinGW
\item Installation de OC sous Windows avec Cygwin (in French)
\item Install-OC-Linux
\end{itemize}

Step by step installation:

\begin{itemize}
\item The code is written in the new Fortran standard and requires a
  compiler like GNU Fortran 4.8 or similar.

\item If you have not already installed MinGW and the Fortran compiler
  you must do that from https://SourgeForge.net or some similar site.
  If you have MinGW but not the Fortran compiler you must add that.
  The MinGW software is free.

\item Rename the file ``linkmake'' to linkmake.cmd so it can be executed.

\item If you have access to several CPUs you can test OC with
  parallelization using Open MP.  In that case you should use the
  linkfile ``linkpara'' (after renaming it to linkpara.cmd) on

\item Open a terminal window.  If you do not know what is a terminal
  window you should ask a local expert.  Keep him or her with you
  until you finished the installation.

\item In the terminal window exectute the file you just renamed by
  typing its name.

\item {\bf If you have errors running the linkmake or linkpara command
  files please contact a local expert.}

\item For the graphics you must download and install the free GNUPLOT
  software, for example from SoureForge.

  Make sure your PATH includes the directory with the GNUPLOT program.
  If you do not know how to set your PATH ask a local expert.

\item Creating a home directory for OC
  \begin{itemize} 
    \item Create a directory called OCHOME at you home directory,
      usually\\
      ``C:${\backslash}$Users${\backslash}$yourname''.

    \item Copy the file ochelp.hlp to this directory
    
    \item Later you may also add a mcro file that you want to run
      everytime you start OC on this directory or subdirectores
      with databases or calculated results

    \item Create an environment variable for your account called
      OCHOME with the path to your OCHOME directory as value.  If you
      do not know how to create an environment variable please ask a
      local expert.
      
      Normally you have to restart your computer to have the
      environmet variable available.

% On my Swedish Windows I create a environment variable by clicking
% on the Windows icon and then type ``milj�''
% and select Redigera milj�variabler f�r kontot

    \item If you want to start the OC program from several directories
      copy also the executable to OCHOME and add the path to OCHOME
      to your \%PATH\%
  \end{itemize}

\item There is a documentation directory with several PDF files.  Read
  the ``Getting\_started'' documentation to understand how to operate
  the program.  Also look at the ``OC-macro'' documentation which
  describes several macro files which you have on a directory
  ``macro''.  These are useful to test the software and gives some
  ideas how to use it.  There is also a news-OC4 for a summary of the
  features of OC.

\item There is also a preliminary user guide ``ochelp'' in the manual
  directory.  The user guide is also available in the file
  ``ochelp.hlp'' in LaTeX format because it is used for the on line
  help.  The user guide is still very primitive as many commands are
  changing or not yet implemented.

\item The source code is in the directories ``minimizer, models,
  numlib, stepmapplot, userif'' and ``utilities''.

\item The documentation of the source code is in the directory
  ``documentation'' in several PDF files: ``gtp3'' for the model
  package and ``hms2'' for the minimizer.  The other software
  documentation, ``smp2'' for the step/map/plot routines, ``assess''
  for the assessment module and ``ocasi'' for the application software
  interface are very preliminary.

\item The TQ4lib directory has a few examples for using the software
  interface for Fortran and C/C++.  An attempt to implement isoC
  binding to C++ has been made.

\item Contributions of new and improved source code are welcome.  You
  can do this using the github repository.  Contact Bo Sundman if you
  want to know more.

\item the command line interface has a ``VAX/VMS'' flavor which
  reflects the age of the developer.  It means the commands are
  ``verbs'' like {\em set, list, calculate, enter} etc.  After the
  verb several objects are usually possible like {\em Set
    ~condition/status} etc.  There is some redundancy so the same
  effect can sometimes be achieved by different combinations of verbs
  and objects.

  Each command and subcommand can be abbreviated, usually 3 characters
  are sufficient.

  If you want a graphical user interface you are welcome to develop
  it.
\end{itemize}

You are welcome to provide a better installation guide also!

\bigskip

{\large \bf Have fun and help make OC useful!}

\end{document}

