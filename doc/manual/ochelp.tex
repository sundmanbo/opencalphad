\documentclass[11pt]{article}
\usepackage[utf8]{inputenc}
% NEXT LINE is needed for generate a HTML file with targets (hyperref)
\usepackage{hyperref}
%
% This is a file for a printable PDF and HTML versions of the user guide
% AND as on-line help, either directly or processed to remove LaTeX specials
%
% The PDF file is generated by pdflatex ochelp6.tex
%
% I have made this user guide also available for on-line help
% as an HTML file.  That is very nice but a bit complicated.
%
%=======================================================
% To generate the HTML file use the program ``htlatex''
%=======================================================
%
% the package \usepackage{hyperref}
% creates possibilities to find text inside the HTML file
% using \hypertarget{labeltext}{shown text} (the shown text can be empty)
% written in the LaTeX file.
%
% I have written a program listhyper
% extracts relevant hypertagets from the software to help entering these
% in this user guide
%
% The idea is that the user can type a ? to have help at any question.
% All questions by the OC software is asked by a GPARxyz routine and
% in this routine the programmer (me) can provide a text for a hypertarget.
% When the user types a ? this hypertarget is used to find a relevant
% help text in the HTML version and this is displayed in a separate
% browser window.  The user can the scroll and search
% the whole manual in this window
%
% Some additional problemns with this explained below ...
%
\topmargin -5mm
\oddsidemargin -1mm
\evensidemargin -1mm
\textwidth 170mm
\textheight 225mm
\parskip 2mm
\parindent 3mm
% THIS WATERMARK IS REMOVED AS IT IS NO LONGER SUPPORTED
%\usepackage[firstpage]{draftwatermark}
%\SetWatermarkScale{4}
% this should allow \subsubsubsection ...
\newcommand{\subsubsubsection}[1]{\paragraph{#1}\mbox{}\\}
\setcounter{secnumdepth}{4}
\setcounter{tocdepth}{4}
%\pagestyle{empty}

% LOOK for ALERT: for checks how the source code behaves!!
%
% HYPERTEXT provides browser access to ochelp HTML:
% in a LaTeX file \hypertarget{labeltext} 
% is written in the html file as: <a id=''labeltext''/a>
% NOTE:  Case does not seem to matter, Enter or ENTER is OK
% BUT NO TRAILING SPACES!!
%
% Problem when opening the HTML file inside OC to list the help text:
% OLD: This is in the metlib3.F90 file, the q1help subroutine
% This is in the metlib4.F90 file, the q4help subroutine
% on WINDOWS: SPECIAL CHARACTER PROBLEM with    "   (doublequote)
% to find an labeltext from OC help file using system command:
% "C:\program files\mozilla firefox\firefox.exe" -file ./manual/ochelp/html/ochemp5.html#labeltext
% the above does not work (on Windows 10) BUT the following command works:
% "C:\program files\mozilla firefox\firefox.exe"  "file://c:\users\bosse\documents\oc\oc\src\manual\ochelp\html\ochemp5.html#labeltext"
% BUT it does NOT work when called through execute_command_line(command)
% It seems a problem having  4 doublequotes in the same line ... why?
% BUT AGAIN, using ``8.3'' Windows file names for Explorer this works:
% C:\progra~1\intern~1\iexplore.exe ``file:/C:\user\bosse\ochome\ochemp.html#labeltext
% on Linux using firefox this works:
% /usr/bin/firefox "file:/home/bosse/ochome/ochelp.html#labeltext"
%
% USE OF HYPERTARGET
% From OC version 5+ a modified online help system using \hypertarget{}{}
% Now the hypertaget is specified in PMON and the GPARxyz routinines
% replacing the subroutine name!
% This works very nicely.  The only problem is with menu and submenu commands
% because these should just list the menu when the user types ?
% and display the user guide only when the user types ??
% But this seems to work now using a special '?TOPHLP' as ``target''
% when using a gparxyz routine to ask for a command or subcommand.
%
% I am not sure if the hypertext targets are case sensitve but I assume so
% and thus this has to be standardized.  First letter capitalm rest lower case
% unless there is a reason, for example TDB is in upper case.
% the listhyper will now detect missing targets in UG
% but there are also questions with targets in the gtp routines ...
%
% NOTE: To make it easier to find all \hypertargets{}{} used in the
% source code I have added a question mark ? as first character of
% the argument in the gparxyz file, for example:
%
% call gparcx('Symbol name: ',cline,last,1,name1,' ','?Amend something')
%
% which will search for the \hypertarget{Amend symbol}{} in the HTML file
% The ? will be stripped before searching.  The ? is not mandatory, just
% an attempt to simplify updates and maintaining the help system.
%
%---------------------------------------------------
%
% The current version of this guide is generated manually but maybe a
% software program should be developed to update this automatically whenever
% the software is changed
%
%---------------------------------------------------
%
% SOME IMPORTANT ADVICE:
%
% The commands and subcommands are arranged alphabetically
%
% It will be difficult to update the help text for the
% questions after the commands and subcommands as they are sometimes
% not part of the command monitor.
%
% The _ used in many commands should be replaced by \_ or just -
%
%---------------------------------------------------
%
% GUIDE how to add searchable items in the help file for online help
% ^^^^^^^^^^^^^^^^^^^^^^^^^^^^^^^^^^^^^^^^^^^^^^^^^^^^^^^^^^^^^^^^^^
% The online help is based on using a browser like firefox or explorer
% If the user types ``?'' as answer to a question the subroutine
% asking the question, GPARxyz, has a character variable as its last
% argument which is a ``hypertarget'' in the HTML version of
% the user guide.  The HTML file opened with this character as a hypertarget
%
% htmlhelp=trim(ochelp%browser)//' "file:'//&
%              trim(ochelp%htmlfile)//'#'//ochelp%target(1:kk)//'" &'
%
% where browser and htmlfile are set when the program starts. kk is used
% to strip the target of irrelevant parts which may be set by the user.
%
% The user guide will open in a separate window at a text indicated by
% the hypertarget which should be relevant for the question.  The user can
% scroll the guide in this window for other parts if he needs more help.
%
% Special care is taken for menu commands and for these a single ?
% will just display the menu.  Only a ?? will open the HTML file.
% The hypertarget for menu help is the first 3 words of the promt for
% the menu.  Thus any variable part of the promt should be preceeded
% by at lset 3 words, for example:
%
% Amend for phase LIQUID what? /COMPOSITION_SET/: ??
%
% will use as hypertarget ``Amend for phase'' ignoring the variable ``LIQUID''
% 
% Updating the software and the user guide in parallel can be complicated.
% To help with this I have added a ? at each searchable hypertarget in the
% source code.  Thus one can search for ``'?'' to find a hypertarget for
% any question and check it is identical to the hypertarget in the
% user guide. UPPER or lower case does not seem to matter.
% But trailing spaces inside \hypertarget matter!
%
%---------------------------------------------------
%
\begin{document}

\begin{center}

{\Huge \bf User Guide to the 

OpenCalphad software package

version 7.0

}

\bigskip

{\Large DRAFT}

\bigskip

Bo Sundman, \today

\end{center}

\vspace{25mm}

Updates of OC User Guide
\begin{itemize}
\item version 7, 2021-04-01 prerelease
\item version 6, 2019-11-11 prerelease
\item version 5, 2018-09-19 first use of hypertargets
\item version 4, 2016-10-06
\item version 3, 2016-01-01
\end{itemize}

Earlier versions of OC had no User Guide

\newpage

This page intentionally blank

\newpage

\tableofcontents

\newpage

\section{Introduction}

The development of the OpenCalphad (OC) sofware was started by a small
group of dedicated scientists who wanted to provide an open source
multicomponent thermodynamic software.  It aims to provide a free high
quality software for thermodynamic calculations, including property
and phase diagrams, assessment of databases and a thermodynamic
library for simulations for inorganic systems i.e.  gases. liquids,
alloys and other materials using many different kinds of models for
the phases.  There are three basic papers published about
OC~\cite{15Sun1,15Sun2,16Sun}.  General information about
thermodynamic models, calculations and assessments based on the
Calphad technique can be found in the book by Lukas et
al~\cite{07Luk}.  This software is provided free with a GNU GPL
license.

In OC there is also a framework to store different kinds of materials
properties that depend on temperature, pressure and composition when
such properties are related to the phases of the system and used in
simulations as described in~\cite{20Her}.  The OC software can also be
used to assess model parameters for such properties from experimental
and theoretical values.

Complimentary (and maybe sometimes contradictory, I am not perfect)
information about the OC software can be found in getting-started.pdf,
news-oc7.pdf and the other parts of the OC documentation.

\section{Some general features}

The different parts of the OC software are documented separately for
each module: thermodynamic models (GTP), equilibrium calculations
(HMS), step/map/plot routines (SMP) and the application software
interface (OCASI/TQ).  With OC version 6 the old utility package
metlib, originally written in F77, has been converted completely to
the new Fortran standard and is included in the documentation.  The
documentation of the assessment module is not finished.

OC uses the free numerics packages LAPACK and BLAS and two routines
from MINPACK~\cite{lmdif}, LMDIF and HYBRD developed at Argonne 1980.
LMDIF is a least square minimizer used for assessments and
HYBRD~\cite{lmdif} solves systems of non-linear equations needed to
calculate $T_0$ and paraequilibria.  For graphics OC generates a
command file which can be plotted with the free GNUPLOT~\cite{gnuplot}
software.  If GNUPLOT is properly installed GNUPLOT is invoked
automatically by OC.

\subsection{Command line user interface}

OC is operated by commands typed by the user or read from a macro
file.  The command monitor has a menu of command and each of these
usually has sub-menus and finally some questions may be asked like
phase names, a value or an expression.  In most cases a default answer
is provided which can be selected by just pressing the RETURN key or
by typing a comma, ``,'', on the same line as the command.  At all
levels the user should be able to type a ? and get some help, usually
an extract from this manual, sometimes just a menu or examples of
answers.

A command line interface is superiour when it comes to enter complex
equilibrium conditions for example to calculate the minimum of a
liquidus line defined by the condition ``x(liq,cr)-x(bcc,cr)=0'' in
the Fe-Cr system.  To follow a second order transition one can set the
difference between the site fractions of the same element, for example
``y(bcc-B2,Al)-y(bcc-B2,Al\#2)=0.01'' as condition.

For the menu commands a single ? will just display the menu, in order to
obtain the User Guide type two, ??.

If you prefer a graphical user interface (GUI) there is at least two
independent efforts to provide a GUI to OC.

\subsubsection{Command line editing and history}

On Windows the OS provides history and on-line editing of commands but
on Linux and other OS this has to be provided by the software itself.
Thus a C routine with an iso-C interface written by Urban S Jost
(2009) copied from
http://www.urbanjost.altervista.org/LIBRARY/libCLI/Getkey/getkey.html
has been added and there is a seperate documentation of this if you
want to change anything.

The command history is saved inside OC and by typing ``upparrow''
(normally ctrl-P but it can be different on different terminals)
earlier command can be retrieved and also edited.

\subsubsection{Popup window for read/save}\label{sc:popup}

To open a file for reading or saving one need a file browser and from
OC version 5.018 I have included a routine ``TINYFILEDIALOGS''
developed by Guillaume Vareille (2014-2018) available at
http://tinyfiledialogs.sourceforge.net.  This will open a popup window
to open a file (for a macro, a database or to save a calculation).  In
this window you can browse your directories to find the file.

This has some consequencies for editing your macro files which you
should be aware of and which are explained below.

You can turn off the open file popup window feature with the command
{\bf set~advanced~open\_popup\_off~Y}.  You can turn it on again with
the same command finishing with anything but Y.

\begin{itemize}
\item The directory where you start the session with OC is called the
  ``working directory''.  On a linux system you can find this
  directory by typing ``pwd'' before starting OC (or if you type {\em
    @pwd} inside OC).  On a Windows system you can see the working
  directory and its files if you type {\em @dir} inside OC.

\item When the popup window is opened the directories and files
  matching the ``filter'' in the working directory should be listed.
  If not you can select a directory inside the popup window.  The
  filter when open a macro file is ``OCM'' and when opening a database
  file it is ``TDB'' which means only files with these extensions are
  listed.  You can change the directory in the popup window to select
  the file you want and you can read a file with another extension.
  OC will save internally the directory where you start the macro.
  
\item Inside a macro file you normally read a TDB file and if you do
  not specify the name of the database on the same line as the command
  {\em read tdb} the popup window will open so you can specify the
  file in this window.

\item But normally you know which database you want to use inside the
  macro and if you give the file name on the same line as the commad:
  {\em read tdb filename} the popup window will not open and OC will
  search for the specified database file starting from the ``working
  directory''.  But if the database file is in the same directory as
  the macro file you MUST prefix ``filename'' with ``./'', i.e. {\em
    read tdb ./filename}.  You may include directories in
  ``filename'', (including ``../'' to go to the directory above).  OC
  will replace the ``./'' by the directory where you started the macro
  or prefix ``../'' by this directory.

\item In the macro file you can give the full path to the file to be
  opened but that is rather clumsy.

\item When you open a file for write inside a macro, like output from
  a plot, you can also specify the file name in the command prefixed
  by ``./'' if you want to save the file on the same directory as the
  macro file.  Otherwise it will be saved at the working directory.

\item If you use the switch ``/output='' or ``/append='' after a
  command to redirect output from the command you can also use the
  popup window to specify the file name or use a filename with or
  without the prefix ``./''.  The default extension in this case is ``DAT''.

\end{itemize}

Opening files on different directories can be complicated inside OC.
For example during assessments you may use many different files for
generating graphics and unformatted save files.  Preferably you keep
all of these on the same directory.

You are welcome to provide feedback on this popup feature and other
parts of the user interface.

\hypertarget{Info helpsystem}{}
\subsubsection{On-line help}\label{sc:on-line-help}

A recent feature added to OC is providing on-line help using a browser
window where this user guide is available as a searchable HMTL file.

Whenever the user wants an explation of a question the OC software
asks he can type a ? and the OC software will open a separate browser
window positioned at the relevant text in the user guide.  You can then
search the whole user guide for related information.

Whenever the user types ? at a menu level just the menu will be
displayed but if you type ?? the user guide will be opened at the
relevant menu text with additional explanations.

This feature is new and is still under development. Feedback is
helpful.  It can be turned off (or on again) by the command {\bf
  set~advanced~help\_popup\_off~y} in section~\ref{sc:help-popup}.

For installation of the help system please read the installation guide
to create an environment variable OCHOME with a link to the directory
with the help file.

\subsubsection{Environment and startup macro file}

The OC program will look for an environment variable called OCHOME and
if it finds this it will look for a file start.OCM which will be
executed before the user gets control.  This can typically be useful
to set some variables like the plot terminals, see
section~\ref{sc:gnuterm}.  If there is no OCHOME environment variable
the current ``working directory'' will be searched.

The ochelp.tex and ochelp.html file should be copied from the
directory ``manual'' in the installation directory to this OCHOME
directory.

\subsubsection{Macro files}\label{sc:macro}

The macro command is very useful for preparing complex calculations
and to remember how you did them.  A macro file is simplest to create
staring from a log file (created by the {\bf SET LOG} command).  See the
macros directory for examples.

After a macro command the popup window will allow you to search for
the file on all your directories unless you type the name of the
file on the same line.  In the latter case the macro file must be
on you ``working directory'', see section~\ref{sc:popup}

When you open files, such as databases, inside a macro file and you
type the file name on the same line as the command as ``read tdb
./steel1'', you must prefix the file name, ``steel1'' with ``./'' if
the tdb file is on the same directory as the macro file.  If your
command line is just ``read tdb'' the popup window will be activated
and you can specify the file there.

If you open another macro file inside a macro (typically when you do
assessments) you must also prefix the name of the macro with ``./''
unless you want to select the macro using the popup window.

\hypertarget{Macro comments}{}
\subsubsubsection{Comments, stops and questions in macro files}\label{sc:macroadd}

It is useful to insert comments in the macro file to explain what it
doing.  A line starting with ``@\$'' is a comment and will be ignored
by the OC software.

You can insert stops in the macro file with ``@\&'' at the beginning
of a line.  This can be useful to have time to inspect the output.
The macro continues after pressing the ENTER/RETURN key.  Depending on
the graphical driver you use the program will normally pause after
each plot and you must click on the graphical window to continue.

You can also, inside the macro, ask the user for values needed for the
calculations.  For example if you have a complicated calculation you
would like to use several times with different values of the
compositions or temperature you can, instead of editing the macro each
time, insert questions in the macro.  The macro will then stop and ask
the user to input that value from the keyboard before continuing.  In
the macro file you can ask for the condition on the temperature in
this way:

\begin{verbatim}
  @$ Ask user for the condition on the T
  set cond T
  @?Input-new-temperature

\end{verbatim}

When the macro comes to this point the program will write the text
``Input-new-temparature'' on the screen and wait for user input.
After the value has been typed on the keyboard and ENTER/RETURN
pressed OC will set the value as the temperature and continue with the
next command in the macro file.

There is no way to insert loops or conditions in the macro file.

A macro file should be terminated with the command {\bf SET
  INTERACTIVE} which gives back control to the keyboard (or the
calling macro file) otherwise the program may terminate at the end of
the macro.

Macro files can be nested 5 levels deep.

\subsubsection{User interface feedback}

OC has grown organically and although the basic concepts has been
quite clear the implementations of several of these has become rather
confusing.  This will eventually require some cleaning up of the user
interface.

A central part of any thermodynamic software is the modeling of the
phases.  A new PDB format for databases may help a little with the
specification of the models.  An attempt has been made in this version
to clean up the way a model is specified and used.  At present you
must first ENTER the phase to give a name, basic model, sublattices
and constituents.  Then use the AMEND command to add magnetism, a
disordered fraction set and/or use BCC/FCC permutations.  Originally
some of these things were set by the command SET PHASE ... BIT and
that was not very clear.

Some computational options like for the grid minimizer are still set
with several different commands.  It is useful for the developers to
have some feedback from users to organize this better.

\subsection{Names and symbols}

There are many symbols and names used in this package.  A symbol or
name MUST start with a letter A-Z.  It usually can contain digits and
the underscore character after the initial letter.  All names are CASE
INSENSITIVE, i.e. fe, FE, fE and Fe is the same.  Some special symbols
are used:

\begin{itemize}
\item /- is used to denote the electron. /+ or /- -1 can be used for a
  positive charge.
\item * can be used to mean ``all'' or ``all stable''.
\item \# are used to identify composition sets after a phase name or
  sublattice after a constituent name.  It is also used as wildcard
  to obtain the DGM of all phases including metastable ones.
\item \& are used in some parameter identifiers to specify the
  constituent for the parameter, like for mobilities, the mobility of
  Fe in the BCC phase is denoted MQ\&FE(BCC).
\end{itemize}

A name of an element is one or two characters, a species maximum 24
characters (note that a species name does not have to be its
stoichiometric formula).  A phase name is 24 characters but can also
have a pre- and suffix 4 characters long and possibly a composition
set number after a hash symbol, \#.

State variable symbols and TP-fun symbols can be 16 characters long.
TP-funs are expressions used to describe the $T$ and $P$ dependence of
model parameters.

For user input it is possible to use abbreviations of names but you
must be careful with names that have the same abbreviation and avoid
phase names that are abbreviations of another phase!

\hypertarget{Info elements}{}
\hypertarget{Info species}{}
\subsection{Elements, species, components, constituents and system}\label{sc:elements}

Much of the confusion using thermodynamics is due to the fact that the
user has no clear idea of the terms in the title of this section.  A
strict definition used in OC is:
\begin{itemize}
\item An element is from the periodic chart.  The user can also enter
  fictitious elements.
\item A species is a molecularlike aggregate of elements with fixed
  ratios.  It can also have a charge and be called an ion.  The
  vacancy, representing an empty lattice site, is also a species.
\item The constituents of a phase is a subset of the species.
\item The set of components limits the composition of the system.  By
  default the elements are the components but the user can enter any
  orthogonal set of species as components by a command, see
  section~\ref{sc:amendcomp}.
\end{itemize}

A system is defined by its components.  Conditions on the amounts or
chemical potentials can only be set for the components, not for any
arbitrary species.  But the chemical potential of a molecule is
related to that of the elements at equilibrium.  Thus one can use the
relation:
\begin{eqnarray}
\mu_{\rm H_2O} &=& 2\mu_{\rm H}+\mu_{\rm O}
\end{eqnarray}
to set a condition on a sum of chemical potential of the elements.

The phases can gave different models and sets of constituents to
describe Long Range Ordering (LRO) and Short Range Orderng (SRO).
Some phases can exist for a specific composition only or for a limited
subset of the components of a system.

\hypertarget{Info phases}{}
\subsection{Phases, composition sets and phase tuples}

Many come across thermodynamic calculations the first time in
chemistry writing chemical reactions.  In such reactions the solid and
liquid phases are usually treated as stoichiometric and only the gas
can have several constituents.  In the Calphad approach most phases
are treated as solutions with variable composition but with different
models for their Gibbs energy functions.  But some phases can exist
only for a specific or very restricted composition.

Each phase in a system has a name and a thermodynamic model and set of
constituents, see section~\ref{sc:elements}.  The models are explained
in a separate documentation.  The phases can be entered interactivly
or read from a database or a saved file together with the last
calculation.

\hypertarget{Info compset}{}

In some cases a phase can be stable with two ore more different
compositions for example inside miscibility gaps or when the phase has
order/disorder transitions.  In such a case you use a composition set
index to separate these.  The composition set index is appended to the
phase name preceeded by a hash ``\#'' character, like liquid\#2.

Composition sets can be created manually, see the command {\bf AMEND
  PHASE} in section~\ref{sc:amend_phase_cs} or automatically by the
grid minimizer or application software.

The phase tuple has been introduced to have a single index for both
phases and composition sets in application software.  The tuple index
thus contain both the phase number and the composition set index.  The
array of tuple indices is updated internally whenever a new
composition set is created or deleted.

\subsection{The use of wildcards for phase names}

In many cases you can use an asterix ``*'' as a name and this normally
means ``all''.  For setting status of phases you can use the special
``*S'' for all suspended phase, ``*D'' for all dormant phases.  If you
plot the composition of a phase, such as x(liquid,*), values will be
listed or plotted only in the range the liquid is stable.

When using ``*'' for output, for example NP(*) for the amount of all
phases it means ``all stable''.  Thus to plot the driving force for
metastable phases, see section~\ref{sc:dgm}, there is a special
wildcard ``\#'' which can be used in in DGM(\#) for plotting the driving
force for all metastable phases.  The driving force, DGM, is also
included in listing of results for all phases.

\hypertarget{Info statevariables}{}
\subsection{State variables}\label{sc:statevar}

\begin{table}[!ht]
\caption{A preliminary table with the state variables and their
  internal representation.  Some model parameter properties are also
  included.  The "z" used in some symbols like Sz means the optional
  normalizing symbol M, W, V or F.  There is some redundancy, for
  example NM(FE) is the same as X(FE).}\label{tab:statev} {\small
\begin{tabular}{|lllcccl|}\hline
Symbol~&\multicolumn{2}{c}{Id}&\multicolumn{2}{c}{Index}&Normalizing~&Meaning\\
      & A & z  & 1 & 2                     &  suffix     & \\\hline
\multicolumn{7}{|c|}{Intensive properties}\\\hline
T      & 1 &- & -         & -    & - & Temperature\\
P      & 2 &- & -         & -    & - & Pressure\\
MU     & 3 &- & component & -/phase  & - & Chemical potential\\
AC     & 4 &- & component & -/phase  & - & Activity\\
LNAC   & 5 &- & component & -/phase  & - & LN(activity)=MU/RT\\\hline
\multicolumn{7}{|c|}{Extensive and normallized properties}\\\hline
U      & 6 & 1 & -/phase\#set & - & - & Internal energy for system\\
UM     & 6 & 2 & -/phase\#set & - & M & Internal energy per mole\\
UW     & 6 & 3 & -/phase\#set & - & W & Internal energy per mass\\
UV     & 6 & 4 & -/phase\#set & - & V & Internal energy per m$^3$\\
UF     & 6 & 5 & phase\#set   & - & F & Internal energy per formula unit\\
Sz     & 7 & * & -/phase\#set & - & * & entropy\\
Vz     & 8 & * & -/phase\#set & - & * & volume\\
Hz     & 9 & * & -/phase\#set & - & * & enthalpy\\
Az     & 10 & * & -/phase\#set & - & * & Helmholtz energy\\
Gz     & 11 & * & -/phase\#set & - & * & Gibbs energy\\
NPz    & 12 & * &  phase\#set & - & * & Moles of phase\\
BPz    & 13 & * & phase\#set & - & * & Mass of phase\\
Qz     & 14 & * & phase\#set & - & *  & Stability of phase\\
DGz    & 15 & * & phase\#set & - & *  & Driving force of phase\\
Nz     & 16 & * & -/phase\#set/comp & -/comp & *  & Moles of component\\
X      & 17 & - & phase\#set/comp & -/comp & 0  & Mole fraction\\
X\%    & 17 & - & phase\#set/comp & -/comp & 100 & Mole per cent\\
Bz     & 18 & * & -/phase\#set/comp & -/comp & *  & Mass of component\\
W      & 19 & - & phase\#set/comp & -/comp & 0 & Mass fraction\\
W\%    & 19 & - & phase\#set/comp & -/comp & 100 & Mass per cent\\
Y      & 20 &- & phase\#set & const\#subl & -& Constituent fraction\\\hline
\multicolumn{7}{|c|}{Some model parameter identifiers}\\\hline
TC     & - &- &phase\#set & - & - & Curie temperature\\
BMAG   & - &- & phase\#set & - & - & Aver. Bohr magneton number\\
MQ\&A  & - &- & phase\#set & constituent A & - & Mobility of A\\
THET   & - &- & phase\#set & - & - & Debye temperature\\\hline
\end{tabular}
}
\end{table}

A state variable in a thermodynamic system has a value which at
equilibrium is independent of the way the system has reach its current
state, it depends only on its current state.  All state variables
available in OC are listed in Table~\ref{tab:statev}.  They are used
to set conditions and to obtain results from an equilibrium
calculation.  It is possible to use state variables also when close to
the equilibrium state for example when simulating a phase
transformation.

\subsubsection{Some pecularites of the state variable values}

One has to be careful with the normalizing suffix, thus H means the
enthalpy of a system for its current size.  HM is the enthalpy for the
current system divided by the number of moles of atoms in of the
system.  This is one can expect but one may be surprised that H(phase)
is the enthalpy of ``phase'' for the current amount of moles of atoms
of the phase, which is zero if the phase is not stable.  To obtain the
value of the enthalpy of ``phase'' independently of its current amount
one must use HM(phase), the enthalpy per mole of atoms in the phase.

The value of a state variable also depend on the reference states of
the elements.  The user may define this for each element with a
command, see section~\ref{sc:refstate}.  The default refernce state is
the stable state of the elements at 298.15~K and 1~bar, called SER.
Whenever necessary this is indicated by an final suffix ``S'', for
example ACS(C) indicate the activity of C using the reference state
SER whereas AC(C) is always be the activity relative the current
reference state, either the default or that set by the user.

If all elements have the same phase as reference then the integral
properties will also be refered to that phase, they wll represent an
``excess''.  If the elements in a system have different reference
phases the integral value of the state variable will normally be
relative to SER because anything else would be meaningless.

\hypertarget{Info dgm}{}
\subsubsection{The driving force}\label{sc:dgm}

Most state variables have a welldefined thermodynamic meaning but the
driving force, DGM(phase), is a property related to the stability of
the phase at an equilibrium.  All stable phases are on a common
tangent planne of chemical potentials and have DGM=0.  For a
metastable phase the value of the DGM variable is the distance in
Gibbs energy (normallized by dividing it by the value of $RT$) between
the stable tangent plane and the point on the Gibbs energy surface of
the metastable phase that is closest to the tangent plane of the
stable phases.  DGM is negative for a metastable phase and if is close
to zero it means the phase is close to become stable.  The only case a
phase can have positive DGM is for phases which have the dormant
status and it means the phase would be stable if its status is changed
to be entered.

\hypertarget{Info databases}{}
\subsection{Thermodynamic databases}\label{sc:databases}

The use of thermodynamic software depend on assessed model parameters
for phases and elements.  With the OC software one can make
assessments of such model parameters using experimental and
theoretical data, see section~\ref{sc:assess}.  However, this user
guide does not describe the construction of such databases or how one
can obtain them.

\subsection{Model parameters}

All data is organized relative to a phase and the phase is identified
by a name.  Each phase can have a different model for the composition
dependence but the way to enter model parameters is the same for all
models.  However, the meaning of a model parameter will depend on the
model of the phase.

Many types of data can be stored as explained in the section on
parameter identifiers.  The parameter also has a constituent
specification explained in the constituent array section and possibly
a degree, the meaning of which is model dependent and a bibliographic
reference.

The basic syntax of a parameter is

``identifier'' ( ``phase name'' , ``constituent array'' ; ``degree'' ) ``expression'' ``bibl.ref.''

These parts are explained in more detail below.

\subsubsection{Model Parameter Identifiers}\label{sc:paramid}

The OC thermodynamic package can handle any phase property that depend
on $T, P$ and the constitution of the phase using the models
implemented.  It is easy to extend the number of properties by
declaring property identifiers in the source code.  If the parameters
should have an influence on the Gibbs energy (like the Curie
temperature) or a diffusion coefficient (like the mobility) the
necessary code to calculate this must be added.

A list of the model parameter identifiers as shown in
Table~\ref{tab:mpis} can be obtained by the command {\bf LIST
  MODEL-PARAM-ID}

\begin{table}[!h]
  \caption{Current set of model parameter identifiers}\label{tab:mpis}
  {\small
\begin{verbatim}
Indx Ident T P Specification                Status Note
   1 G     T P                                   0 Energy
   2 TC    - P                                   2 Combined Curie/Neel T
   3 BMAG  - -                                   1 Average Bohr magneton numb
   4 CTA   - P                                   2 Curie temperature
   5 NTA   - P                                   2 Neel temperature
   6 IBM   - P &<constituent#sublattice>;       12 Individual Bohr magneton num
   7 THET  - P                                   2 Debye or Einstein temp
   8 V0    - -                                   1 Volume at T0, P0
   9 VA    T -                                   4 Thermal expansion
  10 VB    T P                                   0 Bulk modulus
  11 VC    T P                                   0 Alternative volume parameter
  12 VS    T P                                   0 Diffusion volume parameter
  13 MQ    T P &<constituent#sublattice>;       10 Mobility activation energy
  14 MF    T P &<constituent#sublattice>;       10 RT*ln(mobility freq.fact.)
  15 MG    T P &<constituent#sublattice>;       10 Magnetic mobility factor
  16 G2    T P                                   0 Liquid two state parameter
  17 THT2  - P                                   2 Smooth step function T
  18 DCP2  - P                                   2 Smooth step function value
  19 LPX   T P                                   0 Lattice param X axis
  20 LPY   T P                                   0 Lattice param Y axis
  21 LPZ   T P                                   0 Lattice param Z axis
  22 LPTH  T P                                   0 Lattice angle TH
  23 EC11  T P                                   0 Elastic const C11
  24 EC12  T P                                   0 Elastic const C12
  25 EC44  T P                                   0 Elastic const C44
  26 UQT   T P &<constituent#sublattice>;       10 UNIQUAC residual parameter
  27 RHO   T P                                   0 Electric resistivity
  28 VISC  T P                                   0 Viscosity
  29 LAMB  T P                                   0 Thermal conductivity
  30 HMVA  T P                                   0 Enthalpy of vacancy form.
  31 TSCH  - P                                   2 Schottky anomaly T
  32 CSCH  - P                                   2 Schottky anomaly Cp/R.
  33 NONE  T P                                   0 Unused
\end{verbatim}
  }
\end{table}
 
Several of these identifiers have no supporting software implemented,
this is an ongoing project.  The columns T P indicate if the parameter
may depend on $T$ or $P$.  Some identifiers require additional
specification of the constituent and sublattice, like the mobility of
a constituent.  Currently it is not yet clear if mobilities should
depend on the sublattice or not but the notation allows that.

A slightly more detailed explanation of the identifiers are:

\begin{itemize}
\item G, the Gibbs energy parameter for an endmember or an
  interaction.  G(LIQUID,FE;0) is the Gibbs energy for pure liquid Fe.
  Note that the parameter will be used also below the melting
  temperature of Fe for a liquid phase containing Fe.  G(LIQUID,CR,FE;0)
  is the regular parameter for Cr and Fe in the liquid.
\item TC, a parameter for the critical temperature for ferro or
  antiferro magnetic ordering using the Inden model.
\item BMAG, a parameter for the average Bohr magneton number using
  the Inden model.
\item CTA, a parameter for the Curie temperature for ferromagnetic
  ordering using a modified Inden model.
\item NTA, a parameter for the Neel temperature for antiferromagnetic
  ordering using a modified Inden model.
\item IBM\&C, a parameter for the individual Bohr magneton number for
  constituent C using a modified Inden model.  For example
  IBM\&FE(BCC,FE) is the Bohr magneton number for BCC Fe.  The
  identifier IBM\&FE(BCC,CR) means the Bohr magneton number of a
  single Fe atom in BCC Cr.  An identifier IBM\&FE(BCC,CR,FE) can be
  used to decribe the composition dependence of the Bohr magneton
  number for Fe in BCC.
\item THET, a parameter for the Debye or Einstein temperature.
\item V0, a parameter for the volume at 298.15~K and 1 bar.
\item VA, a parameter for the integrated thermal expansion.
\item VB, a parameter for the Bulk modulus.
\item G2, a parameter for the two-state liquid model.
\item LAMB, a parameter for the thermal conductivity.
\item MQ\&C, a parameter for the logarithm of the frequency factor of
  the mobility of constituent C.
\item MF\&C, a parameter for the activition energy of the mobility of
  constituent C.
\item MG\&C, a parameter for the magnetic factor of the mobility of
  constituent C.
\item THT2, The T for a smooth change of C$_P$ 
\item DCP2, The value of the smooth change in J/mol
\item VISC, a parameter for the viscosity.
\item LPX, a parameter the lattice parameter in X direction.
\item LPY, a parameter the lattice parameter in Y direction.
\item LPZ, a parameter the lattice parameter in Z direction.
\item LPTH, a parameter the angle between lattice directions.
\item EC11, a parameter for the elastic constant C11.
\item EC12, a parameter for the elastic constant C12.
\item EC44, a parameter for the elastic constant C44.
\item UQT\&C, a parameter for the UNIQUAC residual energy for species C
\item RHO, a parameter for the electrical resistivity.
\item HMVA, a parameter for the enthalpy of vacancy formation.
\item TSCH, the T for a Schottky anomaly.
\item CSCH, the Schottky anomaly $\Delta C_P$.
\item QCZ, the bond number in the FactSage quasichemical model.
\end{itemize}

The current value of any of these parameter identifiers can be obtaind
by the command {\bf LIST STATE\_VARIABLE} using the identifier and
appropriate phase and component specifiers, see
section~\ref{sc:list_statevar}.

For details of the meaning of the model identifier refer to the model
documentation.  As already mentioned many of the identifiers, like the
mobility, does not influence the Gibbs energy but as they depend on
the $T, P$ and constitution of the phase it is convenient to model
them in the same way as the thermodynamic data.

\subsubsection{Constituent array and degrees}

A constituent array specifies one or more constituent in each
sublattice.  A constituent must be entered as a species with fixed
stoichiometry.  Between constituents in different sublattices you must
give a colon, ":", between interacting constituents in the same
sublattice you must give a comma, "," or a space.  A constituent array
with exactly one constituent in each sublattice is also called an
``endmember'' as it give the value for a ``compound'' with fixed
stoichiometry.  Constituent arrays with one or more interaction
constituents describe the composition dependence of the property.
Without such parameters the property will vary linearly between the
endmembers.

If there are no sublattices, like in the gas, you just give the phase
and the constituent

G(GAS,C1O2)

If no degree is specified it is assumed to be zero.  For endmembers
the degree must be zero but it may sometimes be useful to specify the
zero in order to distinguish the parameter from the expression for the
calculated value of the property, like the chemical potential of a
component.  In the gas phase you normally assumes there are no
interactions but it is possible to add such parameters.  For an fcc
phase with 4 sublattice for ordering and one for interstitials an
endmember parameter is

G(FCC,AL:NI:NI:NI:VA;0)

This would be the Gibbs energy of an fcc AL1NI3 ordered compound.

An interaction between vacancies and carbon in the austenite is

G(FCC,FE:C,VA;0)

For an interaction parameter you should always specify a degree but
also in this case an omitted degree is interpreted as zero.

\subsubsection{Ternary extrapolations}\label{sc:excessparameters}

The main binary excess model implemented in OC is the symmetric binary
Redlish-Kister method combined with the Muggianu ternary
extrapolation.  Other binary methods, such a polynomial or Legendre
polynom can always be converted to a set of Redlich-Kister parameters.

\begin{eqnarray*}
L_{\rm A,B} = \sum_{\nu=0}^n  ~^{\nu}L_{\rm A,B} (y_{\rm A} - y_{\rm B})^{\nu}
\end{eqnarray*}
where the degree, $\nu$, of the interaction parameter is specified
after a semicolon, L(phase,A,B;$\nu$).

For ternary parameters and for reciprocal parameters the Hillert model
for composition dependence is implemented, see~\cite{07Luk}.

You can store many different types of data in OC with different
parameter identifier.  Some of the parameters are not related to the
thermodynamic properties but as they depend on the phase, T, P and
composition it is convenient to store them together with the
thermodynamic data.  For example the mobility of Fe in BCC (including
an empty interstitial sublattice) is specified as: MQ\&FE(BCC,FE:VA).

An explanation of the identifiers implemented in OC can be found in
section~\ref{sc:paramid}.  The current list can be obtained by the
command {\bf LIST MODEL\_PARAM\_ID}.  All of them can be composition
dependent.  Some cannot depend on $T$ or $P$ or neither.  Many kinds
of the parameters are available but in some cases the software for the
models to handle them are not implemented.  The value of a model
parameter can be obtained using {\bf LIST MODEL\_PARAM\_VAL} or simply
{\bf SHOW}.  You must specify phase and endmember for the parameter.

From OC version 7 it will be possible to specify differnt ternary
extrapolation methods for each a ternary subsystem of a phase.  A
ternary subsytem in a phase may be assigned a symmetric Kohler or
assymetric Toop ternary method together with the Redlich-Kister binary
method.  See section~\ref{sc:kohler-toop2}.

\subsubsection{The TPFUN expression and bibliographic reference}\label{sc:tpfun}

The expression for a parameter can be a single value or a function of
$T$ and $P$.  It must start with a low temperature limit, usually
298.15~K and must finish with a high temperature limit.  These
expressions as well as their first an second derivatives will be
calculated by the TP-fun package.  To simplify that there is a strict
syntax for the expression.  A term in the expression is

``numeric value'' * ``name of TP function'' *T** ``power'' *P** ``power'' 

You can construct very complex expression by referring to other
functions.  If ``power'' is zero the corresponding *T** or *P** can be
omitted.  If it is negative it must be surrounded by parenthesis like
(-1).  If it is unity the **1 can be skipped.

Several terms, seperated by signs, forms an expression and it must be
terminated by a semicolon, ``;''.  After the semicolon there must be a
high temperature limit or a breakpoint in temperature.  A breakpoint
must be followed by the letter ``Y'' and then a new expression for
temperatures above the breakpoint.  

{\bf It is the responsability of the database manager to ensure the
  expression is continuous at the breakpoint.  If there are jumps in
  the value at a breakpoint strange things will happen when
  calculating equilibria.}

After the high temperature limit the letter ``N'' must be given
followed by a bibliographic reference for the parameter.  Use the
commands AMEND or ENTER BIBLIOGRAPHIC to give the reference.

{\bf The database manager should always add a bibliographic reference
  even if it is just his or her name and a date.  This avoids people
  to mistake a value inspired by your experience for a carefully
  validated parameter.}

A term can be used inside a natural logarithm, LN, or exponential,
EXP. And the LN or EXP can be multiplied with a term.  On the other
hand you are not allowed to have any parenthesis, except around powers
or arguments to LN and EXP.  A valid expression is

\begin{verbatim}
 298.15 -8856.94+157.48*T-26.908*T*LN(T)+.00189435*T**2
        -1.47721E-06*T**3+139250*T**(-1); 2180 Y 
        -34869.344+344.18*T-50*T*LN(T)-2.88526E+32*T**(-9); 6000 N 91Din
\end{verbatim}

where 91Din is the bibliographic reference to the SGTE unary database.

\subsection{The reference state of a component}\label{sc:refstate}

The values of most thermodynamic data must have a defined reference
state.  By default the reference state for the components is SER
(Stable Element Reference) which is the stable state of the element at
298.15~K and 1~bar.  (NOTE: the default reference state is defined by
the database but today almost all databases have SER as reference
state.)

For each component (also for other components than the elements) you
can specify a phase at a given temperature and pressure as reference
state, see section~\ref{sc:setref}.  The phase must exist for the
component as pure.

A state variable like the chemical potential, MU(O), will refer to the
user defined reference state if set.  To obtain the value for the SER
state you can use a suffix S, i.e. MUS(O) to obtain the chemical
potential refered to SER.  All state variables are listed in
Table~\ref{tab:statev}.

Note that the value of integral properties like Gibbs energy, $G$,
enthalpy, $H$, etc. may have mixed reference states unless all
components have the same phase as reference state.  In order to have
the enthalpy of mixing of a phase all components must have that phase
as reference state.  For the volume, $V$, SER is always used as
reference state unless all components have the same reference state.

\hypertarget{Info equilibrium}{}
\subsection{Equilibrium calculations}

The basic application of OC is to calculate the equilibrium of a
system as described in section~\ref{sc:calceq}.  The user can
specifying the external conditions like $T, P$ and the composition,
see section~\ref{sc:calceq}..  The minimizing algorithm~\cite{15Sun2}
use Lagrangian multiplier so many different sets of state variables
can be used for specifying the external conditions. Each condition is
set separately and it is possible to extract phase amounts and
compositions after the calculation.  By changing the status of the
phases it is possible to calculate metastable state.

In order to do any calculation the user must provide a database with
the model parameters for his system or enter these manually.

The conditions can also be set using the command {\em
  set\_input\_amount}, see section~\ref{sc:setinpuam}.

\hypertarget{Info propertydiagram}{}
\subsection{Property diagrams}

A property diagram is calculate with the STEP command.  First you must
set conditions to calculate a single equilibrium and then set set one
of the conditions as an axis.  After the STEP command, see
section~\ref{sc:step} you can plot how any state variable varies with
the selected axis variable.  See the section~\ref{sc:plot} and the OC
macros guide.

\hypertarget{Info phasediagram}{}
\subsection{Phase diagrams}

A phase diagram show the regions of different sets of stable phases in
a system.  It can have two or more axis variables, in OC the maximum
number of axis is two at present.  As for property diagrams you must
first calculate a single equilibrium and then select two conditions as
axis variables.  The command MAP, see section~\ref{sc:map}, will then
trace the lines in your systems where the set of stable phases
changes.  There is no limit on the number of components for a phase
diagram calculation.

After calculating a diagram you can plot it with many different types
of axis, see section~\ref{sc:plot} and the OC macros guide.

\subsection{Diagrams simulating phase transformations}

Thermodynamics is essential to simulate phase transformations but
requires good understanding also of the kinetics such as diffusion and
kinetcs.  For such applications OC has an Appication Software
Interface (OCASI) with subroutines to calculate local driving forces
and chemical potentials in various parts of a sample as described
in~\cite{16Sun,20Her}.  However, there are a few cases when one can
simplify the kinetics sufficently to use the facilities of OC
directly.

\subsubsection{Scheil-Gulliver solidification model}\label{sc:scheil1}

In a Scheil-Gulliver solidification simulation the diffusion in the
solid phases are ignored and the liquid is considered as homogeneous.
That is a realistic model for the interdendric reqion during a normal
solidification.  It can be calculated with a STEP calculation by using
small time steps and modify the overall composition to be that of the
liquid after each step.  The solid formed is removed from the system.
In such a simulation the liquid will be stable until it reaches an
invariant equilibrium, usually very far from its initital composition,
see section~\ref{sc:scheil2}.

\subsubsection{Paraequilibrium calculation}\label{sc:paraeq1}

In some alloys, most particularly in steels, there are fast diffusing
elements such as C or N which can maintain a constant chemical
potential during the whole transformation, and thus change their
composition in different phases.  The other alloying elements may
transform to a new phase without changing their fractions.  This can
be modelled as a paraequilibrium or a ``No Partitioning Local
Equilibrium'' (NPLE) situation and it requires no kinetic data.  In OC
the CALCUALATE or STEP PARAEQUILIBRIUM simulates such a
transformation, see sections~\ref{sc:paraeq2}, \ref{sc:paraeq3}.

\subsubsection{Tzero calculation}\label{sc:tzero1}

The $T_0$ point, line or rather ``hypersurface'' between two phases
are defined by $T$ where the Gibbs energy of the two phases are the
same.  Such a point is the limit of a diffusionless transformation of
one phase to the other and it is useful to understand for example the
martensite transformation.  How to calculate the ``Tzero'' point or
line are explaned in sections~\ref{sc:tzero2} and \ref{sc:tzero3}.

\subsection{Assessment of model parameters for databases}

One of the important uses of the OC software is to assess model
parameters in the phases of a system using experimental and
theoretical data.  This is done by recalculating the experimental data
from the model and by varying the model parameters a least square
routine, LMDIF developed at Argonne National Lab in 1981, is used to
find the best set.

Assessments are a very difficult procedure as you must also take into
account the extrapolations of the model outside the range of
experimental data.  So called ``First Principles Calculations'' or the
somewhat simpler ``Density Functional Theory'' (DFT) which are based
on the electronic structure of the elements can provide information
for metastable as well as for the stable state.  But you must be
careful that the result from such calculations does not represent a
mechanically unstable state with imaginary phonon frequencies.

Experimental data can be direct measurements of thermodynamic data
like enthalpies, chemical potentials, heat capacities, activities, etc
but very important are also measurements of phase diagrams,
solubilities etc because they are also related to the equilibrium
state.

There are several commands related to the assessment procedure in OC
but during the assessment you will also use the basic facilities to
calculate equilibria for different kinds of conditions as well as many
different kinds of diagrams to verify the results.

\subsubsection{Entering coefficients to be assessed}

The command ``enter optimizing coefficients'', see
section~\ref{sc:optcoeff} creates symbols A00 up to A99 that can be
used as coefficients in the thermodynamic model parameters.  Maximum
number of coefficients are 100.

\subsubsection{Entering phases and model parameters}

The elements, species and phases with their appropriate models are
entered using the appropriate commands.  Normally this is on a macro
file in order to have proper documentation.  Keep also in mind that an
assessment is often revised after a few years when new data become
available or you find that the extrapolations of an assessment to a
higher order system is not reasonable.

The model parameters are entered using ``enter parameter'', see
section~\ref{sc:enterparam} or ``enter tpfun'', see
section~\ref{sc:entertpf} as many parameters may share some properties
and a TP-function can be used in several parameters.  The optimizing
coefficents A00 to A99 with different T and P dependence can be used
instead of numerical values as their values should be assessed.

\subsubsection{Entering experimental data}

This is done either by entering single equlibria with conditions and
in addition using the command ``enter experiment'', see
section~\ref{sc:enterexp} where the experimental data is given with an
uncertainty.  Each equilibrium with an experiment is given a unique
name.

Often there are tables with values and instead of entering each of
them there is a command ``enter many\_equilibria'', see
section~\ref{sc:entermany} with a simplified syntax.

When all equilibria with experiental data has been entered you have to
give the command ``set range'', see section~\ref{sc:setrange} to give
the first and last equilibrium number that should be used in the
assessment.  If necessary this range can be extended during the
assessment.

All the experimental data should also be entered as a mcro file to
keep a documentation.

\subsubsection{Saving the state of the assessment}

Any time during an assessment it is possible to save the values of all
assessed parameters and the calculated experimental equilibria by the
command ``save unform {\em filename}'', see section~\ref{sc:saveunf}.
With this command the data inside OC will be written as an unformatted
Fortran file and this can be saved and later read back into the OC
software by the command ``read unfomatted {\em filename}'', see
section~\ref{sc:readunf}.  If these commands are inside a macro file
prefix the filename with ``./'' to read and write on the same
directory as the macro file.

These unformatted files are very convenient but beware that they may
not be portable to other operating systems or even other versions of
OC compiled with different Fortran compilers.  It may change in future
releases of the OC software.  Thus keep printouts and macro files also
if you later want to make modifications.

\subsubsection{Performing the assessment}\label{sc:assess}

There are many decisions to make during the assessment and a general
description how to perform an assessment can be found in the book by
Lukas et al~\cite{07Luk}.  It is never possible to try to assess all
parameters using all experiments in a single step.  Normally the user
selects different sets of experimental data by the ``set weight''
command, see section~\ref{sc:setw} and fits a few model parameters to
these using the command ``set variable-coeff'', see
section~\ref{sc:setvar}.  This can typically an enthalpy of mixing or
a heat capacity function for a compound.

The command to run the least square fit is ``optimize'' followed by
the maximum number of iterations, see section~\ref{sc:optim}.  If zero
is given a single loop is made through all equilibria with nonzero
weights within the specified range is made.  It is also possible to
use the command ``calculate all'', see section~\ref{sc:calcall}, to
calculate all non-zero weight equilibria.  With the latter command you
can turn on the grid minimizer, in the optimize command the grid
minimizer is always turned off.

When the optimize command is given with nonzero maximum there will be
output on the screen at regular intervals giving the current values of
the optimizing coefficients and the value of the sum of squares.  When
the oprimization is finished there will also be a listing of the
errors for all experiments.

With the command ``list opt short'', see
section~\ref{sc:listoptshort}, the current values of the optimizing
coefficients and all equiliria with the experimental data is listed
together with the sum of squares.  New selection of equilibria or
weights can be made and the values obtained for the optimizing
coefficients must also be reasonable but to know what is reasonable is
not always easy.  These steps are repeated until the user is satisfied
or exhausted.

Macro files to calculate and plot of the calculated properties
overlayed with the experimental data should be preoared and run
regularly as just looking at numbers is not sufficient.

At a later stage solubilities and phase diagram data are used but in
many cases reasonable guesses of the start values of model parameters
must be made to be able to calculate the equilibrium with the
experiment.  Great care must be taken that the calculated equilibria
for the inital model parameters are reasonably close to the
experimental.  Parts of the experimental phase diagram may have to be
assessed separately and the metastable extrapolations of the different
phases checked.

Sometimes a phase appears in a region where it should not be stable
and additional fictitious experimental data may have to be added to
prevent this to happen.

At the end the assessment should be written up and published.

\subsection{Application software}

There is a separate guide for using OpenCalphad Application Interface
(OCASI) in application software.  For such cases it is convenient to
have the source code which can be compiled together with the
applications software.  A special feature is also the possibility to
use OpenMP to calculate in parallel.

\newpage

% below the commands are documented in alphabetical order

\hypertarget{All commands}{}
\section{The command menu}

The commands in alphabetical order as listed with the ?.  The commands
with an * has subcommands.

% NOTE the ~ used for the long commands needed for the HTML output
\begin{tabular}{llll}
ABOUT           & EXIT           & MAP           & SELECT * \\
AMEND *         & FIN            & NEW           & SET * \\
BACK            & HELP           & OPTIMIZE~     & SHOW  \\
CALCULATE *~    & HPCALC         & PLOT *        & STEP * \\
DEBUG *         & INFORMATION *~ & QUIT    \\
DELETE *        & LIST *         & READ * \\
ENTER *         & MACRO          & SAVE *\\
\end{tabular}

Many of the commands have ``subcommands'' and usually OC will provide
a default answer (listed within slashes /default/) which is selected
by pressing return.  You can type commands, subcommands and other
parameters (separated by a space) on the same line if you know the
order.

To select a default when typing several commands and answers to
questions (command arguments) on the same line, you can use a
comma,``,'' to select the default answer.  For example ``l,,,,'' will
list on the screen with the current list options.

Many commands will ask additional questions, all of them are not
included in this guide but those which are will be {\bf shown in
  bold}.  Examples and references to other commands are sometimes in
{\bf bold}, sometimes in {\em italics}.

Whenever the program asks a question you do not understand you can
type a question mark, ``?'', to obtain help.  If the online help
system is correctly installed, see section~\ref{sc:on-line-help}, this
will open a browser window with this manual and hopefully position the
manual at the relevant part.  You can browse the whole manual in this
window if you need additional help.

\hypertarget{Command options}{}
\subsection{Options}\label{sc:options}

There are some options that can be set for the whole session or for just a
single command.  The options are identified by a / in front like
/output=myfile.dat.

An option must be specified directly after a command for example:

{\bf list /out=equil5 result 2}

Only a few options are implemented.

\begin{itemize}
\item /OUTPUT={\em file name} open a file and write on it.  Note that
  if you have popup windows enabled this will open unless you type the
  file name (with path) on the same line as the command.  In a macro
  file must prefix the file name with ``./'' to have the output (or
  append) on the same directory as the macro file.  See also
  section~\ref{sc:popup} and \ref{sc:macro}.
\item /APPEND={\em file name} append output to a file, any previous
  content is kept.
\item /ALL apply for all.
\item /FORCE override normal restrictions.
\item /VERBOSE write information while executing.
\item /SILENT do not write anything except fatal error messages.
\end{itemize}
%===================================================================
% The online help will not print lines starting with % or \
\hypertarget{About}{}
\section{About}

This is OpenCalphad (OC), a free software for thermodynamic calculations as
described in B Sundman, U R Kattner, M Palumbo and S G Fries, Integrating
Materials and Manuf. Innov. (2015) 4:1; B Sundman, X-G Lu and H Ohtani,
Comp Mat Sci, Vol 101 (2015) 127-137 and B Sundman et al., Comp Mat Sci,
Vol 125 (2016) 188-196

It is available for download at http://www.opencalphad.org or
the sundmanbo/opencalphad repository at http://www.github.com

This software is protected by the GNU General Public License
You may freely distribute copies as long as you also provide the source code
and use the GNU GPL license also for your own additions and modifications.

The software is provided "as is" without any warranty of any kind, either
expressed or implied.  The full license text is provided with the software
or can be obtained from the Free Software Foundation http://www.fsf.org

Copyright 2011-2021, Bo Sundman, Gif sur Yvette, France.
Contact person Bo Sundman, bo.sundman@gmail.com

%===================================================================
\hypertarget{Amend}{}
\section{Amend}

Intended to allow changes of already entered data. Only some
of the subcommands are implemented.

\begin{tabular}{llll}
 ASSESSMENT\_RESLT~ & ELEMENT      & OPTIMIZING-COEFS~ & REDUNDANT\_SETS \\
 BIBLIOGRAPHY~      & EQUILBRIUM~  & PARAMETER~        & SPECIES\\
 COMPONENTS         & GENERAL      & PHASE *           & SYMBOL \\
 CONSTITUTION~      & LINES        & QUIT              & TPFUN-SYMBOL\\

\end{tabular}

The default selection is PHASE.
%--------------------------------

\hypertarget{Amend assess result}{}
\subsection{Amend assessment result}

After assessing a set of parameters for a system each of these has a
Relative Standard Deviation (RSD) listed in the result.  Using this
RSD it is possible to modify one parameter and recalculate how much
all the other parameters should change due to this modification
without rerunning the actual assessment.

This command allows to calculate such a change and it can be tested be
reassessing the parameters using the experiments.


%--------------------------------
\hypertarget{Amend bibliography}{}

\subsection{{\em amend} Bibliography}

{\bf Reference identifier:}

The text for bibliographic reference identifier can be amended.  The
reference identifier is CASE INsensitive.

{\bf Reference text, end with ``;'':}

The text for this reference will be set to the text supplied.  It can be
several lines terminated with a ``;''

%--------------------------------
\hypertarget{Amend components}{}
\subsection{{\em amend} Components}\label{sc:amendcomp}

{\bf Give all new components:}

By default the elements are the components.  This command can set any
orthogonal set of species as components.  The number of components
cannot be changed by this command.  The new components must exist as
species and be orthogonal.  For example in the system Ca-O-Si one can
define CaO SiO2 and O as components.

The components are important as you can only use components to
specify compositions, such as x(cao)=.3 is possibly only if CaO
is a component.  See also {\bf set input-amount}~\ref{sc:setinpuam}.

Note that when you have other components than the elements you may
have negative mole fractions and phase amounts (but never negative
mass).

%--------------------------------
\hypertarget{Amend constitution}{}
\subsection{{\em amend} Constitution}

% There will be several questions ``Phase name:'' in various part of the
% software, the initial ``Amend const'' here is to locate THIS question

\hypertarget{Amend const phase name}{{\bf Phase name:}}

The program will ask for a phase name and you can set the amount
and constitution of the phase.  This will be used as initial
constitution for a calculation unless the grid minimizer is used.

\hypertarget{Amend const amount}{{\bf Amount of phase:}}

% This command should be moved to AMEND PHASE ...

{\bf Current (Y), default (D) or new (N) constitution:}

Answer Y to keep current constituion, D to set a default constitution
(if you have set such a constitution) or N to provide a new
constitution.

{\bf Fraction of component: }

You can specify a value between 0.0 and 1.0.  The sum of all
constituents must be unity, values below 0.0 or 1.0 are not allowed.
If you want the fraction of a constituent the be 1.0-(all the other
fractions) you can set its value to REST.  Otherwise the last
constituent is set to the ``rest''.

%--------------------------------
\hypertarget{Amend element}{}
\subsection{{\em amend} Element}

The data for the element can be amended, not implemented yet.

%--------------------------------
\hypertarget{Amend equilibrium}{}
\subsection{{\em amend} Equilirium}

Not sure what could be amended and anyway not implemented.

%--------------------------------
\hypertarget{Amend general}{}
\subsection{{\em amend} General}

A number of general settings can be amended by the user:

\begin{itemize}
\item The name of the system.
\item The level of the user (beginner, frequent user, expert).  This
  may affect the behavior of the program (not implemented yet).
\item If global minimization is allowed or not.
\item If the grid minimizer is allowed to merge gridpoints in the same
  phase after global minimization.
\item If the grid minimizer can automatic create composition sets is
  allowed or not.
\item If redundant composition sets can be deleted automatically after
  an equilibrium calculaion.
\end{itemize}

Note that these and some other general feautures can also be changed
by the command {\bf SET BIT GLOBAL}

%--------------------------------
\hypertarget{Amend line}{}
\subsection{{\em amend} Line}

After a STEP or MAP command it is possible to give the command LIST
LINE to list all calculated equilibria or AMEND LINE which allows you
to EXCLUDE lines or INCLUDE lines from the plotting.

{\bf Only excluded? /Y/:}

Sometimes a line may be excluded from plotting if there was an error
while it was calculated.  Answering Y will make it possble to restore
such a line and also lines you have previously excluded.

{\bf Exclude this line? /N/:}

For an included line you can exclude from the plot.

{\bf Include this line? /N/:}

For an excluded lines you can include it in the plot.

%--------------------------------
\hypertarget{Amend optim coeffs}{}
\subsection{{\em amend} All optimizing coefficients}

The values of each optimizing coefficients, see
section~\ref{sc:setrange} can be rescaled (start values set to current
values) or recovered (current values set to previous start values).

%--------------------------------
\hypertarget{Amend parameter}{}  
\subsection{{\em amend} Parameter}

The possible parameters that can be amended depend on the model of the
phase.  By specifying a parameter you can change its expression.

This is not yet implemented you must use the command {\bf ENTER
  PARAMETER} to change the parameter expression.

%--------------------------------
% we need to add these for the seach on the help file ... suck
\hypertarget{Amend for phase}{}
\hypertarget{Amend phase}{}
\subsection{{\em amend} for Phase ``phase-name''}\label{sc:amendph}

You must first specify the phase name and then you can amend some of
the properties of the phase:

If you want to amend something for a composition set you must specify
the composition set number together with the phase name after a hash
character (\#) (like liquid\#2).

{\bf Phase name:}

You must specify the name of the phase you want to amend.

{\small
\begin{tabular}{llll}
ADDITION *         & DEFAULT-CONSTIT  & FCC-PERMUTATIONS~ & TERNARY-EXTRAPOL\\ 
AQUEUS-MODEL       & DIFFUSION        & QUASICHEM-MODEL~  & UNIQUAC-MODEL \\  
BCC-PERMUTATIONS~  & DISORDERED-FRACS & QUIT \\ 
COMPOSITION-SET    & FCC-CVM-TETRAHDR~ & REMOVE-COMPSETS\\
\end{tabular}
}

%--------------------------------
\hypertarget{Amend addition}{}
\subsubsection{{\em amend} phase ``phase-name'' Addition }

Additions are used to give a contribution to the Gibbs energy of a
phase using more or less physically based model.  Usually they require
additional model parameters, see section~ref{sc:paramid}.  The
difference between addition and other things that can be amended may
not always be very clear.  The possible additions are

{\small
\begin{tabular}{llll}
 ELASTIC-MODEL-1~ & MAGNETIC-CONTRIB~ & SMOOTH-CP-STEP \\
 GADDITION        & QUIT              & TWOSTATE-LIQUID\\
 LOWT-CP-MODEL    & SCHOTTKY-ANOMALY~ & VOLUME-MODEL1\\
\end{tabular}
}

BEWHERE! The OC software allows you to mix many types of additions for
a phase but it is up to YOU as user to defend the physical reasons for
this!

\hypertarget{Add per formula unit}{{\bf Per formula unit?}}

The theoretical equation for most additions usually gives the value
per mole of atoms.  As the Gibbs energy is calculated per mole formula
unit of the phase in OC (as well as most thermodynamic software) the
addition must be multiplied with the number of atoms per formula unit
of the phase.

Some of the additions, for example mobilities, are for properties that
does not contribute to the thermodynamics but which depend on the
phase, $T, P$ and phase constitution in the same way as the Gibbs
energy and it is thus convenient to model and store the data together
with the thermodynamic data.

%. . . . . . . . . . . .

\hypertarget{Amend elastic-model-1}{}
\subsubsubsection{{\em amend phase ... addition} Elastic\_model\_1}

A contribution to the Gibbs energy due to elastic strain can be added.
This also requires values of the elastic constants and lattice
parameters, see section~\ref{sc:paramid}.  

There is no code to calculate the elastic energy implemented yet.

%.........................
\hypertarget{Amend Gaddition}{}
\subsubsection{{\em amend phase} ... Gaddition}

You can add a constant value of the Gibbs energy to a phase in Joule
per formula unit.  This is a crude but simple way to implement a for
example a nucleation barrier.

{\bf Addition to G in J/FU (formula units)/0/:}

%. . . . . . . . . . . .
\hypertarget{Amend lowt-Cp-model}{}
\subsubsubsection{{\em amend phase ... addition} LowT\_Cp\_model}

The Einstein model for heat capacities from 0~K has been implemented.
It requires a value of the property Einstein T as listed in
section~\ref{sc:paramid}.

%. . . . . . . . . . . .
\hypertarget{Amend magnetism}{}
\subsubsubsection{{\em amend phase ... addition} Magnetic\_contrib}

The Inden-Hillert and the modified Inden-Qing-Xiong model for the
magnetic contribution to the Gibbs energy can be set by this command
This depends on model parameters describing the Curie and Neel
temperatures and the Bohr magneton number, as listed in model
parameters identifiers~\ref{sc:paramid}, for the phase.

You also must also enter model parameters for the constituents of the
phase, see the documentation of the model or Lukas~\cite{07Luk}.

{\bf Antiferromagnetic factor:}

The Qing-Xiong model is selected by giving zero (0) for the question
about the anti-ferromagnetic factor.  For the original Inden-Hillert
model -3 is used for FCC and HCP whereas -1 is used for BCC.

The Inden-Hillert model is described in Lukas et al~\cite{07Luk}.  The
Inden-Qing-Xiong modified model requires separate values of the Curie
and Neel Temperatures and either an ``effective'' Bohr magneton number
or individual Bohr magneton numbers for the constituents of the phase.

%. . . . . . . . . . . .
\hypertarget{Amend addition quit}{}
\subsubsubsection{{\em amend phase ... addition} Quit}

You did not really wanted to add any addition.

%. . . . . . . . . . . .
\hypertarget{Amend Schottky-anomaly}{}
\subsubsubsection{{\em amend phase ... addition} Schottky\_anomaly}

Some physical phenomena can create a ``bump'' in the heat capacity for
a phase at a certain $T$ and this addition can describe this.  It uses
two model parameter identifiers, TSCH and CSCH that may depend on the
composition.  TSCH specify the T for the anomaly and CSCH the maximum
contribution to the heat capacity (J/mol/formula unit) divided by $R$,
i.e. as a factor of the gas constant, $R$.

%. . . . . . . . . . . .
\hypertarget{Amend smooth-Cp-step}{}
\subsubsubsection{{\em amend phase ... addition} Smooth-Cp-step}

The 3rd generation thermodynamic databases extrapolate to 0~K and
require that the heat capacity is zero at 0~K.  This means it is
impossible to use $T*\ln(T)$ terms (and also negative powers of
$T^{-n}$) but there may be some physical phenomena that causes an
incremental increase of the heat capacity at some temperature.
Ignoring the physical reason for such an increase this
``smooth\_$C_P$\_step'' addition will provide such this using two
parameters, THT2 to specify $T$ and DCP2 to specify the increement in
heat capacity.  DCP2 is a factor of $R$.  It uses the same
mathematical expression as the Einstein heat capacity function but has
no enthalpy contribution.

%. . . . . . . . . . . .
\hypertarget{Amend twostate liquid}{}
\subsubsubsection{{\em amend phase ... addition} Twostate-liquid}

The two-state model for the hear capacity for the undercooled liquids
can be added.  It assumes a low T amorphous state modeled as an
Einstein solid and requires an Einstein T.  For the liquid transition
it uses the model\_parameter\_ident {bf G2}, both of which are listed
in section~\ref{sc:paramid}.

{\bf Is G2 composition dependent? /Y/:}

G2 parameters are usually evaluated for thr pure elemenents.  Using
interaction parameters for the G2 parameter may create unexpected
phenomena.

You must specify parameters for THET and G2 for all constituents of
the phase and possibly also interaction parameters to specify the
composition dependence.

The implementation of this addition is not finished.

%--------------------------------- end of amend phase ... addition

\hypertarget{Addition aqueus-model}{}
\subsubsection{{\em amend phase} ... Aqueous-model}

A model with dilute configurational entropy.  Not implemented yet.

%.........................
\hypertarget{Amend BCC-permutations}{}
\subsubsection{{\em amend phase} ... BCC-permutations}

This is intended for the 4 sublattice CEF model for BCC ordering.  Due
to crystallographic symmetry several model parameters must be
identical such as

G(BCC,AL:FE:FE:FE)=G(BCC,FE:AL:FE:FE)=G(BCC,FE:FE:AL:FE)=G(BCC,FE:FE:FE:AL)

and this command means these parameters need to be entered only once.
This affects the data storage and the calculation of the Gibbs energy
is slightly more efficient.  The same applies for the
FCC\_permutations but the BCC tetrahedron is asymmetric which makes it
a bit more complicated than the FCC.  There can be a 5th sublattice
with interstitials.

%.........................
\hypertarget{Add new cs}{}
\subsubsection{{\em amend phase} ... Composition set}\label{sc:amend_phase_cs}

Each phase has by default a single composition set.  If the same phase
can exist as stable (or metastable) with two or more compositions
(miscibility gaps or order/disorder transformations) you may have to
amend the phase by creating additional composition sets.  

Composition sets can also be created automatically by the grid
minimizer during an equilibrium calculation.  In such a case the
composition set will have the suffix \_AUTO,

Composition sets of a phase can be created and deleted.  Phases with
miscibility gaps or which can exist with different chemical ordering
like A2 and B2 must be treated as different composition sets.  You can
specify a prefix and suffix for the composition set.  Extra
composition sets will always have a suffix \#digit where digit is a
number between 2 and 9.  You cannot have more than 9 composition sets.

The composition set number is given after the phase name and preceeded
by a hash character \#.  In the OCASI interface and some more cases
phase tuples are used to identify a phase and a composition set by a
single number.  As composition sets can be created and deleted a phase
tuple index for the 2nd or higher composition set may change between
calculations.

In some cases it may be interesting to calculate metastable states
inside miscibility gaps and you can prevent the automatic creation of
composition sets by turning off the global minimazation using {\bf
  AMEND GENERAL} or for an individual phase by {\bf SET PHASE ... BIT
  NO\_AUTO\_COMP\_SET}

%.........................
\hypertarget{Amend phase default constit}{}
\subsubsection{{\em amend phase} ... Default Constitution}

The default constitution of a phase can be set.  Unless the grid
minimizer is used this will be used for the first calculation with the
phase and sometimes if there are convergence problems.  NOTE that if
you want to specify a default constitution for the second or higher
composition set of a phase you must specify the composition set with
the phase name!

Depending on the minimizing software used the initial constitution can
be important to find the correct equilibrium if the phase has ordering
or a miscibility gap.

For each constituent you can specify a minimum $>$ or maximum $<$
fraction or give NONE if there are no default.

If a phase has miscibility gaps and you have created composition
sets with default constitutions the grid minimizer will try to select
the composition set with a composition closest to the default for a
stable phase.

To temporarily set a new constitution of a phase use the command {\bf
  AMEND CONSTITUTION} $<$phase$>$ or {\bf CALCULATE PHASE ... }.

%.........................
% UNFINISHED
\hypertarget{Add diffusion}{}
\subsubsection{{\em amend phase} ... Diffusion}

This is to specify how the diffusion coefficient matrix should be
calculated when simulating a phase transformation.  Normally the
mobilities for the constituents of the phase are read from the
database but you may use different ``depended'' and ``independent''
constituents in the diffusion model and also some other factors.  This
command is intended for such use.  It is not implemeted yet.

There is no intention that OC itself should simulate diffusion but as
the diffusion coefficents are strongly dependent on the thermodynamic
factor (the Darken stability matrix) which represent the second
derivatives of the Gibbs energy it is convenient to include some
properties used in a simulation in the thermodynamic software.

%.........................
\hypertarget{Amend phase disordfrac}{}
\subsubsection{{\em amend phase} ... Disordered fraction sets}

For phases with several sublattices the Gibbs energy of the phase can
be divided into two sets of fractions where the second or
``disordered'' set have only one or two sublattices and the fractions
on these represent the sum of fraction on some or all of the first or
``ordered'' set of sublattices.

There are two different ways to handle the disordered fraction set
depending on the fact if the phase can be totally disordered.  The
latter is the case for phases like B2, L1$_2$ etc which can be totally
disordered as BCC/A2 or FCC/A1.  The calculation of the Gibbs energy
in the latter cas will subracted the contribution from the ordered
part when the phase is disordered, see for example Lukas et
al~\cite{07Luk}.

This is particularly important to model the Gibbs energy for phases
with ordering like FCC, BCC and HCP and for intermediate phases like
SIGMA, MU etc.

%.........................
% UNFINISHED no question?
\hypertarget{Amend FCC-CVM-tetradrn}{}
\subsubsection{{\em amend phase} ... FCC\_CVM\_tetradrn}

This model is intended for the CVM tetrahedron model for FCC and HCP.
Not implemented yet.

%.........................
\hypertarget{Amend FCC-permutations}{}
\subsubsection{{\em amend phase} ... FCC\_permutations}

This is intended for the 4 sublattice CEF model for FCC ordering.  Due
to crystallographic symmetry several model parameters must be
identical such as

G(FCC,AL:FE:FE:FE)=G(FCC,FE:AL:FE:FE)=G(FCC,FE:FE:AL:FE)=G(FCC,FE:FE:FE:AL)

Setting this means that unique model parameters need to be entered
only once, the software will take care of all permutations.  HCP
permutations are also handled with this command as the HCP tetrahedron
model is identical to the FCC.  There can be a 5th interstitial
sublattice.

%.........................
\hypertarget{Amend quasichemical}{}
\subsubsection{{\em amend phase} ... Quasichemical}

There are several quasichemical models for the liquid that only
describes the short range ordering (SRO).

None of them are yet implemented.

%.........................
\hypertarget{Amend phase ... quit}{}
\subsubsection{{\em amend phase} ... Quit}

Do not amend anything for the phase.

%.........................
\hypertarget{Amend phase ternary extrapol}{}
\subsubsection{{\em amend phase} ... ternary-extrapolation}\label{sc:kohler-toop2}

The default ternary extrapolation is the symmetric Muggianu method
which uses the binary excess Gibbs energy closeset to the overall
composition, see section~\ref{sc:excessparameters}.  However, there is
also a symmetric Kohler method and an ansymmetric Toop method which
can be defined separately for each ternary.  For this you must specity

{\bf Ternary extrapolation (K, T or Q to quit)}

If you specify T for Toop you must specify the Toop constituent,
othewise just any of the three constituents as the {\bf first
  constituent}.  After that you will be asked for

{\bf Second constituent:} and\\
{\bf Third constituent}

For each ternary subsytem in the phase this can be specified.  Those
not specified will use a Muggianu method, see
section~\ref{sc:excessparameters}.

%.........................
\hypertarget{Amend phase UNIQUAC}{}
\subsubsection{{\em amend phase} ... UNIQUAC}

The UNIQUAC model for polymers has been implemented and there is a
macro ``uniquac'' showing how it can be used.

%-------------------------------- end of amend phase
\hypertarget{Amend quit}{}
\subsection{{\em amend} Quit}

Do not amend anything (more).

%--------------------------------
\hypertarget{Amend redundant-sets}{}
\subsection{{\em amend} redundant-sets}

Sometimes a large number of composition sets are created for certain
phases and they may create trouble at later calculations.  This
command will set all metastable composition sets as dormant which may
simplify convergence.  A dormant compositon set may be set stable by
the gridminimizer.  It is also possible to delete composition sets but
that is fragile and they may anyway be created again by the grid
minimizer.

%--------------------------------
\hypertarget{Amend species}{}
\subsection{{\em amend} for Species}

This is implemented for UNIQUAC species which has a specific volume
and area used in the configurational entropy.

{\bf UNIQUAC surface area (q) /1/:}

{\bf UNIQUAC segments (r) /1/:}

These two parameters are necessary to calculate the configurational
entropy of the UNIQUAC model.

%--------------------------------
\hypertarget{Amend symbol}{}
\subsection{{\em amend} Symbol}\label{sc:amendsym}

For a symbol that is a constant this command means changing the value
of symbol.

For some other symbols it is very special.  It is intended for use in
assessments to specify that a particular symbol must not be evaluated
except when specified explicity, or when calculating a specific
equilibrium.

The main problem is that a symbol can have an expression using another
symbols and thus all symbols are normally evaluated whenever the value
of a specific symbol is requested.  This is to ensure that all symbol
values are consistent and refer to the same calculated equilibrium.
But in certain cases you may want to enter a symbol that is only
evaluted when referenced explicity or at a specific equilibrium and
this can be set with this command.

Symbols representing ``dot derivatives'', for example ``H.T'' for the
heat capacity are automatically set to be evaluated only when
referenced explicitly.  For all other symbols except constants OC will
ask:

{\bf You can specify:\\
  V for a symbol evaluated only when referenced explicitly\\
  X for a symbol to be evaluated at a particular equilibrium\\
  Please specify V or X /X/:}

When you want to compare the value of a thermodynamic property, like
the enthalpy, in two equilibria you must be able to store the
calculated enthalpy from one equilibrium in a symbol.  For example if
you have experimental data on the heat difference for a compound at
various $T$.  In such a case the enthalpy at the reference $T$ can be
stored in a symbol, which has been amended with this command to
specify at which equilibrium it should be evaluated.  In all other
equilibria the value of this symbol will have the value at the
specified equilibrium.  See also the documentation on the assessment
procedure, section~\ref{sc:assess}.

If you specify X you will be asked

{\bf Specify equilibrium number:}


%--------------------------------
\hypertarget{Amend TPfun}{}
\subsection{{\em amend} Tpfunction}

You can replace a TP function with a new expression.  If it is a
constant you can give a new value.

%ALERT: Check that this forces new calculation of all TP functions.

%===================================================================
\hypertarget{Back}{}
\section{Back }

Return back from the command monitor to the application program.  In
the OC software itself it means terminate the program.

%===================================================================
% Two calculate needed for help at command level ??
\hypertarget{Calculate}{}
\section{Calculate }

Many different things can be calculated.  The normal thing to calculate is
{\bf equilibrium}, the other things are special.

{\small
\begin{tabular}{llll}
 ALL-EQUILIBRIA~ & GLOBAL-GRIDMIN~ & PHASE *         & TRANSITION      \\
 BOSSES\_METHOD  & NO-GLOBAL~      &  QUIT           & TZERO-POINT \\
 CAREFULLY       & ONLY\_GRIDMIN    & SYMBOL         & WITH-CHECK-AFTER\\    
 EQUILIBRIUM     & PARAEQUILIBRIUM~ & TPFUN-SYMBOLS~  \\
\end{tabular}
}

%--------------------------------
\hypertarget{Calculate all}{}
\subsection{{\em calculate} All equilibria}\label{sc:calcall}

Intended for the assessment procedure.  Calculates all equilibria with
non-zero weight as set by the command {\bf SET RANGE}.  It can also be
used for other purposes, for example testing the parallelization.  The
equilibria can be entered by the command {\bf ENTER MANY\_EQUILIB}.

This command can be looped to measure calculation times.

%--------------------------------
\hypertarget{Calculate Bosses-method}{}
\hypertarget{Calculate carefully}{}
\subsection{{\em calculate} Bosses-method or Carefully}\label{sc:calcbosse}\label{sc:calccare}

  These two ways provide a fairly similar way to handle cases when
  there are convergence problems, in particulat for multicomponent
  systems.  They require that the conditions are $T, P$ and mass
  balance so the grid minimizer can be used.  The difference is that
  after the gridminimizer has founc a set of stable phases all other
  phases are set as suspended and the iterative calculation will just
  use those phases selected by the gridminimizer, this should normally
  be successful.  Afterwards all suspended phases are set as dormant
  and a new iterative calculation is made.  If no dormant phase has a
  positive driving force all phases are set as entered and the
  equilibrium has been calculated.

  If one or more dormant phases have a positive driving force these
  are set as entered one by one followed by an iterative calculation.
  Normally this will finish when all dormant phases have negative
  driving force and the equilibrium has been calculated.  If it fails
  it may anyway be possible to identify the phases causing the
  convergence problems and maybe check its parameters.
  
%--------------------------------
\hypertarget{Calculate equilibrium}{}
\subsection{{\em calculate} Equilibrium}\label{sc:calceq}

The normal command to calculate the equilibrium of a system for the
current set of conditions and phase status.  You can calculate a
metastable equilibrium if some phases that should be stable have been
set dormant or suspended or if automatic creation of composition sets
is not allowed.  If the conditions allow, the grid minimizer will be
used to find start values unless the grid minimizer is explicitly
turned of.

Before this command you must have entered thermodynamic data from a
database or interactivly and used the command {\bf set condition},
section~\ref{sc:setcond}, to set as many conditions as you have
components plus two.  The commands {\bf set status phase},
section~\ref{sc:set-status-phase}, and {\bf set input-amount},
section~\ref{sc:setinpuam} can also be used to set conditions.

For the first equilibrium calculation it is recommended to set
conditions on $T, P$ and the overall composition.  Those conditions
allow the grid minimizer to be used to find the best set of stable
phases and their constitutions that should give the global minimum.
However, the density of the grid may in some cases have to be
increased to ensure that.

For later equilibria you can use a very flexible set of conditions,
see section~\ref{sc:setcond} and the gridminimizer may not be able to
use the grid minimizer.  In such a case OC will use the current set of
stable phases and their constitution as start values.  If you want to
check that such a calculation is the global you can use the command
{\em calculate with} which will call the grid minimizer called AFTER
the equilibrium calculation (if it has converged) to check that it is
indeed a global equilibrium.

%--------------------------------
\subsection{{\em calculate} Global-Gridmin}
\hypertarget{Calculate global-gridmin}{}

Calculate with the global grid minimizer without using this result as a
start point for the general minimizer.  Used to debug the grid
minimizer.

%--------------------------------
\hypertarget{Calculate no-global}{}
\subsection{{\em calculate} No-Global}

Calculate the equilibrium without using a global grid minimizer to
generate start constitutions.  The current equilibrium is used as
start point.  Can be quicker when only small changes of conditions
made since previous calculation and this is how equilibria is
calculated during STEP and MAP.  It means no check of new miscibility
gaps.

%--------------------------------
\hypertarget{Calculate paraeq2}{}
\subsection{{\em calculate} Paraequilibrium}\label{sc:paraeq2}

The paraequilibrium is described in section~\ref{sc:paraeq1}.

%hypertarget{Calculate para2}{{\bf Matrix phase:}}
{\bf Matrix phase:}
              
Note all phases except the matrix and growing phase should be
suspended.  You should provide name of the matrix phase

%hypertarget{Calculate para2}{{\bf Growing phase:}}
{\bf Growing phase:}

%hypertarget{Calculate para2}{{\bf Fast diffusing element:}}
{\bf Fast diffusing element:}

The element that diffuse so fast that its chemical potential is the
same in both phases.  The other elements will have the same
composition in both phases.

%--------------------------------
\hypertarget{Calculate what for}{}
\hypertarget{Calculate phase}{}
\subsection{{\em calculate} Phase ``phase-name''}

This is to calculate properties for a single phase independent of the
current conditions except the values of $T$ and $P$.

%hypertarget{Calculate phase}{{\bf Phase name:}}
{\bf Phase name:}

%hypertarget{Calculate phase}{{\bf Amount of phase:}}
{\bf Amount of phase:}

%hypertarget{Calculate phase current constitution}{{\bf Current (Y), default (D) or new (N) constitution?}}
{\bf Current (Y), default (D) or new (N) constitution?}

You must provide a phase name, the amount of the phase and if you
should use the current constitution or enter a new.

%hypertarget{Calculate Phase}{}

The Gibbs energy of a phase and possible derivatives and some other
things can be calculated.  Mainly for debugging the implementation of
models and testing the software.

{\small
\begin{tabular}{lll}
ALL-DERIVATIVES  & DIFFUSION-COEFF~ & ONLY-G   \\
CONSTITUTION-ADJ~ & G-AND-DGDY        \\
\end{tabular}
}

\hypertarget{Calculate phase ... all-derivatives}{}
\hypertarget{Calculate phase ... loop}{}
\subsubsection{{\em calculate phase} ... All-Derivatives}

The Gibbs energy, all $T$ and $P$ derivatives and all first and second
derivatives with respect to constituents for the specified phase for
current $T,P$ are calculated and listed.

It is possible to loop this calculation to measure calculation times.

\hypertarget{Calculate phase adjust}{}
\subsubsection{{\em calculate phase} ... Constitution\_Adjust}

You will be asked to enter a new composition of the phase (the current
constitution but the current is the default) and this command will
then calculate the Gibbs energy and all chemical potentials for the
given composition.

For a phase with sublattices the constitution of the phase will be
adjusted to have the minimum Gibbs energy for the given composition.

It is useful when one or more components are parts of several
constituents, for example in a gas and for phases with order/disorder
transitions, in particular when the corresponding subroutine is used
in simulations.

\hypertarget{Calculate phase ... diffusion-coeff}{}
\subsubsection{{\em calculate phase} ... Diffusion\_Coefficients}

You will be asked to enter a new composition (default is current) of
the phase and this command will then calculate the Darken stability
matrix
\begin{eqnarray*}
  \frac{\partial^2 G}{\partial N_{\rm A}\partial N_{\rm B}}
\end{eqnarray*}
for all components (see the documentation of the minimiser) and also
all mobility values (if there are any).

\hypertarget{calculate phase ... G-and-dGdy}{}
\subsubsection{{\em calculate phase} ... G\_and\_dGdy}

The Gibbs energy, all $T$ and $P$ derivatives and all first derivatives
with respect to constituents for the specified phase for current $T,P$
are calculated and listed.

IMPORTANT NOTE: The value of $\frac{\partial G_m}{\partial y_i}$ is
NOT the chemical potential, $\mu_i$ of component $i$.  The
understanding of thermodynamics is often poor and the user is reminded
that the chemical potential of a component $i$ is defined as:
\begin{eqnarray*}
\mu_i &=& \left(\frac{\partial G}{\partial N_i}\right)_{T,P,N_{j\ne i}}
\end{eqnarray*}
where $G$ is the integral Gibbs energy and all $N_i$ are independent
variables.  When we model the molar Gibbs energy, $G_m$ as a function
of the constituent fractions, $y_i$, these fractions are not
independent and for a substitutional model, where $y_i=x_i$ i.e. the
mole fractions, the chemical potential is calculated from $G_m$ using:
\begin{eqnarray*}
  \mu_i &=& G_m + \left(\frac{\partial G_m}{\partial x_i}\right)_{T,P,x_{j\ne i}}
  - \sum_j x_j \left(\frac{\partial G_m}{\partial x_j}\right)_{T,P,N_{k\ne j}}
\end{eqnarray*}
because the mole fractions, $x_i$ are not independent.

\hypertarget{calculate phase ... only-G}{}
\subsubsection{{\em calculate phase} ... Only-G}

The Gibbs energy and all $T$ and $P$ derivatives calculated and listed for
the specified phase for the current values of $T,P$.

If the phase has additions the Gibbs energy and its first derivatives
and its second derivative of T of each addition are also listed

\hypertarget{Calculate phase ... quit}{}
\subsubsection{{\em calculate phase} ... Quit}

Do not calculate anything for the phase.

%--------------------------------
\hypertarget{Calculate quit}{}
\subsection{{\em calculate} Quit}

Do not calculate anything at all.

%--------------------------------
\hypertarget{Calculate symbol}{}
\subsection{{\em calculate} Symbol}

A state variable symbol or function is calculated using the results
from the last equilibrium or grid minimizer calculation.  It is used
in particular for calculation of ``dot derivatives'' like $H.T$ for
the heat capacity.

If a wildcard, ``*'', is given as name all symbols, except dot
derivatives and symbols that must be specified explicity and those
that should be calculated for another specified equilibria.  See
section~\ref{sc:amendsym}.


%--------------------------------
\hypertarget{Calculate TPfun}{}
\subsection{{\em calculate} Tpfun-Symbols}

All or a specific TPFUN symbol are calculated for current values of $T$
and $P$.

%--------------------------------
\hypertarget{Calculate transform}{}
\subsection{{\em calculate} Transition}

After calculating an equilibrium you can calculate directly when a
phase will appear or disappear by releasing one of the conditions you
have specified.  Typically this is used to calculate the melting
temperature of an alloy or a solubility limit.  

You specify the phase name and the condition to be released.  The
program will set this phase as FIXED with zero amount and remove the
condition you specified and calculate the equilibrium.  The
calculation may fail if the phase cannot be set stable with zero
amount.  If successful the removed condition will be set to the value
calculated and the phase set stable with zero amount.

%--------------------------------
\hypertarget{Tzero}{}
\subsection{{\em calculate} Tzero point}\label{sc:tzero2}

The T0 (or T zero) point is where two phases have the same Gibbs
energy.  It is a limit of diffusionless transformation between these
phases.  This can be calculated by varying T (or a composition)
calculating the Gibbs energy for the two phases separatly using the
same overall composition.  NOTE in many cases there are no such point!

It is particularly interesting in steels to predict the martensite
transformation which is normally some 100~K below the T0 point.

%--------------------------------
\hypertarget{Calculate with check}{}
\subsection{{\em calculate} with check after}

When the conditions does not allow for the gridminimizer to be used to
find an initial set of phases this command can be used to call the
gridminimizer after the iterative calculation.  If the gridminimizer
finds a phase that should be stable the equilibrium will be
autmatically recalculated.

This type of calculations is regularly done durig STEP and MAP
commands as such calculations normally have a phase as FIX which
prevents use of the gridminimizer.

%===================================================================
\hypertarget{Debug}{}
\section{Debug }

Several possibilities to trace calculations will be implemented in
order to find errors but very little is working yet.  This command is
mainly for the software development.

{\small
\begin{tabular}{llll}
BROWSER     & GRID             & STOP-ON-ERROR~ & TRACE   \\
ELASTICITY~ & MAP-STARTPOINTS~ & SYMBOL-VALUE \\
FREE-LISTS  & SPECIES          & TPFUN~        \\
\end{tabular}
}

%---------------------------------------------------------------
\hypertarget{Debug elasticity}{}
\subsection{{\em debug} Elasticity}

Intended to test the model for strain and stress.  Not implemented.

%---------------------------------------------------------------
\hypertarget{Debug free_lists}{}
\subsection{{\em debug} Free lists}

Only for experts.

%---------------------------------------------------------------
\hypertarget{Debug map_startpoints}{}
\subsection{{\em debug} Map-startpoints}

An attempt to generate automatic startpoints for mapping a phase diagram.

%---------------------------------------------------------------
\hypertarget{Debug symbol value}{}
\subsection{{\em debug} Symbol value}

This is used to in macro files to test if the software calculates the
same value of a symbol as when the macro was created.  If not there is
some new bug introduced (or a bug corrected?).  After the symbol the
expected value must be given and if the relative difference with the
calculated value differ more than $10^{-6}$ the program will abort.

%---------------------------------------------------------------
\hypertarget{Debug stop-on-error}{}
\subsection{{\em debug} Stop\_on\_Error}

The program will stop at the command level after printing the error
message if an error has occurred when using macro file.  This should
make it easier to to find errors occurring when running macro files.

However, it is not implemented.

%===================================================================
\hypertarget{Delete}{}
\section{Delete }

It is quite difficult to delete anything when the data structure is so
involved.  In many cases it may be better to enter the data again
without the data that should be deleted.  But there are a few things
that must occationally be deleted.

{\small
\begin{tabular}{llll}
 COMPOSITION\_SET~ & EQUILIBRIUM~ & QUIT    & STEP\_MAP\_RESULTS\\
 ELEMENTS          & PHASE        & SPECIES~\\
\end{tabular}
}

%------------------------------------------------------
\hypertarget{Delete composition set}{}
\subsection{{\em delete} Composition set}

The first composition set of a phase cannot be deleted.  Otherwise
there is usually no problem unless several equilibria are entered
because the composition set must be deleted in all equilibria.
Composition sets are created and deleted during normal equilibrium
calculations to detect miscibility gaps.

%------------------------------------------------------
\hypertarget{Delete element}{}
\subsection{{\em delete} Element}

Dangerous and will probably never be implemented.

%------------------------------------------------------
\hypertarget{Delete equilibrium}{}
\subsection{{\em delete} Equilibrium}

Dangerous but sometimes necessary.  Done automatically at a second
STEP or MAP command if you specifies to delete previous results.

%------------------------------------------------------
\hypertarget{Delete phase}{}
\subsection{{\em delete} Phase}

Dangerous and will probably never be implemented.

%------------------------------------------------------
\hypertarget{Delete quit}{}
\subsection{{\em delete} Quit}

Do not delete anything.

%------------------------------------------------------
\hypertarget{Delete species}{}
\subsection{{\em delete} Species}

Not implemented yet and will probably never be.

%------------------------------------------------------
\hypertarget{Delete step-map-results}{}
\subsection{{\em delete} Step\_Map\_Results}

This removes all equilibria and saved equilibria associated with
STEP and MAP commands.  It also deletes the axis.

%===================================================================
\hypertarget{Enter}{}
\section{Enter }

In most cases data will be read from a database file.  But it is
possible to enter all thermodynamic data interactively.  This should
normally start by entering all elements, then all species (the
elements will automatically also be species) and then the phases.

A species have a fixed stoichiometry and possibly a charge.  The
species are the constituents of the phases.

A phase can have sublattices and constituents and also various
additions like magnetic, low T heat capacity etc. which are specified
by the {\bf AMEND} command efter entering the phase (but normally
before any model parameters for the phase are entered).

TPFUN symbols can be used to describe common parts of model
parameters.  See section~\ref{sc:tpfun} for an explation.

Each model parameter of a phase is entered separately.  You may use
TPFUN symbols which are already entered.

At present the multicomponent CEF model and the ionic 2-sublattice
liquid model are the only basic models implemented.  The CEF model
includes as special cases the gas phase, regular solutions with
Redlich-Kister Muggianu model and phases with up to 9 sublattices and
ionic constituents.  These models describe the basic configurational
entropy contribution to the phase, models such as the magnetic
contribution and low T heat capacity can be added to a phase with the
{\bf AMEND} command.

The enter command is also used to enter bibliographic data, equilibria
for assessments and many other things.

The subcommands are:

{\small
\begin{tabular}{llll}
 BIBLIOGRAPHY     & EQUILIBRIUM      & OPTIMIZE-COEFF~   & SPECIES\\
 COMMENT          & EXPERIMENT       & PARAMETER        & SYMBOL\\ 
 CONSTITUTION     & GNUPLOT-TERMINAL~ & PHASE            & TPFUN-SYMBOL\\ 
 COPY-OF-EQUILIB~  & MANY-EQUILIBRIA  & PLOT-DATA\\
 ELEMENT          & MATERIAL         & QUIT\\
\end{tabular}
}

%--------------------------------
\hypertarget{Enter bibliography}{}
\subsection{{\em enter} Bibliography}\label{sc:bibref}

Each model parameter must have a bibliographic reference to ensure
everyone can find the origin of its value.  When entering a parameter
a bibliographic reference symbol must be given and with this command
you can give a full reference text for that, for example a published
paper, a report or simply a reason for the value together with the
date and your name so the origin of the parameter can be traced.

{\bf Reference identifier:}

The text for bibliographic reference identifier can be amended.  The
reference identifier is case insensitive.

{\bf Reference text, end with ``;'':}

The text for this reference will be set to the text supplied.  It can be
several lines terminated with a ``;''

%--------------------------------
\hypertarget{Enter comment}{}
\subsection{{\em enter} Comment}

A line of comment text can be added to the current equilibrium.  It is
particularly important when entering experimental data to give the
reference to the data.

%--------------------------------
\hypertarget{Enter constitution}{}
\subsection{{\em enter} Constitution}

The constitution (fraction of all constituents) of a phase can be
entered.  This is a way to provide start values for an equilibrium
calculation (when not using grid minimizer).  To calculate the Gibbs
energy for a specific phase at a specific constitution use the command
{\bf CALCULATE PHASE}.

%--------------------------------
\hypertarget{Enter copyof}{}
\subsection{{\em enter} Copy of equilibrium}

This command creates a copy of the current equilibrium with the same
set of conditions and related data.

Must be used with care.

%--------------------------------
\hypertarget{Enter element}{}
\subsection{{\em enter} Element}

The data for an element is entered.  It consists of is symbol, name,
reference phase, mass, H298-H0 and S298.

The element symbol must be one or two letters, they will be converted
to UPPER case automatically.  The element name and reference phase is
never used anywhere but included for completeness.  The reference
phase SER means the Stable Element Reference phase, the phase stable
at 298.15~K and 1~bar.  The mass is needed for input of amount (using
state variable B), mass fractions or mass percent of the element.

The values of H298-H0 and S298 are never used for any calculation but
included for completeness.

%--------------------------------
\hypertarget{Enter equilibrium}{}
\subsection{{\em enter} Equilibrium}

You can have several equilibria each with a unique set of conditions
including phase status (dormant, suspended, fix or entered) but all
with the same components and thermodynamic data.  This is useful for
compare different states, to simulate transformations and to assess
model parameters as each experimental or theoretical information
represented as an equilibrium.

All equilibria use the same thermodynamic data but they have an
independent set of conditions and result data structure, also for TP
functions and symbols, and they can be calculated in parallel.

After entering the equilibrium you can select if your following
commands, such as {\em enter condition} etc. will apply to the new
equilibrium.

%--------------------------------
\hypertarget{Enter experiment}{}
\subsection{{\em enter} Experiment}\label{sc:enterexp}

This is used for assessments, experimental data can be specified for
an equilibrium.  The experiment is a state variable or symbol which
can be set equal to the experimental value followed by a colon, ``:''
and its uncertainty.

In some cases an experimental value can be an upper or lower limit.
In such cases the ``$>$'' or ``$<$'' can be used.  The value of the
uncertainty will then be interpreted as a penalty factor if the
calculated value is outside the specified limit.

%--------------------------------
\hypertarget{Enter GNUTERM}{}
\subsection{{\em enter} GNUPLOT Terminal}\label{sc:gnuterm}

For plotting OC generates a command file for the
GNUPLOT~\cite{gnuplot} software.  GNUPLOT can be downloaded free for
most OS but depending on your screen and other hardware you may prefer
to specify your prefered set of terminals.  On Windows the defaults
are:

% IMPORTANT if table changed here also change in
% \subsection{{\em plot xaxis yaxis} Graphics format}

{\bf The terminals listed in the table depend on your installation.}

\begin{tabular}{rlcl}
  & Name   &=~& GNUPLOT definition\\
 1~& SCREEN & & set terminal wxt size 940,700 font "arial,16"\\
 2~& PS     & & set terminal postscript color solid fontscale 1.2\\
 3~& PDF    & & set terminal pdf color solid size 6,5 enhanced font "arial,16"\\
 4~& GIF    & & set terminal gif enhanced fontscale 0.7\\
 5~& PNG    & & set terminal png enhanced fontscale 0.7\\
\end{tabular}

The text after the $>$ is written on the GNU command file.  You can
change these or add additional terminals.  You can also change these
in the source code (userif/pmon6.F90 file) or use a macro file
OCHOME/start.OCM file to set them.

%--------------------------------
\hypertarget{Enter many equil}{}
\subsection{{\em enter} Many Equilibria}\label{sc:entermany}

This command is intended for adding tables of experimental data of the
same type.  It can also be used for calculation of many equilibria
using the {\bf calculate all} command.  The user first enters a TABLE
HEAD giving the necessary phase status, conditions, experiments etc.
In this ``head'' some values of text can be referred to columns in the
following table using the ``@'' character followed by a digit 1 to 9,
where the digit is the column number.

The prompt for input to the table head is ``table head::''\\ In the
examples below, taken from the parallel2.OCM macro file, user input is
{\bf in bold} and explanations {\em in italics}.

\begin{itemize}
\item By default all phases are suspended so the user must forst specify the
  phases with dormant, entered of fixed status (including amount) like\\
  Table head: {\bf entered 0 *} {\em all phases should be entered}\\ 
  Table head: {\bf fix 0 liquid} {\em liquid should be fix with 0 moles}\\ 
  Table head: {\bf fix 1 @2} {\em the phase in column 2 should be fix
    with 1 moles}

\item The conditions can be given using the @ character to indicate vaules
  that are given in the specified column in table to follow.\\
  Table head: {\bf condition t=@1 p=1e5 n=1 w(cr)=@3 w(mo)=@4 }

\item Optional calculations of entered symbols\\
  Table head: {\bf calculate cp}

\item Optional listing of state variables\\
  Table head: {\bf list HM tc(bcc)}

\item Optional experimental data\\
  Table head: {\bf experiment x(liquid,cr)=@5:.01, x(bcc,cr)=@6:.02}
      
\item Optional reference state\\
  The reference state for a component can be set.

  Table head: {\bf reference O gas * 1e5}\\ 
  The reference state for the component O will be gas at the current
  $T$ and 1 bar.

\item Optional plot\_data specifying a dataset number and coordinates
  to be plotted and a symbol.  The coordinates can be table columns.
  Use the dataset numbers to have data of the same type together like
  enthalpies, phase diagram data etc.\\
  Table head: {\bf plot 1 @1 @2 5}

\item Optional comment\\
  Table head: {\bf comment experimental data from Kubaschewski 1955}

\item The table head is finished by an empty line or ``table\_start''
\end{itemize}
  
\hypertarget{Enter table row}{}

For the rows in the table the user must first provide a unique name
for each equilibrium (that is counted as column 0 (zero)) and values
for all columns referenced in the table head like:\\
Table row: {\bf EQ1 1573 BCC 0.3 0.05 0.12 0.28}\\
Table row: {\bf EQ2 1623 BCC 0.3 0.10 0.18 0.24}\\

The table is finished by an empty line or\\
Table row: {\bf table\_end}

%--------------------------------
\hypertarget{Enter material}{}
\subsection{{\em enter} Material}

The user will be asked for a name of the material and possibly a
database.  Then he can give elements and their amount in mass percent
or mole fraction.  Finish with an empty line.

Finally he can specify the temperature and the program will
automatically make a calculation at 1 bar with the given composition.
For example:

\begin{verbatim}
OC4:enter mat
Database:steel7
Elements: C , MO, V , CR, FE, SI,
Major element or material:fe
Input in mass percent? /Y/:
Input expected in mass percent

First alloying element:c
Mass percent: /1/:
Second alloying element:cr
Mass percent: /1/: 5
Third alloying element:mo
Mass percent: /1/: 8
Next alloying element:v
Mass percent: /1/:
Next alloying element:
 3E reading a TDB file
 3D em:  W%(C)=1  W%(CR)=5  W%(MO)=8  W%(V)=1   N=1
Temperature /1000/:
 3Y Constitution of metastable phases set
 3Y Composition set(s) created:            1
Gridmin:   18846 points   6.25E-02 s and      78 clockcycles, T= 1000.00
Phase change: its/add/remove:     5    0   21
Equilibrium calculation   19 its,   6.2500E-02 s and      94 clockcycles

\end{verbatim}

The user can use the same command to specify another composition of
the alloy or use other commands such as {\bf SET CONDITION} and {\bf
  CALCULATE} or calculate diagrams using {\bf SET AXIS} and then {\bf
  STEP} or {\bf MAP}.

%--------------------------------
\hypertarget{Enter coeffs}{}
\subsection{{\em enter} Optimizing coefficient}\label{sc:optcoeff}

The number of TP symbols for the coefficients to be optimized are
entered.  They have the names A00 to A99.  They are used in model
parameters and can be varied by the optimization procedure to minimize
the difference between the experimental data and the same property
calculated from the models of the phases.

You can also specify the size of the workspace needed for the
optimization.  The default value, 2500, is usually sufficient.

%--------------------------------
\hypertarget{Enter parameter}{}
\subsection{{\em enter} Parameter}\label{sc:enterparam}

A model parameter is defined by its identifier, the phase and
constituent array and the degree.  A parameter can be a constant or
depend on T and P.  The parameter will be multiplied with the
fractions of the constituents given by its constituent array.  See the
documentation of the GTP model package or the book by Lukas et
al\cite{07Luk} for more information about thermodynamic models.

For example G(LIQUID,CR) is the Gibbs energy of liquid Cr relative to
its reference state, normally the stable state of Cr at 298.15 K and 1
bar, and called an endmember.

For a gas molecule the parameter G(GAS,C1O2) is also an endmember and
represent the Gibbs energy of the C1O2 molecule relative to the
reference states of C (carbon) and O (oxygen).

For interaction parameters the components are separated by a comma
``,'' as in G(LIQUID,CR,FE).

For phases with sublattices the constituents in each sublattice are
separated by a colon, ``:'' and interacting constituents in the same
sublattice by a comma, ``,''.  For example:\\
G(FCC,FE:C,VA) is the interaction between C (carbon) and VA (vacant
interstitial sites) in the FCC phase.

Different ternary extrapolation methods can be used, see
section~\ref{sc:excessparameters}.

%--------------------------------
\hypertarget{Enter phase}{}
\hypertarget{enter phase name}{}
\subsection{{\em enter} Phase}

The user must specify a unique phase name:

{\bf Phase name:}

All thermodynamic data are connected to a phase as defined by its
parameters, see {\bf enter parameter}.  A phase has a name with can
contain letters, digits and the underscore character.  It must start
with a letter.

%...................................
\hypertarget{Enter phase model}{{\bf Phase model:}}

After the phase name you must specify a model.  The model specfication
is implemented in a rather rudimentary way. The only recognized models
are

\begin{itemize}
\item IDEAL for a single lattice phase without interactions (like GAS)
\item RKM for a substitutional phase with interactions (like metallic
  liquid)
\item I2SL for the ionic liquid phase (2 sublattices with variable
  site ratios).  If the phase name is IONIC\_LIQUID this prompted as
  the default model.
\item CQC means the ``Corrected Quasichemical model'' for liquids.
\item CEF for any other phase with two or more sublattices
\end{itemize}

This list may be extended in a future version of OC.  Many other model
features like magnetism, quasichemical etc are specified with the {\bf
  AMEND PHASE} command, see section~\ref{sc:amendph}.  The AMEND PHASE
command is also used to specify disordered fraction set, low
temperature CP model and many other things.

%...................................
\hypertarget{Enter phase subl}{{\bf Number of sublattices:}}

For a phase with Long Range Orderng (LRO) you must specify the number
of sublattices.  After that you have for each sublattice specify the
number of sites and consttuents.  Even if you have just one lattice
you must specify the number of atoms on that lattice per formula unit.

For most models OC will ask for the number of sublattices and a phase
can have 1 to 9 sublattices and you must specify the number of sites
on each.  Preferably use small integer values, if fractions are used
at least 6 digits should be provided.

\hypertarget{Enter phase sites}{{\bf Number of sites on a sublattice }}

For some models, like the ionic liquid model, the number of sites may
change with the composition of the phase so the number specified is
irrelevant.  See the book by Lukas et al~\cite{07Luk} for more details
on models.

\hypertarget{Enter phase bonds}{Models with bonds}

Some models depend on the number of nonds between atoms, such as the
quasichemical model.  The modified quasichemical model have a single
sublattice and include additional species to decribe the Short Range
Ordering (SRO).

%...................................

\hypertarget{Enter phase constituents}{}

For each sublattice you must specify the constituents on the
sublattice.  A constituent that is not an element must already have
been entered as a species, see section~\ref{sc:entersp}.

You may have to use the {\bf AMEND PHASE} command, see
section~\ref{sc:amendph}, for some additional model features like
magnetism, low $T$ heat capacity or permutations.

%--------------------------------
\hypertarget{Enter plot data}{}
\subsection{{\em enter} Plot\_data}

This is when entering experimental data for assessments when combining
experimental data in single equilibria with those entered in tables
using the command ``MANY\_EQUILIBRIA''.

You can add points to a dataset 1 to 9 to be plotted the current
equilibrum.  The dataset must already have created by a PLOT command
inside a {\bf ENTER MANY\_EQUILIB} command, see
section~\ref{sc:entermany}.

%--------------------------------
\hypertarget{Enter quit}{}
\subsection{{\em enter} Quit}

Quit entering things.

%--------------------------------
\hypertarget{Enter species}{}
\subsection{{\em enter} Species}\label{sc:entersp}

A species consists of a name and a stoichiometric formula.  It can have
a valence or charge.  The name is often the stoichiometric formula
but it does not have to be that.  Examples:

\begin{itemize}
\item enter species water h2o
\item enter species c2h2cl2\_trans c2h2cl2
\item enter species c2h2cl2\_cis c2h2cl2
\item enter species h+ h1/- -1
\end{itemize}

Single letter element names must be followed by a stoichiometric
factor unless it is the last element when 1 is assumed.  Two-letter
element names have by default the stoichiometric factor~1.

There can be a problem with ambiguity with a species name like h2o if
there is also a species h2o2.  In such cases use a final unity, i.e.
h2o1.

\begin{itemize}
\item enter species carbonmonoxide c1o1
\item enter species cobaltoxide coo
\item enter species carbondioxide c1o2
\item DO NOT USE enter species co c1o1
\end{itemize}

The species name is important as it is the name, not the
stoichiometry, that is used when referring to the species elsewhere
like as a phase constituent.  It is of course convenient to choose a
species name similar to its stoichiometric formula but as shown above,
that is not always sufficient.

{\bf Species symbol:}

The symbol must start with a letter, A-Z, and contain just letters,
digits and the special characters ``\_'' (underscore), ``-'' (minus),
 ``+'' (plus) and ``/'' (slash).

{\bf Species stoichiometry:}

The stoichiometry must contain element symbols followed by a
stoichiometry factor.  The stoichiometry factor 1 can be omitted for
two-letter element symbols.  The charge is given as ``/-'' or ``/+''
followed by a stoichiometric factor.

%--------------------------------
\hypertarget{Enter symbol}{}
\subsection{{\em enter} Symbol}

The OC package has both ``symbols'' and ``tpfun\_symbols'', the latter
has a very special syntax and can be used when entering parameters.

The symbols are designed to handle relations between state variables,
you can define expressions like \\
{\bf enter symbol KLBCR = X(LIQUID,CR)/X(BCC,CR);}\\
where KLBCR is set to the partition of the Cr mole fractions between
liquid and bcc.

The symbols also include ``dot derivatives'' like $H.T$ which is the
second derivative of the Gibbs energy with respect to the for the
current system at the given set of conditions.

{\bf enter symbol CP = H.T;}

If $T$ and $P$ are conditions and all other conditions are mass
balance conditions CP is the heat capacity.  It also takes account of
the change of configurational entropy.

Currently $H.T$ is the only dot derivatives allowed but more will be
added as soon as possible.

%--------------------------------
\hypertarget{Enter TPfun}{}
\subsection{{\em enter} Tpfun\_Symbol}\label{sc:entertpf}

This symbol is a special type of expression depending on $T$ and $P$
that can be used when entering parameters.  A TPfun can refer to
another TPfun but not any other state variable or symbol.

The program requests a name and if the symbol should be a FUNCTION,
CONSTANT or a TABLE (tables not implemented).

If it is a FUNCTION you must specify a low $T$ limit, and expression
consisting of simple terms (signed coefficients multiplied with $T$
and $P$ possibly raised to powers).

A term may also be multiplied with another TP function or with LN(FX)
for the natural logarithm of ``FX'' or EXP(FX) for the exponential of
the expression of function ``FX''.

The ``FX'' inside the parenthesis of an LN or EXP may refer to another
TP function or it can be a coefficient multiplied with powers of T or
P.

It is not allowed to use parenthesis except around arguments of LN and
EXP or around negative powers such as $T**(-1)$.

A very special unary function is INTEIN(THETA) which calculates
\begin{eqnarray}
1.5*R*FX + 3*R*T*LN(EXP(-THETA/T) + 1) 
\end{eqnarray}
and first and second derivatives of that with respect to $T$.  It is
the Einstein heat capacity function integrated to a Gibbs energy.  The
argument THETA should the the Einsten themperature and must be a
positive constant.

The expression must be terminated by a semicolon followed by an upper
$T$ limit.  After the upper $T$ limit you must specify either N or Y.
If you give Y it means there is another expression above this T limit.
The last T-range limit must be followed by N and a bibliographic
reference, see section~\ref{sc:bibref}.

TPFUNs have a strict syntax because the software must be able
calculate not only its value but also its first and second derivatives
with respect to $T$ and $P$ millions of times during a phase diagram
calculations, see section~\ref{sc:tpfun}.

%===================================================================
\hypertarget{Exit}{}
\section{Exit}

Terminate the OC software in Swedish, Ha en bra dag.

%===================================================================
\hypertarget{Fin}{}
\section{Fin}

Terminate the OC software in French, Au revoir.

%===================================================================
\hypertarget{Help}{}
\hypertarget{Help for which *}{}
\section{Help and ?}

{\bf Which command:}

Can give a list if commands or subcommands or parts of this help text.
The user guide is also available as a searchable HMTL file.

For a submenu question a single ? will give the menu and two ?? will
give an extract of this user guide.  Then the question will be asked
again.

%===================================================================
\section{HPcalc }

Start the reverse polish calculator.

%===================================================================
\hypertarget{Info}{}
% this hypertarget is because changes is the default ...
\hypertarget{Topic?changes}{}
% This hypertaget when answering ?? for Topic?/QUIT/:
\hypertarget{Topic?quit}{}
\section{Information }

on the following topics:

\begin{tabular}{llll}
 CHANGES         & ELEMENTS     & PHASE-DIAGRAM~   & STATE-VARIABLES\\ 
 COMPOSITION-SET~& EQUILIBRIUM~ & PROPERTY-DIAGRAM~\\
 CONDITIONS      & HELP-SYSTEM  & QUIT-INFO\\
 DATABASES       & PHASE        & SPECIES\\
\end{tabular}

This command is still not fully implemented.

The intention is to provide the on-line help to users who does not
like to read manuals.  But it is not yet implemented.

{\bf Topic? /CHANGES/:}

Will list the most recent changes in the OC software from the
changes.txt file (if it can be found).  Stop listing by a q.

You can explore different parts of this User Guide online by selecting
other topics.

Give QUIT or press return to go back to top
level.

%===================================================================
\hypertarget{List}{}
\section{List }

Many things can be listed.  Output is normally on the screen unless it
is redirected by the /output={\em file name} or /append={\em file
  name} option, see~\ref{sc:options}.

{\small
\begin{tabular}{llll}
 ACTIVE\_EQUILIBR~ & EQUILIBRIA       & OPTIMIZATION  & STATE\_VARIABLES~\\
 AXIS              & ERROR-MESSAGE    & PARAMETER~    & SYMBOLS\\ 
 BIBLIOGRAPHY~     & EXCELL-CSV-FILE  & PHASE         & TPFUN\_SYMBOLS\\
 CONDITIONS        & LINE-EQUILIBRIA~ & QUIT          \\
 DATA              & MODEL-PARAM-ID   & RESULTS          \\
 ELEMENTS          & MODEL-PARAM-VAL~ & SHORT            \\ 
\end{tabular}
}

%--------------------------------
\hypertarget{List active}{}
\subsection{{\em list} active-equilibria}

This is used during assessment to list equilibria with non-zero weights.

%--------------------------------
\hypertarget{List axis}{}
\subsection{{\em list} Axis}

Lists the axis set by you.

%--------------------------------
\hypertarget{List biblio}{}
\subsection{{\em list} Bibliography}

List the bibliographic references for the data.

%--------------------------------
\hypertarget{List conditions}{}
\subsection{{\em list} Conditions}

Lists the current set of conditions set by you.  If the degrees
of freedoms are zero you can calculate an equilibrium.

%--------------------------------
\hypertarget{Output format for screen}{}
\hypertarget{Output format}{}
\hypertarget{List data}{}
\subsection{{\em list} Data}

Lists all thermodynamic data.  The default is on SCREEN but you can
also choose among the formats: LaTeX, MACRO, PDB and TDB.

The only format implemented at present is SCREEN.

%...............................
\hypertarget{List data LaTeX}{}
\subsubsection{{\em list data} LaTeX}

The thermodynamic data will be formatted according to LaTeX for later
inclusion in publications.  Not implemented.

%....................................
\hypertarget{List data macro}{}
\subsubsection{{\em list data} Macro}

The thermodynamic data will be written as a macro file that can later
be read back into the OC software.  Not implemented.

%....................................
\hypertarget{List data PDB}{}
\subsubsection{{\em list data} PDB}

A ``Phase related Data Format'' similar to the TDB file
format adapted for OC.  Not yet implemented.

%....................................
\hypertarget{List data TDB}{}
\subsubsection{{\em list data} TDB}

A variant of the TDB file format with Thermo-Calc flavor.  Not implemented.

%--------------------------------
\hypertarget{list equilibria}{}
\subsection{{\em list} Equilibria}

Lists the equilibria entered.  To list the results of the calculation
of an equilibrium use {\bf list result}.

%--------------------------------
\hypertarget{List error msg}{}
\subsection{{\em list} Error message}

The message associated with an error code generated by OC can be listed

%--------------------------------
\hypertarget{List line-equilibria}{}
\subsection{{\em list} Line equilibria}

Lists the equilibria calculated during STEP or MAP commands.  See also
the command {\bf AMEND LINE-EQUILIBRIA}.

%--------------------------------

\hypertarget{List model parameter id}{}
\subsection{{\em list} Model parameter identifiers}

Lists the model parameter identifiers available in the current version
of OC, see section~\ref{sc:paramid}.

%--------------------------------
\hypertarget{List model parameter val}{}
\subsection{{\em list} Model parameter value}

The current value of a model parameter identifier can be listed.  Note
that the value is always phase dependent and may also depend on the
composition set.

%--------------------------------
\hypertarget{List Optimization}{}
\subsection{{\em list} optimization}

Lists results of an optimization, several sub-options will be
implemented but currently there is a short version only.  To save this
on a file use the option /output= or /append=, see~\ref{sc:options}.

{\small
  \begin{tabular}{llll}
COEFFICIENTS      & DEBUG        & GRAPHICS~      & MACRO     \\
CORRELATION\_MTRX~ & EXPERIMENTS~  & LONG          & SHORT \\
  \end{tabular}
  }
%.....................................
\hypertarget{List optimization coefficiets}{}
\subsubsection{{\em list optimization} coefficients}

This gives a list of the coefficients and their values.

%.....................................
\hypertarget{list optimization debug}{}
\subsubsection{{\em list optimization} debug}

Not implemented yet.

%.....................................
\hypertarget{List Optimization correlation-mtrx}{}
\subsubsection{{\em list optimization} correlation\_matrix}

Not implemented yet.

%.....................................
\hypertarget{list optimization experiments}{}
\subsubsection{{\em list optimization} experiments}

List of experiments in the equilibria with non-zero weights.

%.....................................
\hypertarget{List optimization graphics}{}
\subsubsection{{\em list optimization} graphics}

A figure with the experimental values on the X axis and calculated values
on the Y axis for all experiments.  Not implemented yet.

%.....................................
\hypertarget{List optimization long}{}
\subsubsection{{\em list optimization} long}

Not implemented yet

%.....................................
\hypertarget{List optimization macro}{}
\subsubsection{{\em list optimization} macro}

A listing of all thermodynamic data and current values of model
parameter and experimental data with current weight.  This can be read
back as a start of a re-assessment and an important documentation of
the current state of the assessment.  But not yet implemented.

%.....................................
\hypertarget{List optimization short}{}
\subsubsection{{\em list optimization} short}\label{sc:listoptshort}

This specifies tha data and hour of the listing and first a table with
the optimizing coefficents with name, current value, start value,
scaling factor and its relative standard deviation.

In the first table all the optimizing coefficents with non-zero values
are listed together with the current values, the start values and
their scaling factor (usually ths same as the start value).  In the
column ``RSD'' the Relative Standard Deviation'' should appear but it
is not yet calculated correctly.  Last column is the name of the TP
symbol(s) where the coefficient is used.

After that all equilibria with non-sero weights are listed together
with their experimental data, both the prescribed value, the
uncertainy and the currently calculated one.  In the last column the
error is listed.

\begin{verbatim}
Listing of optimization results: date 2018.08.20 : 12h47

List of coefficients with non-zero values
Name  Current value  Start value   Scaling factor RSD          Used in
A11     3.46818E+02   4.00095E+02   4.00095E+02   1.25070E-06  _GFCCAB0
A12    -5.66234E+01  -6.52871E+01  -6.52871E+01   1.33802E-06
A13    -2.10028E-02  -1.30393E-02  -1.30393E-02   8.97167E-06  _GFCCAB0

List of     4 equilibria with     8 experimental data values
  No Equil name    Weight Experiment $ calculated                   Error
   2 FCC1_ZA        1.00 SM=17:1 $ 17                               9.8995E-09
   2                1.00 CP1=18:1 $ 17.28685                        7.1315E-01
   3 FCC2_ZB        1.00 HDIFF=9000:500 $ 9997.813                 -1.9956E+00
   3                1.00 CP1=20:DCP $ 22.55698                     -2.5570E-02
   4 FCC3_ZC        1.00 HDIFF=15000:500 $ 14719.24                 5.6152E-01
   4                1.00 CP1=22:DCP $ 24.65726                     -2.6573E-02
   5 FCC4_ZD        1.00 HDIFF=20000:500 $ 19860.72                 2.7856E-01
   5                1.00 CP1=24:DCP $ 26.75754                     -2.7575E-02

Final sum of squared errors:      4.88614E+00 using    8 experiments and
  3 coefficient(s).  Degrees of freedom:    5, normalized error:    9.7723E-01
\end{verbatim}

In the list of equilibria with non-zero weight the first column is a
sequential equilibrium number assigned by the software.  Then the name
of the equilibrium assigned by the user. The third column is the
weight, only equilibria with nonzero weight are listed.  Then comes a
columm with the experimental property and value and after the dollar
sign its calculated value with the present set of coefficients.  The
rightmost column gives the difference for each experiment $i, q_i$
that should be as close to zero as possible:
\begin{equation}
q_i = \frac{z^{\rm exp}_i - z^{\rm calc}}{\sigma_i} w_i
\end{equation}
where $i$, $z_i^{\rm exp}$ is the experimental property, $z_i^{\rm
  calc}$ is the same property calculated from the model and $\sigma_i$
is the experimental uncertanty and $w_i$ is the weight assigned to
equilibria with the experiment.  If $w_i = 1$ and $q_i$ is between -1
and 1 the experiment has been fitted within the experimental
uncertanty.

The least square routine tries to determine coefficients to make the
sum of all $q_i^2$ as small as possible.

At the end of the listing $\sum_i q_i^2$ is listed.  The degrees of
freedom is the number of experiments minus the number of coefficients.

%--------------------------------
\hypertarget{List parameter}{}
\subsection{{\em list} Parameter}

List a specific parameter.

%--------------------------------
\hypertarget{List phase}{}
\hypertarget{List what for}{}
\subsection{{\em list} Phase ``phase-name''}

You must first specify the phase name.  Then you can specify if you
want the phase CONSTITUTION, DATA or some MODEL information.  To write
on a file use the options /output= or /append=, see~\ref{sc:options}.

%...................
\subsubsection{{\em list phase} ... Constitution}

List the constitution of the phase.

%...................
\subsubsection{{\em list phase} ... Data}

List the model and model parameter expressions.

%...................
\subsubsection{{\em list phase} ... Model}

List some model data for example if there is a disordered fraction set.

%--------------------------------
\subsection{{\em list} Quit}

You did not really want to list anyting.

%--------------------------------
\hypertarget{List results}{}
% when user asks ?? on command level the help system will search
% for command and default in UPPER CASE
\hypertarget{LIST RESULTS}{}
\subsection{{\em list} Results}

List the results of an equilibrium calculation.  This is the most
frequent list command.  The listing will contain the current set of
conditions, a table with global data, a table with component specific
data and then a list of stable phases with amounts, compositions and
possibly constitutions.  It is possible to list also unstable phases.

There are 9 options for the formatting:
\begin{itemize}
\item 1 Output in mole fractions, phase constituents in value order
  (constituent with highest fraction first).
\item 2 as 1 but include also the phase constitution (sublattices and
  their fractions) in value order.
\item 3 as 1 with the phase composition in alphabetical order
\item 4 Output in mass fractions, phase composition in value order.
\item 5 as 4 with the phase composition in alphabetical order.
\item 6 as 4 and also include the phase constitutions in value order.
\item 7 Output all phases will with composition in mass fractions and
  in value order.  Unstable phases will have a negative driving force.
\item 8 Output all phases will with composition in mole fraction and
  constitution in alphabetic order.  Unstable phases will have a
  negative driving force.
\item 9 as 8 but in in value order.
\end{itemize}

For each phase the name, its status
(S=suspended/D=dormant/E=entered/F=fix), moles (or mass), volume,
number of formula units, atoms per formula units and driving force (in
dimensionless units) is given on one line.

The moles of a phase is the number of formula unit multiplied with
atoms per formula units.  The gas phase and phases with interstitials
and vacancies have a varying amount of moles of atoms per formula
units.  The composition of the phase can be in value order or
alphabetical order.

To write the output on a file use /output= or /append=,
see~\ref{sc:options}.

%--------------------------------
\hypertarget{List short}{}
\subsection{{\em list} Short}

There are 4 options: A/C/M/P

The A option lists a single line for each element, species and phases
with some essential data.

The C option lists one line for each component.

The M option lists the models and constitution for all phases.

The P option lists one line for each stable phase and then one line
for some of the remaining phases in decreasing order of stability.

%--------------------------------
\hypertarget{List state variables}{}
\subsection{{\em list} State\_Variables}\label{sc:list_statevar}

Values of individual state variables like G, HM(LIQUID), X(LIQUID,CR)
etc. can be listed.  Terminate the command by an empty line.  Note
that the values of symbols and TP functions cannot be listed here,
they are calculated by the CALCULATE SYMBOL or CALCULATE TP command.

The current values of parameter identifiers, see
section~\ref{sc:paramid} can be listed with the command, like TC(BCC)
will give the calculated Curie temperature for BCC.  A symbol like
MQ\&FE(FCC) will give the logarithm of the mobility of Fe in the FCC
phase.

This command is the same as the SHOW command, section~\ref{sc:show}.

%--------------------------------
\hypertarget{List symbols}{}
\subsection{{\em list} Symbols}

All state variable symbols listed but not their values, they are
calculated by the CALCULATE SYMBOL command.

\begin{verbatim}
List of all state variable symbols
 No Special Name= expression ;
  1         R= 8.31451;
  2         RT= R*T;
  3         T_C= T-273.15;
  4      D  CP= HM.T;
  5      C  DCP= 1
  6     7X  H298= HM;
\end{verbatim}

In the ``special'' column the ``D'' means the symbol that is a ``dot
derivative'' which is calculated only when explicitly specified, ``C''
means a numeric value that can be amended.  The special 7X means a
symbol that is evaluated only at equilibrium 7 which means you can
refer to the value of this symbol calculated at the specified
equilibrium in other equilibria.  See also section~\ref{sc:amendsym}.

%--------------------------------
\hypertarget{List excell CSV}{}
\subsection{{\em list} excell CSV file}

The result from a STEP calculation can be listed in a file using the
Commma Separated Value (CSV) format.  This can be read by Excell or
similar software for later processing.  One may use other state
variables for the table than used for the step command as one can do
for plotting.

{\bf Independent variable:}

The independent variable must be a single valued state variable, for
example $T$.

{\bf Dependent variable(s):}

The dependent variable may have multiple values, for example phase
amounts, NP(*), or the driving force, DGM(\#).

{\bf Output file:}

%--------------------------------
\hypertarget{List TPfun}{}
\subsection{{\em list} Tpfun Symbols}

All or some TPFUN expressions listed.  By giving * all are listed,
bu giving the g* all TP functions starting with G are listed.

Note that all parameters are also TP functions, they can be listed by
giving ``\_*'' as name.  The abbreviation ``\_g*'' will list the
function for all parameters with identifiers starting with G.

To obtain the values of TP functions use the {\bf calculate TP}
command.

%===================================================================
\hypertarget{Macro}{}
\section{Macro }

By specifying a file name commands will be read from that file.  The
default extension is OCM.  A macro file can open another macro file
(max 5 levels).  When a macro file finish with SET INTERACTIVE the
calling macro file will continue or the user can continue
interactively.  See section~\ref{sc:macro}.

When you start OC you can give a macro file name on the same line and
the program will drictly start reading from this file.

With the popup window facility there are some special things.  If you
open the macro file with the popup window OC will save the directory
where the macro file was found.  If there are references to other
files such as datbases or other macro files inside the macro and these
file names are on the same line as the command {\bf read tdb ./steel1}
the file name must be preceeded by a ``./'', otherwise OC will try to
open the file on its ``working directory'', see
section~\ref{sc:popup}.

%===================================================================
\hypertarget{Map}{}
\hypertarget{Map old data}{}
\section{Map }\label{sc:map}

For phase diagram calculations.  You must first set two axis with
state variables which are already set as conditions.

{\bf Reinitiate?}

If you give several MAP commands you can choose to erase or keep the
previous results at each command.

During mapping each calculated equilibria is saved and for plotting
any state variable can be used.

%===================================================================
\section{New }
\hypertarget{New}{}

To remove all data and calculated results to enter a new system.  It
is fragile.

The user must confirm with UPPER CASE Y.

%===================================================================
\hypertarget{Optimize}{}
\section{Optimize}\label{sc:optim}

The command is part of the facility to assess model parameters for
thermodynamic databases.  You have already entered elements, phases
and model parameters with coefficients to be assessed and all the
experimental data yu can find.  Estimated and theoretical data
calculated by DFT can also be entered as experimental data.

The model parameters to optimized are selected by SET VARIABLE\_COEFF
and there is a least square routine LMDIF which will vary these to
obtain the best least fit the experimental data provided.

As already state you must have entered the thermodynamic descriptions
of the phases with model parameters depending on optimizing
coefficients and the experimental data before this command.  You must
also set the weights of the experiments and which coefficents to be
variable.

You provide a maximum number of iterations allowed.  If you give zero
a ``dry run'' will be made with the current values of the optimizing
coefficients.  This is useful to check that there are no problems
calculating the equilibria.  Usually you have to change the set of
model parameters, weights of the experimental data and other criteria
many times before you get a satisfactory result.

Developing better assessment software is one of the main aspects of
the OC software.  There will be more options to this and related
commands.

%===================================================================
\hypertarget{Plot command}{}
\section{Plot }\label{sc:plot}

Plot the result from a STEP or MAP calculation.  A simple interface to
GNUPLOT~\cite{gnuplot} has been implemented in OC.  This generates a
command file which is automatically plotted using GNUPLOT after the
``render'' command.

In OC you must first specify the state variable on the horizontal
(x-axis) and vertical (y-axis) axis.  Then you can give several of the
options below, finish with RENDER or QUIT.

%----------------------------------------------------
\hypertarget{Horizontal axis variable}{}
\subsection{{\em plot} Horizontal axis variable}

Specify the state variable or symbol to be plotted on the horizontal axis.

Note that if you plot a phase diagram with "tie-lines in the plane"
you should specify a fraction variable as X(*,C) and not X(C)
because you want the carbon content in all stable phases.

%----------------------------------------------------
\hypertarget{Vertical axis variable}{}
\subsection{{\em plot xaxis} Vertical axis variable}

Specify the state variable or symbol to be plotted on the vertical axis.

Note that if you plot a phase diagram with "tie-lines in the plane"
you should specify a fraction variable as X(*,C) and not X(C) because
you want the carbon content in all stable phases.

%----------------------------------------------------
\hypertarget{Plot}{}
\hypertarget{Plot options}{}
\subsection{{\em plot xaxis yaxis} Options?/RENDER/}

You can choose various options before plotting. Typing a ? gives a
menu, typing ?? will give this text of the online help is correctly
installed.  The menu here is not very clear and will be reorganized.
The default option is RENDER meaning to plot when you specified all
your options.

The simplest way to generate a complex plot to be saved as PDF or PNG
format is to first select the approriate axis and then set a few
options like scaling, axis texts and text labels and plot on the
screen.  If you are not satified you can plot again (without changing
the axis variables, if you change these all options you have set will
be cleared) and add or modify the options.  When you are satisfied
with the plot on the screen you plot a final time and set the
GRAPHICS-FORMAT option and plot in the desired format on a file.  Or
you can select to plot on a file in the GNUPLOT window.  Note that
some texts and formats may not be exactly identical to those you see
on the screen.

Default plotfile is ``ocgnu.plt''.  On this file all the GNUPLOT
commands and data will be written to be executed by GNUPLOT.  If
GNUPLOT is correctly installed then OC will start GNUPLOT and generate
the graphics output when you RENDER the plot.

You can change the name of the plotfile before plotting with the
command ``output file''.  Whenever you set a new terminal you can also
set the output file name.  Or you can rename the file after the RENDER
command and before you generate a new plot.

GNUPLOT is a very powerful graphics software, only a few of its
facilities are available within OC.  The gnuplot command file
generated by OC can be edited to exploit additional facilities in
GNUPLOT.

\bigskip

{\small
\begin{tabular}{llll}
 APPEND        & FONT              & POSITION\_OF\_KEYS~ & SCALE\_RANGES~ \\
 AXIS\_LABELS~ & GRAPHICS\_FORMAT~ & QUIT                & TEXT\_LABEL \\
 EXTRA         & OUTPUT\_FILE      & RENDER              & TITLE\\  
\end{tabular}
}

A short summary:

\begin{itemize}
\item APPEND means overlay the current plot with another GNUPLOT file
\item AXIS-LABELS you can specify the label on X or Y axis
\item EXTRA provides less frequent plot options
\item FONT select the font for all texts, depend on what GNUPLOT has istalled
\item GRAPHICS-FORMAT to select the GNUPLOT output device (PS, PDF, PNG etc)
  In GNUPLOT plot window there is also an option to save on file.
\item OUTPUT-FILE the GNUPLOT file is saved on this file (default ocgnu.plt)
\item POSITION\_OF\_KEYS, the identification labels for the curves
\item QUIT no plot generated
\item RENDER finally plot
\item SCALE-RANGES for X and Y axis you can specify min and max value plotted
\item TEXT-LABEL you can place a text inside the plot
\item TITLE the heading of the plot (can be suppressed, see EXTRA)
\end{itemize}

The EXTRA command provides less used options:

{\small
\begin{tabular}{llll}
AXIS\_FACTOR     & LINE\_TYPE         & NO\_HEADING   & SPAWN       \\
COLOR            & LOGSCALE           & PAUSE\_OPTION & TIE\_LINES \\
GIBBS\_TRIANGLE~ & LOWER\_LEFT\_TEXT~ & QUIT        \\
GRID             & MANIPULATE\_LINES~ & RATIOS\_XY  \\
\end{tabular}
}

%----------------------------------------------------
\hypertarget{Plot append}{}
\subsection{{\em plot xaxis yaxis} Append}

A GNUPLOT file prevously generated by OC with possible manually
changes or any file following the GNUPLOT standard can be specified to
be overlayed on the current plot.

%----------------------------------------------------
\hypertarget{Plot axis labels}{}
\subsection{{\em plot xaxis yaxis} Axis\_Labels}

You specify for the X or Y axis the axis labels.  By default the state
variable or symbol plotted will be used as label.

{\bf For X or Y axis?}

Specify the axis for which you want to enter the label

{\bf Axis label:}

The default label is given in the question.

%----------------------------------------------------
\hypertarget{Plot fonts}{}
\subsection{{\em plot xaxis yaxis} Font}

%----------------------------------------------------
\hypertarget{Plot formats}{}
\subsection{{\em plot xaxis yaxis} Graphics format}

The GNUPLOT terminals entered in section~\ref{sc:gnuterm} can be used.
For other formats than SCREEN you can also specify an output file
which will be written for the specified format.

Graphics format index:

The default terminal indices are:

% IMPORTANT if table changed here also change in
% \subsection{{\em enter} GNUPLOT Terminal}\label{sc:gnuterm}
\begin{tabular}{rlcl}
  & Name   &=~ & GNUPLOT definition\\
 1~& SCREEN & & set terminal wxt size 940,700 font "arial,16"\\
 2~& PS     & & set terminal postscript color solid fontscale 1.2\\
 3~& PDF    & & set terminal pdf color solid size 6,5 enhanced font "arial,16"\\
 4~& GIF    & & set terminal gif enhanced fontscale 0.7\\
 5~& PNG    & & set terminal png enhanced fontscale 0.7\\
\end{tabular}

You can change these or enter more graphics formats with the {\bf
  enter gnuplot} command. \ref{sc:gnuterm}.  The SCREEN driver is
usually ``wxt'' for Windows and ``Qt'' for Linux but can be selected in
the Makefile for the pmon6.F90 file.

If SCREEN is not selected the you can specify the name of the file
where OC will save the commandfile for GNUPLOT as well as the final
graphics file created by GNUPLOT.  It will have the appropriate
extention depending on the format.  By default OC saves the GNUPLOT
command file on the file ``ocgnu.plt''.  This can be renamed and
edited if you want to keep it for later processing.

Plot file:

In addition to the GNUPLOT command file the graphics a file with the
specified format will be generated.

%----------------------------------------------------
\hypertarget{Plot file}{}
\subsection{{\em plot xaxis yaxis} Output file}

By default plotting will generate a ocgnu.plt file for GNUPLOT.  You
can specify other name here.  If you plot on other terminals than
SCREEN there will be an additional file with extension ``.ps'' for
Postscript, ``.pdf'' for Adobe PDF or ``.gif'' for GIF format.

If the file already exists the user must confirm it it should be
overwritten.

%----------------------------------------------------
\hypertarget{Plot keys}{}
\subsection{{\em plot xaxis yaxis} Position of keys}

The identification (labels) of the curves in the plot can be
positioned with this command.  See the GNUPLOT manual~\cite{gnuplot}
for information.

%----------------------------------------------------
\hypertarget{Plot quit}{}
\subsection{{\em plot xaxis yaxis} Quit}

No plot generated.

%----------------------------------------------------
\hypertarget{Plot render}{}
\hypertarget{Render}{}
\subsection{{\em plot xaxis yaxis} Render}

Press return to plot using all the option set.  Otherwise you
can select any of these options:

{\small
\begin{tabular}{llll}
 APPEND        & FONT              & POSITION\_OF\_KEYS~ & SCALE\_RANGES~ \\
 AXIS\_LABELS~ & GRAPHICS\_FORMAT~ & QUIT                & TEXT\_LABEL \\
 EXTRA         & OUTPUT\_FILE      & RENDER              & TITLE\\  
\end{tabular}
}

%----------------------------------------------------
\hypertarget{Plot limits}{}
\subsection{{\em plot xaxis yaxis} Scale\_Range}

You specify for the X or Y axis the minimum and maximum range.  The
automatic (default) scaling range can always be restored.

%----------------------------------------------------
\hypertarget{Plot texts}{}
\subsection{{\em plot xaxis yaxis} Text}

This is a facility to add a text to a plot at an arbitrary position.

\subsubsection{{\em plot xaxis yaxis text} Modify existing text?:}

If there is already a text item you must first answer if you wants
modify an already existing one.  If so all the texts are listed and
you can select which one you wants to change.

\subsubsection{{\em plot xaxis yaxis text} Which text index?:}

You must provide the index of an existing text to change.

For a new or changed text you must give:

\subsubsection{{\em plot xaxis yaxis text}  X position}

The X coordinate of the text (in the plot scale)

\subsubsection{{\em plot xaxis yaxis text} Y position}

The Y coordinate of the text (in the plot scale)

\subsubsection{{\em plot xaxis yaxis text} Fontscale}

A relative size factor, default is 0.8.  The size of the text will be
scaled accordingly.

\subsubsection{{\em plot xaxis yaxis text} Angle (degrees)}

The text will be written with the specified angle. Zero means
horisontally, negative valus slopes downward, positive upwards. An
ange of 180 means the text will be upside down.

\subsubsection{{\em plot xaxis yaxis text} Do you want to calculate the
equilibrium?/Y/}

If you are plotting a phase diagram you can select to calculate an
equilibrium at the specified coordinates.  The names of the stable
phases will be proposed as text.

The calculation may fail and you can anyway add a text.  Note that the
axis values you sepcified will refer to the axis used when calculating
the diagram.  If you are plotting using other variables there may be
some surprises.

\subsubsection{{\em plot xaxis yaxis text} Text: }

The text to be added to the plot.  The text will start at the
coordinates given.  On Postscript and PDF a greek character can be
given as ``{/Symbol m}'' for $\mu$.

%----------------------------------------------------
\hypertarget{Plot title}{}
\subsection{{\em plot xaxis yaxis} Title}

The default is the date and the conditions.  You can add a text of
your own here.  You can remove the title altogether with EXTRA
NO\_HEADING.  That will make the figure slightly larger.

%----------------------------------------------------
\hypertarget{Extra}{}
\hypertarget{Plot extra}{}
\hypertarget{Extra Gibbs_triangle}{}
\subsection{{\em plot xaxis yaxis} Extra }

Less common options for the plotting is available here.  For really
nice plotting it is recommended to edit the output file from
OC as GNUPLOT has too many facilities to be made available here.

\hypertarget{Extra options}{}
The EXTRA commands provides more obscure options:

{\small
\begin{tabular}{llll}
AXIS\_FACTOR     & LOGSCALE           & PAUSE\_OPTION & TIE\_LINES \\
COLOR            & LOWER\_LEFT\_TEXT~ & QUIT        \\
GIBBS\_TRIANGLE~ & MANIPULATE\_LINES~ & RATIOS\_XY  \\
LINE\_TYPE       & NO\_HEADING        & SPAWN       \\
\end{tabular}
}

\begin{itemize}
\item AXIS\_FACTOR means all values on an axis will be multiplied with
  this.  For example it can be useful to plot in kJ rather than the default J.
\item COLOR you can select some colors
\item GIBBS-TRIANGLE means an equilateral triangular diagram
\item LINE-TYPE means dashed lines or lines with symbols
\item LOGSCALE you can specify that X or Y axis is logaritmic
\item LOWER-LEFT-TEXT you can set a text in the lower left corner
\item MANIPULATE-LINES does not work
\item NO-HEADING means remove title all text above the plot
\item PAUSE-OPTION to select how GNUPLOT should behave after plotting
\item QUIT no extra option selected
\item RATIOS-XY will change the relative length of X and Y axis
\item SPAWN will allow you to contine calculating with the plot window open
\item TIE-LINES if you have tie-lines in the plane you can plot some of them
\end{itemize}

%----------------------------------------------------
\hypertarget{Plot extra factor}{}
\subsubsection{{\em plot xaxis yaxis extra} factor}

You can select a factor for each plot axis to convert from J to kJ for
example.

%----------------------------------------------------
\hypertarget{Plot color}{}
\subsubsection{{\em plot xaxis yaxis extra} color}

You can select color of monovariant equilibria and tie-lines.

%----------------------------------------------------
\hypertarget{Plot Gibbs triangle}{}
\subsubsection{{\em plot xaxis yaxis extra} Gibbs-triangle}

Gibbs triangle plots should only be used for isothermal sections.  A
trial implementation is available which can generate equiaxial
triangular isothermal diagrams.

If you already set this option you can set it again to plot on a square.

%----------------------------------------------------
\hypertarget{Plot line symbols}{}
\subsubsection{{\em plot xaxis yaxis extra} line-with-symbols}

Not implemented yet

%----------------------------------------------------
\hypertarget{Plot logax}{}
\subsubsection{{\em plot xaxis yaxis extra} logscale}

You can set logarithimic scale on X or Y axis (or both).

%----------------------------------------------------
\hypertarget{Extra line-colors}{}
\subsubsection{{\em plot xaxis yaxis extra} manipulate lines}

This is not implemented.  It is intended to allow specification of the
color of the curves in the plot.

%----------------------------------------------------
\hypertarget{Extra lower-left-corner}{}
\subsubsection{{\em plot xaxia yaxis extra} lower left corner text}

You can set a short text in the lower left corner of the plot

%----------------------------------------------------
\hypertarget{Plot spawn}{}
\subsubsection{{\em plot xaxia yaxis extra} spawn}

You can spawn the plot window and continue working looking at it.

%----------------------------------------------------
\hypertarget{Plot no heading}{}
\subsubsection{{\em plot xaxia yaxis extra} no heading}

Remove the text above the plot with date and title.  The plot is
slightly larger this way.

%----------------------------------------------------
\hypertarget{Plot pause}{}
\subsubsection{{\em plot xaxis yaxis extra} pause option}

When you plot on the screen the last command on the file to GNUPLOT
is ``pause mouse''.  You can change this with this command.

%----------------------------------------------------
\hypertarget{Plot ratios}{}
\subsubsection{{\em plot xaxis yaxis extra} ratios XY}

The relative ratios of the X and Y axis can be specied.

%----------------------------------------------------
\hypertarget{Plot tieline}{}
\subsubsection{{\em plot xaxis yaxis extra} tie-line}

Tie-lines in isothermal ternary phase diagram can be plotted.  You
can specify the density of the tie-lines by

Tie-line plot increment?

The increment is related to the actual equilibria calculated.  0 means
no tie-lines plotted, 3 means to plot a tie-line at every 3rd
calculated equilibria and so on.

%===================================================================
\hypertarget{Quit}{}
\section{Quit }

Terminate the OC software in English, have a nice day.

%===================================================================
\hypertarget{Read}{}
\section{Read }

It is possible to read a (non-encrypted) TDB file but it should be not
too different from what is normally generated by the LIST\_DATA
command in TC.

\begin{tabular}{lll}
 DIRECT~        & QUIT~            & TDB\\
 PDB            & SELECTED-PHASES  & UNFORMATTED \\
\end{tabular}

%--------------------------------
\hypertarget{Read direct}{}
\subsection{{\em read} Direct}

{\bf File name:}

In the future it will be possible to save results on a random access
(DIRECT) file.

%--------------------------------
\hypertarget{Read PDB}{}
\subsection{{\em read} PDB}

{\bf File name:}

A PDB file (with extension PDB) should be specified.  The file
should be un the Portable phase dependent Data Base format.

The user can select to read the whole file or select elements.

%--------------------------------
\hypertarget{Read quit}{}
\subsection{{\em read} Quit}

You did not really want to read anything.

%--------------------------------
\hypertarget{Read select phase}{}
\subsection{{\em read} selected phases only}

This is to select a subset of elements and phases from a database.
Normally all phases which can be formed by the elements are included.
With this command one can first select the elements and after that one
can specify the phases to be included.  If one specifies an
abbreviation of a phase name all phases which fit this abbreviation
will be selected.

{\bf Database format:}

Can be TDB or PDB.

{\bf File name:}

It is also possible to read all phases and later suspend those which
are not interesting.

%--------------------------------
\hypertarget{Read TDB}{}
% on command level the subcommand may be lower case!!
\hypertarget{Read tdb}{}
\subsection{{\em read} TDB}

A TDB file (with extension TDB) should be specified.  The TDB file
must not deviate very much from the standard output from Thermo-Calc.

\hypertarget{File name:}{{\bf File name:}}

If you do not use the popup window for opening files you must specify
the database file name.  The file must be on the working directory
(where you started the OC program, see section~\ref{sc:popup}) or you
must provide the path.

\hypertarget{Select element}{}
After opening the file the program will list the elements and ask:

{\bf Select elements /all/:}

If you give RETURN the data for all elements will be read.  If you
answer q or quit nothing will be read.  If you specify one or more
elements the data for those will be read and if you selected a subset
you will have the question:

{\bf Select elements /no more/:}

And you can select some more or just give RETURN (or type quit).  All
phases that can be formed by the elements selected will be read, you
cannot select the phases here but inside OC you can suspend those
phases you are not interested in.

\hypertarget{Read TDB error}{{\bf Error reading TDB file}}
  
In some cases there non-fatal errors or warnings reading TDB files
created by different groups because the TDB format varies a lot.  The
user should carefully check if there are any data missing but can
continue using the data he read if he is confident it is correct.  The
TDB file should be corrected manually.

%--------------------------------
\hypertarget{Read unformatted}{}
\subsection{{\em read} Unformatted}\label{sc:readunf}

{\bf File name:}

For use to read a file created with a SAVE UNFORMATTED command.  It
may not always work to read an old unformatted file as the data
structure is still changing.

%===================================================================
\hypertarget{Save}{}
\section{Save }

There are several forms of save, three forms write a text file that
can be read and modified with a normal editor.  Two forms are
unformatted, either on a sequential file or a direct (random access)
file.

\begin{tabular}{llll}
  DIRECT~          & SOLGAS~ & UNFORMATTED\\
  QUIT            & TDB    & PDB \\
\end{tabular}

%--------------------------------
\hypertarget{Save direct}{}
\subsection{{\em save} Direct}

It will eventually be possible to save the result of STEP and MAP
commands on a random access file for later processing.

%--------------------------------
\hypertarget{Save quit}{}
\subsection{{\em save} Quit}

You did not want to save anything.

%--------------------------------
\hypertarget{Save PDB}{}
\subsection{{\em save} PDB}

Saves current set of model parameters and functions on a file in the
Portable phase dependant Data Base format. 

%--------------------------------
\hypertarget{Save TDB}{}
\subsection{{\em save} TDB}

Saves current set of model parameters and functions on a file in TDB
format.  Same as the command {\bf list data tdb}.

%--------------------------------
\hypertarget{Save SOLGAS}{}
\subsection{{\em save} SOLGAS}

Saves current set of model parameters and functions on a file in a
format that (hopefully) can be read by the FactSage software.

%--------------------------------
\hypertarget{Save unformatted}{}
\subsection{{\em save} Unformatted}\label{sc:saveunf}

With this command you can save the current status of the calculations
on a file and then resume the calculations by reading this file.  Note
that the Fortran unformatted files may not be portable, they depend on
the compiler, the operating system and the hardware.

%===================================================================
\hypertarget{Select}{}
\section{Select }

There are a few things that can be selected, most important which
equilibrium the following commands will operate on.

%--------------------------------
\hypertarget{Select equilibrium}{}
\subsection{{\em select} Equilibrium}

As you can enter several equilibria with different conditions
this command allows him to select the current eqilibria.

%--------------------------------
\hypertarget{Select graphics}{}
\subsection{{\em select} Graphics}

Only GNUPLOT~cite{gnuplot} available.

%--------------------------------
\hypertarget{Select language}{}
\subsection{{\em select} Language}

Only English implemented (except a few French exclamations).

%--------------------------------
\hypertarget{Select minimizer}{}
\subsection{{\em select} Minimizer}

Only Hillert's algorithm implemented in matsmin~\cite{15Sun2} available.

%--------------------------------
\hypertarget{Select optimizer}{}
\subsection{{\em select} Optimizer}

The LMDIF~\cite{lmdif} least square fitting software is the only one
implemented.

%######### >>> edit limit 2018.01.10

%===================================================================
\hypertarget{Set}{}
\section{Set }

Many things can be set.  Things to be ``set'' and ``amended''
sometimes overlap.

{\small
\begin{tabular}{llll}
 ADVANCED          & FIXED\_COEFF     & OPTIMIZING\_COND~ & STATUS\\
 AS\_START\_EQUILIB~&INITIAL\_T\_AND\_P~ & PHASE         & SYSTEM\_VARIABLE\\
 AXIS              & INPUT\_AMOUNTS   & QUIT             & UNITS\\ 
 BIT               & INTERACTIVE      & RANGE\_EXP\_EQUIL~ & VARIABLE\_COEFF\\
 CONDITION         & LOG\_FILE        & REFERENCE\_STATE & VERBOSE\\
 ECHO              & NUMERIC\_OPTIONS~ & SCALED\_COEFF    & WEIGHT\\\\
\end{tabular}
}

%--------------------------------
\hypertarget{Advanced command}{}
\subsection{{\em set} Advanced}

A few options implemented

{\small
\begin{tabular}{llll}
 EEC\_METHOD      & HELP-POPUP-OFF~& OPEN-POPUP-OFF~ & WORKING-DIRECTRY\\ 
 EQUILIB-TRANSF~  & LEVEL          & QUIT\\
 GLOBAL-MIN-ONOFF~& MAP-SPECIALS   & SMALL-GRID-ONOFF~\\
 GRID-DENSITY     & NO-MACRO-STOP  & SYMBOL \\
\end{tabular}
}

%.............................................................
\hypertarget{Set adv EEC-method}{}
\subsubsection{{\em set advanced} EEC-method}\label{sc:eec-method}

In a recent paper\cite{20Sun} a method the compare the entropy of
the liquid and a solid phase can be used to supress the formation of a
solid phase at high $T$ if its entropy is higher than the liquid, the
Equi-Entropy Criteria (EEC).  This simplifies the extrapolation of the
Gibbs energy of solids at high $T$.

This command will activate or deactivate this check.

%.............................................................
\hypertarget{Set adv transfer}{}
\subsubsection{{\em set advanced} equilibrium transfer}

This is only for experts who know what they are doing.

%.............................................................
\hypertarget{Set adv global onoff}{}
\subsubsection{{\em set advanced} global-min-onoff}

Turn on or off the use of the global gridminimizer.

%.............................................................
\hypertarget{Set adv grid-density}{}
\subsubsection{{\em set advanced} grid\_density}

At present the grid density cannot be fine tuned.  For some phases it
is fixed for others you can select a more or less dense grid.

Note that phases with option F or B (4 sublattice order/disorder)
there is a special grid minimizer and also for solids with ionic
constituents and for the 2-sublattice ionic liquid.

%.............................................................
\hypertarget{Set adv help popup}{}
\subsubsection{{\em set advanced} help-popup-off}\label{sc:help-popup}

The user can turn off or on the HTML popup help feature.  He can also
change the browser and help file.

{\bf Turn off popup help? /Y/:}

If the user answers N he will be asked for the browser and HTML file.
These are normally set when compiling the OC software and their
current values are proposed as default within slashes /../.

{\bf Browser including full path //usr/local/firefox/:}

{\bf HTML help file includig full path //home/user/.ochelp/ochelp.html/:}

%.............................................................
\hypertarget{Set adv level}{}
\subsubsection{{\em set advanced} level}

You can specify if you are beginner or expert.  You may have to
declare youself as expert to execute some commands.  The intention of
the beginners status is to provide more help but that is not yet
implemented.

%.............................................................
%\hypertarget{SET Advanced map-special}{}
\subsubsection{{\em set advanced} map-special}

Not implemented yet.

%.............................................................
\hypertarget{Set adv no-macro-stop}{}
\subsubsection{{\em set advanced} no-macro-stop}

This command makes it possible to ignore the ``@\&'' used to stop the
execution of a macro file.  Used when testing the software.

%.............................................................
\hypertarget{Set adv open popup}{}
\subsubsection{{\em set advanced} open-popup-off}

Any other answer than Y will turn off popup windows for opening files.
By answering Y you turn on popup windows for opening files (the
default) provided the program is linked with this facility.

%.............................................................
\hypertarget{Set adv quit}{}
\subsubsection{{\em set advanced} quit}

You did not want to set anything advanced.

%.............................................................
\hypertarget{Set adv symbol}{}
\subsubsection{{\em set advanced} symbol}

Not implemented yet.

%.............................................................
\hypertarget{Set adv workdir}{}
\subsubsection{{\em set advanced} working-directory}

The name of the working directory (where OC was started) is listed.
It cannot be changed at present.  It is related to the popup windows
for opening files, see section~\ref{sc:popup}.

%--------------------------------
\hypertarget{Set as start equil}{}
\subsection{{\em set} As start equilibrium}

The current equilibrium will be copied to the list of start equilibria
for STEP and MAP commands.

%--------------------------------
\hypertarget{Set axis}{}
\subsection{{\em set} Axis}\label{sc:setaxis}

To set an axis you must first has set the conditions necessary to
calculate an equilibrium and also calculated this.

{\bf Axis number:}

The axis are numbered 1, 2 etc and you must set them in sequential
order.  To change an axis variable just give the number of the axis to
change.

{\bf Condition to vary along the axis:}

You can set select one of the condition to vary between a min and max
value along the axis.  If you has just one axis you can use STEP to
calculate a property diagram, i.e. how the system properties varies
with a single variable.  Typically a phase fraction plot or how the
heat capacity varies with the independent axis variable.

{\bf Minimal/maximal value of the axis:}

The calculation will start with the current value and calculate in
both directions.

{\bf Increment:}

By default the increment is 1/40 of the difference beteen max and min.

If you set two or more axis (current limit is 2) the OC software will
map the phase diagram, i.e. follow the lines where the set of phases
changes.  This means OC will replace one axis condition with a condition
that a phase should be stable with zero amount.

To calculate a diagram you must then give a STEP command (if you have
one axis) or a MAP command (if you have 2 or more axis).  For the STEP
command~\ref{sc:step}, there are several options.

%--------------------------------
\hypertarget{Set which status}{}
\subsection{{\em set} Bit}

%%%  ???
\hypertarget{helphere}{}
%%%  ???
\hypertarget{Set status bit}{}
\hypertarget{Global status bits}{}

Many records have status words where the bits are used to signify
different things.  An advanced user can set these bits for the global,
equilibrium and phase records, but only if you know what it means.
\begin{itemize}
\item The GLOBAL record bits are listed below.  Most of them are
  set or reset automatically by the software or by other commands.
  \begin{itemize}
  \item   0  you are a beginner
  \item   1  you are experienced (default)
  \item   2  you are an expert
  \item   3  gridminimizer must not be used
  \item   4  gridminimizer must not merge comp.sets.
  \item   5  there are no data (cleread automatically)
  \item   6  there are no phases (cleared automatically)
  \item   7  comp.sets must not be created automatically
  \item   8  comp.sets must not be deleted automatically
  \item   9  data has changed since last save (set automtically)
  \item  10  means verbose is on (not implemented)
  \item  11  means verbose is permanently on (not implemented)
  \item  12  means be silent (supress warnings)
  \item  13  no cleanup after an equilibrium calculation
  \item  14  use denser grid in grid minimizer (see also SET ADVANCED)
  \item  15  calculations in parallel is not allowed
  \item  16  no global test at node point during STEP/MAP
  \item  17  the components are not the elements
  \item  18  global test of equilibrium AFTER calculation
  \item  19  use old (less dense) grid minimizer
  \item  20  do not recalculate if global test AFTER fails
  \item  21  use old MAP algorithm
  \item  22-31 not yet used
  \end{itemize}
%--------------------------------------------------------------
\item The EQUILIBRIUM record bits are listed below
  \begin{itemize}
  \item   0  No threads allowed (no parallel calculation)
  \item   1  No global minimization allowed for this equilibrium
  \item   2  No equilibrium has been calculated (there are no results)
  \item   3  Conditions and results not consistent
  \item   4  Last equilibrium calculation failed
  \item   5  No automatic generation of composition sets
  \item   6  Equilibrium tested by grid minimizer
  \item   7  Current results are from a grid minimization
  \end{itemize}
%--------------------------------------------------------------
\item To change the phase status word use SET PHASE ... bit
\end{itemize}
%--------------------------------
% the default is set condition
\hypertarget{Info conditions}{}
\hypertarget{Set condition}{}
\subsection{{\em set} Condition}\label{sc:setcond}

Most of the text here also applies to {\bf enter experiment}.

{\bf State variable:}

A condition is a value assigned to a state variable or an expression
of state variables.  All state variables are listed in
Table~\ref{tab:statev} in section~ref{sc:statevar}

By setting the status of a phase to fix you have
also set a condition.  For example

{\em set cond t=1273 p=1e5 n=1 x(cr)=0.1 w\%(c)=1}

Three cases of expressions can be used as conditions, for example a
relation between mole fraction like\\
{\bf set condition x(liq,o)-x(c1\_mo2,o)=0}\\
means that the oxygen content in liquid and c1\_mo2 phases should be the
same.  That is useful to calculate the congruent melting of c1\_mo2.

Another case is if the total anount if some components has a relation,
for example:\\ 
{\bf set condition n(u)+n(zr)=1}\\ 
means that the total number of moles of the components U and Zr should
be unity.

A third case is {\bf y(B2,Al)-y(B2,Al\#2)=0.01} to calculate a send
order transition line when the B2 ordered phase is on the limit of
disorder as the fractions of Al on the two sublattices are almost
equal.

%subsubsection{value}
{\bf Value:}

A numeric value or a symbol representing a constant value is expected.

%--------------------------------
\hypertarget{Set echo}{}
\subsection{{\em set} Echo}

This is useful command in macro files or when demonstrating the program.

%--------------------------------
\hypertarget{Set fix coeff}{}
\subsection{{\em set} Fixed coefficient}

One or more optimizing coefficients are assigned a fixed value.  The
index 0 to 99 is used to indicate the coefficients A00 to A99.  One
can use a range as 15-19 to set all variable cofficients in the range
to their current values.

%--------------------------------
\hypertarget{Set initial TP}{}
\subsection{{\em set} initial\_T\_and\_P}

Local values of T and P can be set.  These are not conditions but are
used for commands like {\bf CALCULATE PHASE ...}.

%--------------------------------
\hypertarget{Set input amounts}{}
\subsection{{\em set} Input-Amounts}\label{sc:setinpuam}

This command allows you to specify a system by giving a redundant
amount of various species in the system.  The software will transform
this to conditions on the amounts of the components.

\hypertarget{Species and amounts}{}
Species and amount as N(..)= or B(...)= :

An example:

{\small
\begin{verbatim}
--->OC5:read tdb cho-gas
--->OC5:set input 
Species and amount as N(..)= or B(...)= : n(c1o2)
Amount: 10
--->OC5:set input n(c1h4)=5
--->OC5:l c
Conditions for equilibrium:   1, DEFAULT_EQUILIBRIUM
  1:N(C)=45, 2:N(O)=80, 3:N(H)=30
 Degrees of freedom are   2
\end{verbatim}
}

The amounts of the species has been split on the components.  Setting
input amounts is just another way to set these directly.  If we set a
$T$ and $P$ we can calculate the equilibrium fraction of all the
species.

{\small
\begin{verbatim}
--->OC5:set c t=1000 p=1e5
--->OC5:l c
Conditions for equilibrium:   1, DEFAULT_EQUILIBRIUM
  1:N(C)=45, 2:N(O)=80, 3:N(H)=30, 4:T=1000, 5:P=100000
 Degrees of freedom are   0
--->OC5:c e
 3Y Constitution of metastable phases set
Gridmin:      85 points   1.56E-02 s and       0 clockcycles, T= 1000.00
Phase change: its/add/remove:     5   11    0
Phase change: its/add/remove:    12   12    0
Phase change: its/add/remove:    17    0   12
Phase change: its/add/remove:    53    0   11
Equilibrium calculation   79 its,   7.8125E-02 s and      93 clockcycles
--->OC5:l
LIST what? /RESULTS/:
Results output mode: /1/:

Output for equilibrium:   1, DEFAULT_EQUILIBRIUM          2018.08.21
Conditions .................................................:
  1:N(C)=45, 2:N(O)=80, 3:N(H)=30, 4:T=1000, 5:P=100000
 Degrees of freedom are   0

Some global data, reference state SER ......................:
T=   1000.00 K (   726.85 C), P=  1.0000E+05 Pa, V=  4.9872E+00 m3
N=   1.5500E+02 moles, B=   1.8507E+03 g, RT=   8.3145E+03 J/mol
GS= -2.80411E+07 J, GS/N=-1.8091E+05 J/mol, HS=-1.2914E+07 J, SS= 1.513E+04 J/K

Some data for components ...................................:
Component name    Moles      Mole-fr  Chem.pot/RT  Activities  Ref.state
C                 4.5000E+01  0.29032 -3.7354E+00  2.3863E-02  SER (default)
H                 3.0000E+01  0.19355 -9.8098E+00  5.4910E-05  SER (default)
O                 8.0000E+01  0.51613 -3.6377E+01  1.5911E-16  SER (default)

Some data for phases .......................................:
Name                Status Moles      Volume    Form.Units Cmp/FU dGm/RT  Comp:
GAS..................... E  1.550E+02  4.99E+00  6.00E+01    2.58  0.00E+00  X:
 O      5.16129E-01  C      2.90323E-01  H      1.93548E-01
 Constitution: There are    73 constituents:
 C1O2         4.54395E-01  C2H3         8.67456E-17  C4H10_1      2.73242E-23
 C1O1         2.95682E-01  C3H4_2       3.04922E-17  C4H10_2      1.38822E-23
 H2O1         1.29270E-01  C3H8         2.73523E-17  C4H2         8.16657E-24
 H2           1.20501E-01  C3H6O1       1.94895E-17  H1O2         4.37267E-24
 C1H4         1.52786E-04  C3H4_1       8.18695E-18  C4H6_5       1.44915E-24
 C1H2O2_CIS   4.04887E-08  C1H3O1_CH3O  3.87833E-18  C4H8         1.04297E-25
 C1H2O1       2.01368E-08  C2H4O1_OXIRA 1.64221E-19  C2H1         7.79712E-26
 C1H2O2_TRANS 5.82767E-09  C1H2         3.98656E-20  C4H8_4       6.39692E-26
 H            7.88542E-10  H2O2         3.27068E-20  C6H6O1       3.00598E-26
 C1H4O1       1.27636E-10  O            1.46838E-20  C1H1         1.81712E-27
 C2H4         1.05140E-10  C2H6O2       1.19305E-20  C3H1         1.68523E-28
 C2H6         3.44726E-11  O2           8.71930E-21  C4H4_1_3     7.73762E-29
 C1H3         1.83302E-11  C4H6_2       5.73533E-21  C1H2O2_DIOXI 4.04963E-30
 C1H1O1       7.24719E-12  C2O1         1.72590E-21  C4H1         1.00000E-30
 C2H4O1_ACETA 2.00054E-12  C4H8_5       9.38081E-22  C2H4O2_DIOXE 1.00000E-30
 H1O1         1.86354E-12  C4H8_3       5.91323E-22  C4           1.00000E-30
 C2H2         1.82837E-12  C4H8_1       4.75317E-22  C2H4O3_123TR 1.00000E-30
 C1H1O2       1.57298E-12  C4H8_2       4.17043E-22  C2H4O3_124TR 1.00000E-30
 C2H4O2_ACETI 7.65642E-13  C2H2O1       1.47405E-22  C2           1.00000E-30
 C1H3O1_CH2OH 1.64978E-15  C4H6_4       8.47392E-23  C60          1.00000E-30
 C3O2         1.11079E-15  C6H6         8.21607E-23  C3           1.00000E-30
 C3H6_2       7.21243E-16  C4H4         5.46648E-23  C5           1.00000E-30
 C3H6         7.13743E-16  C4H6_1       5.05773E-23  O3           1.00000E-30
 C2H6O1       6.22811E-16  C4H6_3       2.87604E-23
 C2H5         4.72671E-16  C4H10_1      2.73242E-23

--->OC5:
\end{verbatim}
}

The calculation shows that mixing 10 moles of CO$_2$ with 5 moles of
CH$_4$ at 1000~K and 1~bar gives a gas with 45\% CO$_2$, 30\% CO, 13\%
H$_2$O and the rest H$_2$

%--------------------------------
\hypertarget{Set interactive}{}
\subsection{{\em set} Interactive}

The last command on a macro file.  Gives command back to the keyboard
of the user, or to the calling macro file.  Without this the program
will just terminate when the macro is finished.

%--------------------------------
\hypertarget{Set logfile}{}
\subsection{{\em set} Log-File}

A useful command to save all interactive input while running OC.  The
log file can easily be transformed to a macro file.  All bug reports
should be accompanied by a log file which reproduces the bug.

%--------------------------------
\hypertarget{Set numeric}{}
\subsection{{\em set} Numeric-Options}

The default number of iterations and accuracy can be specified.
Default values are 500 and 10$^{-6}$.

Some more obscure values may also be asked for, they should never be
changed.

%--------------------------------
\hypertarget{Set optimizer conditions}{}
\subsection{{\em set} Optimizing conditions}

A few variables used to guide the optimization of model parameters can
be set.

%--------------------------------
\hypertarget{Set system variable}{}
\subsection{{\em set} system variable}

This is a new idea to have global variables.  No idea how to use it
yet.

%--------------------------------
\hypertarget{Set phase}{}
\hypertarget{Set for phase}{}
\subsection{{\em set} Phase ``phase-name''}

You must specify a phase name.  Some phase specific things can be set,
also for the model.  Some subcommands allow wildcard ``*'' as name.

%....................
\hypertarget{Set phase amount}{}
\subsubsection{{\em set phase} ... Amount}

You can specify the amount of the phase which is used as initial value
for an equilibrium calculation.

%....................
\hypertarget{Set phase bits}{}
\subsubsection{{\em set phase} ... Bits}

Some of the models and use of data storage depend on the bits of the
phase.  Most of them are set automatically by the software and other
commands like AMEND PHASE.  Changing them with this command will not
have the expected effect and may cause the program to fail.

The bits that can be changed are:

\hypertarget{Extra-dense-grid}{}
\begin{itemize}
\item EXTRA\_DENSE\_GRID makes it possible to have a larger number of
  gridpoints calculated by the gridminimizer for the specified phase.

%. . . . . . . . . .
\hypertarget{No-auto-comp-set}{}
\item NO\_AUTO\_COMP\_SET.  This makes it possible to prevent that the
specific phase has automatic composition set created during
calculations.

%. . . . . . . . . .
\hypertarget{Set bit quit}{}
\item QUIT, do not set any more bits.
\end{itemize}

%....................
\subsubsection{{\em set phase} ... Constitution}
\hypertarget{Amend phase constit}{}
\hypertarget{Set phase constitution}{}

This is the same as {\bf amend phase constitution}.  The amount of the
phase can also be set.  You can specify the constituent fraction of
each constituent.  A fraction must be larger than zero and less than
unity.

As the sum of fractions must be unity the last constituent in each
sublattice will not be asked for unless you specify the fraction for
one of the constitents as ``rest''.  The fraction of that will then be
set as ``the rest'' i.e. one minus the sum of the other fractions.

This is also be used for the command {\bf
  calculate phase} to calculate properties for a single phase.

%....................
\hypertarget{Set phase ... default-constitu}{}
\subsubsection{{\em set phase} ... Default-constitution}

Same as {\bf amend phase default\_constit}.

%....................
\hypertarget{Set phase ... quit}{}
\subsubsection{{\em set phase} ... Quit}

You did not want to set anything for the phase.

%....................
\hypertarget{Set phase status}{}
\subsubsection{{\em set phase} ... Status}\label{sc:setphstat}

Use the SET STATUS PHASE command to set the status of one or several
phases.  The different status are explained for that command,
section~\ref{sc:set-status-phase}.

A phase with the status FIX must also have an amount specified.  For a
phase with the status ENTERED the amount is also requested but
normally it should be set to zero.  A nonzero value means the user
assumes the phase should be stable.

%--------------------------------
\hypertarget{Set quit}{}
\subsection{{\em set} Quit}

You did not really want to set anything.

%--------------------------------
\hypertarget{Set range}{}
\subsection{{\em set} Range of experimental equilibria}\label{sc:setrange}

For an assessment several consequtive equilibria with experimental
data must be entered.  This command specifies the first and last of
those equilibria.  It possible to add more equilibria later one by
one (not yet though).  

\hypertarget{First equilibrium number:}{{\bf First equilibrium number: /2/:}}

{\bf Last equilibrium number:}

The equilibria are assigned the weight one by default.  The weight can
be changed with the SET WEIGHT command.  The weight zero means the
equilibrium is not calculated.

%--------------------------------
\hypertarget{Set reference phase}{}
\subsection{{\em set} Reference-State}\label{sc:setref}

By default the reference state for the components is SER (Stable
Element Reference) which is the stable state of the element at
298.15~K and 1~bar.  (NOTE: in principle SER is defined by the database
but today almost all databases have SER as reference state.)

{\bf Component name:}

{\bf Reference phase:}

For each component (also for other components than the elements) you
can specify a phase at a given temperature and pressure as reference
state.  The phase must exist for the component as pure.

{\bf Temperature:}

Instead of a fixed $T$ you can give a *, indicating current $T$, if
you calculates at different values of $T$.

{\bf Pressure:}

Example:

{\em set reference O gas * 1e5}

Note that state variables like the chemical potential, MU(O), will
refer to the user defined reference state.  To obtain the value for
the SER state you can use the suffix S, i.e. MUS(O) will give the
chemical potential refered to SER.

IMPORTANT NOTE: the value of integral properties like Gibbs energy,
$G$, enthalpy, $H$, etc. will also be affected by the change of the
reference state of an element.  If all elements have the same phase as
reference state the value of the enthalpy obtained by $H$ for that
phase will be the enthalpy of mixing.  If not it is only confusing.

In order to have use SER as reference state use a suffix S.  The
enthalpy relative to SER is $HS$ independent of any reference state
set for the elements by the user.

%--------------------------------
\hypertarget{Set scaled coefficient}{}
\subsection{{\em set} Scaled coefficient}

A coefficient for optimization can be specified with a start value,
scaling factor and a minimum and maximum value.  The {\em set} VARIABLE
command sets the scaling factor equal to the start value and have no
min or max values.

Not implemented yet.

%--------------------------------
\hypertarget{Set status}{}
\subsection{{\em set} Status}

The status of elements, constituents, species or phases can be
changed.  Only phases are implemented.

%....................
\hypertarget{Set status constituent}{}
\subsubsection{{\em set status} Constituent}

A constituent of a phase can be suspended.  Not yet implemented.

%....................
\hypertarget{Set status element}{}
\subsubsection{{\em set status} Element}

An element can be ENTERED or SUSPENDED.  If an element is suspended
all species with this element is automatically suspended.  If such a
species is the single constituent of a phase that phase is also
suspended.

Not yet implemented.

%....................
\subsubsection{{\em set status} Phases}\label{sc:set-status-phase}
\hypertarget{Set status phase}{}

\hypertarget{Status phase}{{\bf Phase name(s):}}

A phase can have one of 4 different status

\begin{itemize}
\item ENTERED, this is the default.  The phase will be stable if that
  would give the most stable state for the current conditions.  The user
  can give a tentative amount.
\item SUSPENDED, the phase will not be included in any calculations.
\item DORMANT, the phase will be included in the calculations but will
  not be allowed to become stable even if that would give the most
  stable equilibrium.  In such a case the phase will have a positive
  driving force.
\item FIXED means that it is a condition that the phase is stable with
  the specified amount.  Note that for solution phases the composition
  is not known.
\end{itemize}

You can use a list of phase names or a wildcard for the phase name and
the must give an equal sign, ``='', before the new status.  You can
also use the special ``*S'' for all suspended phase, ``*D'' for all
dormant phases.

Changing the phase status does not affect anything except the phase
itself.  For a single phase you can use SET PHASE ... STATUS $<$status$>$.

\hypertarget{Set status phase amount}{}

Setting a stable phase as dormant or suspended and calculate the
equilibrium will give you a metastable equilibrium.

Setting a phase status as FIXED means it is a condition that this
phase should be stable.  Setting the liquid fix with the amount zero
is a quick way to calculate the melting temperature of a system if
there is no condition on the T.  For entered phases the amount is used
as a start value.

{\bf Amount: /0/}:

%....................
\hypertarget{Set status species}{}
\subsubsection{{\em set status} Species}

A species can be ENTERED or SUSPENDED.  If a species is suspended
all phases that have this as single constituent in a sublattice
will be automatically suspended.  Not yet implemented.

%--------------------------------
\hypertarget{Set units}{}
\subsection{{\em set} Units}

For each property the unit can be specified like Kelvin, Farenheit or
Celsius for temperature.  Not implemented yet.

%--------------------------------
\hypertarget{Set variable coeff}{}
\subsection{{\em set} Variable coefficient}\label{sc:setvar}

One or more coefficients for optimization, A00 to A99, can be set as
variable to be optimized against the selected experimental data.

A single variable index, 0 to 99, can be used with a start value
provided.  Or a range such as 15-19 which will set all nonzero
variables A15 to A19 as variable.

%--------------------------------
\hypertarget{Set verbose}{}
\subsection{{\em set} Verbose}

Not implemented yet.

%--------------------------------
\hypertarget{Set weight}{}
\subsection{{\em set} Weight}\label{sc:setw}

Intended for assessments.  A weight is zero or a positive value.
Equilibria with weight zero will be ignored in an optimization.  

You can specify the current equilibrium or give an abbreviation that
will set the weight of all equilibria with a name for which the
abbreviation fits.  Or you can give a range of equilibria by giving
two numbers separated by a hyphen like 63-106.

If an abbreviation or a range is given the software will list how many
equilibra that had the weight set to the new value.

%===================================================================
\hypertarget{Show property}{}
\section{Show }\label{sc:show}

This command shows a value of a property, the property can be a state
variable like T, G etc or a user detfined symbol containing several
state variable or a model parameter identifier (which must always have
a phase specification) like the Curie temperature.

The state variables can contain wildcards like X(FCC,*) means all mole
fractions of the FCC phase.  Several properties can be specified on
the same line, SEPARATED BY A SPACE CHARACTER, do not use ``,''.

It is the same as the command {\em LIST state-variables}, see
section~\ref{sc:list_statevar}

\hypertarget{property:}{}
\subsection{property:}

The value of one or more properties or symbols can be shown:
DO NOT USE ``,'' between the properties!

{\small
\begin{verbatim}
--->OC5:show t g tc(bcc) x(bcc,cr) mu(cr) cp
 T=  1.2000000E+03
 G= -5.9565761E+04
 TC(BCC_A2)=  1.0272646E+03
 X(BCC_A2,CR)=  3.100000E-2
 MU(CR)= -7.2489667E+04
CP=   4.08487869E+01
\end{verbatim}
}

%===================================================================
\hypertarget{Step}{}
\section{Step }\label{sc:step}

Requires that a single axis is set.  If a second step command is given
you have the choice of deleting or keeping the previous results.

There are 5 variants of the STEP command, CONDITIONS and NPLE are not
implemented:

\begin{tabular}{llll}
 CONDITIONAL~~ & NPLE              & QUI              & SEPARATE\\
 NORMAL        & PARAEQUILIBRIUM~~ & SCHEIL-GULLIVER~~& TZERO
\end{tabular}

\hypertarget{Step old data}{{\bf Delete previous results?}}

Any previous results from the STEP or MAP commands can be deleted or
kept.  If kept the previous results can be plotted together with the
results from the new STEP command.  The PLOT command also allows
appending previous diagams calculated and plotted by OC.

%--------------------------------
\hypertarget{Step conditional}{}
\subsection{{\em step} Conditional}

A specified symbol is evaluated at each step, not implemented.

%--------------------------------
\hypertarget{Step normal}{}
\subsection{{\em step} Normal}

Calculates equilibria from the low axis limit to the high at each
increment.  The exact axis value for any phase changes is calculated.

%--------------------------------
\hypertarget{Step NPLE}{}
\subsection{{\em step} NPLE}

Step NPLE is similar to step paraequilibrium.

%--------------------------------
\hypertarget{Step paraequilibrium}{}
\subsection{{\em step} paraequilibrium}\label{sc:paraeq3}

Paraequilibrium describes a metastable equilibrium with a fast
diffusing element.  It is described in section~\ref{sc:paraeq1}.  You
should make a calculate paraequilibrium command, see
section~\ref{sc:paraeq2}, before this step command and you must again
specify a matrix phase and a growing phase and the fast diffusing
element.

{\bf Matrix phase:}
              
Note all phases except the matrix and growing phase should be
suspended.  You should provide name of the matrix phase

{\bf Growing phase:}

{\bf Fast diffusing element:}

The element that diffuse so fast that its chemical potential is the
same in both phases.  The other alloying elements will have the same
composition in both phases.


%--------------------------------
\hypertarget{Step Quit}{}
\subsection{{\em step} Quit}

You did not want to {\em step}.

%--------------------------------
\hypertarget{Step Scheil}{}
\subsection{{\em step} Scheil-Gulliver}\label{sc:scheil2}

The Scheil-Gulliver solidification simulation is described in
section~\ref{sc:scheil1}.  It simulates a solidification with no
diffusion in the solid phases and a homogeneous liquid.

%--------------------------------
\hypertarget{Step separate}{}
\subsection{{\em step} Separate}

This command calculates equilibria for each phase separately along the
axis.  It is typically used to separately calculate and plot together
the Gibbs energy curves for a number of phases across a composition
range.

%--------------------------------
\hypertarget{Step Tzero}{}
\subsection{{\em step} Tzero}\label{sc:tzero3}

This will calculate a line with the fraction of the selected element
on one axis and the $T$ on the other and the line is defined by the
fact that the two phases have the same Gibbs energy with the same
composition and at the same $T$.  This is the limit of a diffusionless
transformation.  $T_0$ or Tzero lines are described in
section~\ref{sc:tzero1} and \ref{sc:tzero2}.  Before this step command
you must have calculated a Tzero point.

{\bf First phase:}
              
{\bf Second phase:}

Note all phases except the two phases should be suspended.  You should
provide name of the matrix phase

{\bf Release condition number:}

Normally the step axis is the fast diffusing element and the condition
released is the $T$.  The fast diffusing element will have the same
chemical potential is both phases, the other alloying elements will
have the same composition in both phases.


%===================================================================

% Using this file for on-line help there must be a section after last command
\hypertarget{Summary}{}
\section{Summary }

That's all and I hope enough (when all is implemented).  Have fun and
report all errors or problems providing a macro file and the necessary
data.

\begin{thebibliography}{77zzz}
\bibitem{15Sun1} B Sundman, U R Kattner, M Palumbo and S G
  Fries, {\em OpenCalphad - a free thermodynamic software}, in
  Integrating Materials and Manufacturing Innovation, {\bf 4:1}
  (2015), open access
\bibitem{15Sun2} B Sundman, X-L Liu and H Ohtani, {\em The
  implementation of an algorithm to calculate thermodynamic equilibria
  for multi-component systems with non-ideal phases in a free
  software}, Computational Materials Science, {\bf 101} (2015) 127--137
\bibitem{16Sun} B Sundman, U R Kattner, C Sigli, M Stratmann, R
  Le Tellier, M Palumbo and S G Fries, {\em The OpenCalphad
    thermodynamic software interface}, Comp Mat Sci, {\bf 125} (2016)
  188--196
\bibitem{07Luk} H L Lukas, S G Fries and B Sundman, {\em
  Computational Thermodynamics, the CALPHAD method}, Cambridge Univ
  Press 2007.
\bibitem{20Her} J Herrnring, B Sundman and B Klusemann, {\em
  Diffusion-driven microstructure evolution in OpenCalphad},
  Computational Materials Science, {\bf 175}, (2020) 109236
\bibitem{lmdif}
  https://www.math.utah.edu/software/minpack/minpack/lmdif.html
\bibitem{gnuplot} http://www.gnuplot.info/documentation.html
\bibitem{20Sun} B Sundman, U R Kattner, M Hillert, M Selleby, J
  {\AA}gren, S Bigdeli, Q Chen, A Dinsdale, B Hallstedt, A Khvan, H
  Mao and R Otis, {\em A Method for handling the extrapolation of
    solid crystalline phases to temperatures far above their melting
    point}, Calphad, {\bf 68} 101737
\end{thebibliography}

\end{document}

Old state variable table

\begin{table}[!ht]
\caption{A preliminary table with the state variables and their
  internal representation.  Some model parameter properties are also
  included.  The "z" used in some symbols like Sz means the optional
  normalizing symbol M, W, V or F.}\label{tab:statevold}
{\small
\begin{tabular}{|llcccl|}\hline
Symbol~ & Id & \multicolumn{2}{c}{Index} & Normalizing~ & Meaning\\
        &    & 1 & 2                     &  suffix     & \\\hline
\multicolumn{6}{|c|}{Intensive properties}\\\hline
T      & 1  & -         & -    & - & Temperature\\
P      & 2  & -         & -    & - & Pressure\\
MU     & 3  & component & -/phase  & - & Chemical potential\\
AC     & 4  & component & -/phase  & - & Activity\\
LNAC   & 5  & component & -/phase  & - & LN(activity)\\\hline
\multicolumn{6}{|c|}{Extensive and normallized properties}\\\hline
U      & 10 & -/phase\#set & - & - & Internal energy for system\\
UM     & 11 & -/phase\#set & - & M & Internal energy per mole\\
UW     & 12 & -/phase\#set & - & W & Internal energy per mass\\
UV     & 13 & -/phase\#set & - & V & Internal energy per m$^3$\\
UF     & 14 & phase\#set   & - & F & Internal energy per formula unit\\
Sz     & 2z & -/phase\#set & - & - & entropy\\
Vz     & 3z & -/phase\#set & - & - & volume\\
Hz     & 4z & -/phase\#set & - & - & enthalpy\\
Az     & 5z & -/phase\#set & - & - & Helmholtz energy\\
Gz     & 6z & -/phase\#set & - & - & Gibbs energy\\
NPz    & 7z &  phase\#set & - & - & Moles of phase\\
BPz    & 8z & phase\#set & - & - & Mass of phase\\
Qz     & 9z & phase\#set & - & -  & Stability of phase\\
DGz    & 10z & phase\#set & - & -  & Driving force of phase\\
Nz     & 11z & -/phase\#set/comp & -/comp & -  & Moles of component\\
X      & 111 & phase\#set/comp & -/comp & 0  & Mole fraction\\
X\%    & 111 & phase\#set/comp & -/comp & 100 & Mole per cent\\
Bz     & 12z & -/phase\#set/comp & -/comp & -  & Mass of component\\
W      & 122 & phase\#set/comp & -/comp & 0 & Mass fraction\\
W\%    & 122 & phase\#set/comp & -/comp & 100 & Mass per cent\\
Y      & 130 & phase\#set & const\#subl & -& Constituent fraction\\\hline
\multicolumn{6}{|c|}{Some model parameter identifiers}\\\hline
TC     & - & phase\#set & - & - & Curie temperature\\
BMAG   & - & phase\#set & - & - & Aver. Bohr magneton number\\
MQ\&X  & - & phase\#set & constituent & - & Mobility of X\\
THET   & - & phase\#set & - & - & Debye temperature\\\hline
\end{tabular}
}
\end{table}

