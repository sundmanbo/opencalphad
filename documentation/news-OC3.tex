\documentclass[12pt]{article}
\usepackage[latin1]{inputenc}
\usepackage{graphicx,subfigure}
\topmargin -1mm
\oddsidemargin -1mm
\evensidemargin -1mm
\textwidth 165mm
\textheight 220mm
\parskip 2mm
\parindent 3mm
%\pagestyle{empty}

\begin{document}

\begin{center}
{\Large \bf Summary of features in OC version 3 

including assessments and some older stuff

}

\bigskip

Bo Sundman \today

\end{center}

\section{Background}

The Open Calphad (OC) initiative started in 2010 when a group of
scientists decided that there was a need of a high quality open source
software to gain wider acceptance of computational thermodynamics (CT)
as a useful tool in materials science.  The use of thermodynamic
calculations in many applications is severely restricted by the cost
as well as the hardware and software limitations imposed by most
proprietary thermodynamic software.  Providing a free software would
simplify such implementations and open a much larger market also for
the high quality databases provided by the commercial vendors.

Another aim was to support the scientific interest in new
thermodynamic models and improved algorithms for multicomponent
thermodynamic calculations and a better software for thermodynamic
assessments as decrsibed in the book by Lukas et al.\cite{07Luk}.  At
present such developments can only be done by scientists who are
affiliated to the commercial software companies.

The current OC software is available on~\cite{ocweb}.  For software
collaborations there is also a repository called opencalphad
at~\cite{github}.

The OC software in its present state is mainly of interest for
researchers, scientists and students with programming skills.  In a
few years it may be as stable as the commercial software and can be
used also for teaching computational thermodynamics.  There are two
recently published papers describing OC \cite{15Sun1,15Sun2}.

This summary includes also the essential parts of the new features of
the previous versions.

\section{Structure of the OC software}

The software is divided into packages.  There are well defined
software interfaces between the packages that makes it possible to
extend and change them independently.

\begin{itemize}
\item The General Thermodynamic Package. (GTP) which has data
  structures for storing model parameters, conditions and calculated
  results and code to calculate the Gibbs energy and its first and
  second derivatives of phase when the $T, P$ and constitution of the
  phase is known.

  As this was the first package developed it includes a number of
  general untility facilities needed also by the other packages:

\begin{itemize}
\item The TP function package for storing and calculating functions
  that depend on $T$ and $P$, including first and second derivatives.
\item The METLIB utility package mainly for use by the interactive
  user interface. Originally written in Fortran 77 and modified to the
  new Fortran standard but it includes features that are depreciated
  like ENTRY.
\item The command line user interface with a VAX/VMS flavour is part
  of the METLIB package.
\item The numlib routines for inverting a matrix and solving a system
  of linear equations.  Currently very old and stable but rather
  inefficient routines are used.
\end{itemize}

\item The HMS minimizer implementing the algorithm by
  Hillert\cite{81Hil} for finding the equilibrium state in a
  multicomponent system for many different kinds of external
  conditions.  It makes use of GTP for calculating the Gibbs energy
  and derivatives for each phase.

\item The step/map/plot (SMP) package for calculating and plotting
  diagrams.  It uses HMS for calculating equilibria for conditions
  varying along the axis and the free software GNUPLOT for plotting on
  various devices.

\item The OC-TQ software interface to integrate OC in general
  application software for various simulations.  This has now been
  equipped with an iso-C binding which makes the data structures
  defined in the Fortran package available to programs in C++, java,
  phyton and other software langauges.

\end{itemize}

\section{Features in version 1}

The version 1 release of OC in 2013 could calculate multicomponent
equilibra using Hillert's algorithm\cite{81Hil} for models based on
the Compound Energy Formalism (CEF)\cite{01Hil,07Luk}. It included a
possibility to read unencrypted TDB files and a simple command
interface with macro facilities to set conditions, calculate
equilibria and list results.  It had a grid minimizer to ensure
finding the global minimum and detect miscibility gaps.  There was
also a limited application software interface called OC-TQ.

A compiler for Fortran 95 like GNU gfortran 4.8 or later is required.

\section{Features in version 2}

The most important new facilities since version 1 was generating
property and phase diagrams.  However, these and many of the other
features are still incomplete and fragile and may not work properly in
some cases.  Feedback from users (providing the data and a macro file
reproducing the problem) is the best way to obtain a more stable and
error free software.

A new documentation of the code, a user guide and additional examples
as macro files was also provded.  However, as a complete revision of
both data structures and subroutines are planned for version 3 the
docummentation is not yet fully up to date.  Some of the new features
are summarized below:

\begin{enumerate}
\item The STEP procedure for property diagram.  Such diagrams are
  calculated with a single axis variable and the user can calculate
  and plot how various state variables or model properties depend on
  the axis variable.  A primitive version of the step procedure was
  available also in version 1 but in the new version the exact value
  of the change of the set of stable phases is calculated.

  There is also a ``step separate'' option for Gibbs energy curves and
  similar things when each phase is calculated separately along the
  axis.

  There is a problem with the STEP procedure in a binary system using
  a composition as axis.  The STEP will stop at a phase boundary and
  it does not take into accound that nothing changes in a two-phase
  region except the amount of the phases.

\item The MAP procedure for phase diagrams.  This calculates lines
  where the set of stable phases changes for different values of the
  axis variables.  At present only two axis are allowed but in a
  future release up to 5 axis will be implemented.

  As the mapping has some problems to calculate all lines it is
  possible to execute several map commands and append to the previous
  results.  As an emergency one can remove lines that are wrong by
  editing the input file to GNUPLOT.

  Mapping of binary systems is fairly stable although there are
  problems at the top of miscibility gaps and crossing congruent
  transformations.  This can usually be handelled by several start
  points.

  Mapping of multicomponent system is possible but in general many
  lines are missing.  There is an unresolved problem to exit from
  certain node points.  Invariant equilibria in ternary or higher
  order systems are not implemented.

  The present version of mapping will not discover miscibility gaps.
  The phase diagram for Cr-Fe looks horrible.  Things like that will
  be taken care of in a future release.

  During both MAP and STEP all calculated equilibria are saved and it
  is possible to plot various properties.  All node points are saved
  as equilibria which can be inspected individually and it is also
  possible to copy equilibria along a line to a current equilibria and
  extract values.

  The results from a STEP or MAP command can only be saved
  graphically.  It is not possible to save the results on a file for
  later use but we are working on that.  The user should create MACRO
  files for calculations he would like to repeat.

\item GNUPLOT version 4.6 or later is needed to generate the graphics.
  In the user interface of OC some additional graphics options, like a
  title and ranges of the x and y axis, has been added.  It is also
  possible to edit the output files from OC to take advantage of all
  the graphics facilities of GNUPLOT.

%  GNUPLOT does not support triangular diagrams.

\item The ``dot derivative'' method to calculate derivatives of state
  variables has been implemented.  This allows calculation of
  properties like the heat capacity without resorting to numerical
  derivation.  It makes use of the analytical first and second
  derivatives of the Gibbs energy for $T, P$ and all constituent
  fractions implemented in the model package.

  The implementation is not complete but derivatives of several state
  variables with respect to $T$ are available.

\item The ionic two-sublattice liquid model (I2SL) which can handle
  liquids with and without short range ordering is implemented.

\item For the OC-TQ interface a new method has been introduced to
  identify phases called ``phase tuples''.  A phase tuple is a Fortran
  95 structure (TYPE) with two integer values, one for the phase and
  the other for the composition set.  The user interface of OC also
  use phase tuples when listing phases and composition sets.

  When a phase is entered it has one composition set with number 1 and
  a phase tuple is created with the same index as the phase and the
  composition set index equal to 1.  When a new composition set is
  entered for a phase, either by the user or by the software itself,
  for example the grid minimizer, the phase tuple index for the new
  composition set will be higher than any of the phases and have as
  values the phase number and a composition set number 2 or higher.

  There is an example calling the OC-TQ interface from C++.

\item Minor things
\begin{itemize}
\item The user can select the reference state of the elements for the
  thermodynamic properties.  This should be done before any MAP or
  STEP command.

\item The partitioning of the Gibbs energy for phases with
  order/disorder transformations has been revised and simplified.

\item The grid minimizer for global equilibria has been improved
  handling phases with ions.

\item The user can enter several equilibria for the same system and
  have different conditions in each and calculate them separately and
  transfer data between them.  This facility is used for storing
  step/map results and is a preparation for the software to assess
  model parameters from experimental data.  Each equilibrium is
  independent and they be calculated in parallel.

\item OC has a flexible way to handle properties like mobilities,
  elastic constants etc that may depend on the phase, $T, P$ and the
  phase constitution.  Some properties are predefined but a skilled
  programmer can easily add a specific property and a model to use it
  in a calculation.  The values of such properties can be obtained
  interactivly or by application software in the same way as
  thermodynamic state variables.

\item Reading TDB files is now less strict and it is possible to
  specify the elements to be selected from the database.  With some
  editing of the TDB file it is also possible to read data for ordered
  phases modelled with the Thermo-Calc partitioned method.

\item It is possible to write parameter files in a TDB format but it
  may require editing to be read by other software.

\item Parallelization has been tested for the grid minimizer and for
  the calculation of the inverse phase matrices.  It has been
  indicated in the code where it can be useful to speed up other parts
  of the calculations.  A simple test of the parallelization of
  calulating and inverting of the phase matrices at each iteration
  reduced the time for some calculations by 25\%.

\item Composition sets created automatically by the grid minimizer
  should normally be removed if they are not needed after the full
  equilibrium has been calculated.  

  If the user has created composition sets for phases that may appear
  with specific constitutions, like the cubic MC carbide as a
  composition set of the FCC phase, the OC software will try to assign
  the correct composition to the approriate composition set.

\item Phases with order/disorder transformations like FCC (L1$_2$ and
  L1$_0$) modelled with 4 sublattices can have an ``FCC\_PERMUTATION''
  bit set to simplify entering the parameters.  With this bit set the
  user needs to enter each unique parameter only once, not all the
  permutations.  All kinds of interaction parameters can be entered up
  to second order.  For the BCC ordering permutations are more
  complicated and has not yet been implemented.

\item Phases with LRO ordering (including phases with LRO that never
  disorder like $\sigma, \mu$ and Laves) one can have a disordered
  fraction set for parameters that depend on the overall composition
  of the phase but are independent of the phase constitution.
\end{itemize}

\end{enumerate}

\subsection{Known bugs and problems and missing features}\label{sc:bugs}

Some things are problemnatic and from the long list of things we wanted
to implement but did not manage this time, these are a few:

\begin{itemize}
\item Ternary isothermal sections are difficult to calculate and
  cannot be plotted (even in a square diagram).

\item Redefinition of the components to other species than the
  elements is still not possible.

\item Conditions on some state variables like $V, HM$ etc are not yet
  implemented.  In version 3 one can set condition on $H$ for a
  system.

\item Conditions which are expressions are not implemented.

\item The corrected quasichemical model for liquids is not
  implemented.

\item There is no check on miscibility gaps during a step or map
  command.

\item Saving results from step and map on a file is not possible
  except graphically with GNUPLOT.

\item The mapping is very fragile, lines are frequently missing or
  incomplete.

\item Conditions are not restored after finished step/map.

\item There is no plot of tie-lines.

\item The Scheil-Gulliver solidification model is not implemented.

\end{itemize}

As OC is open source anyone who is interested to implement a
particular feature is welcome to start working on it.

\section{New features in version 3}

The main new feature in version 3 is that OC now has an assessment
facility to determine model parameters.  The hope is to release this
version before the end of 2015 but as this is less than a year after
version 2 was released many things on the wish list from the previous
release have still not been implemented.  Many bugs has been removed
and minor features has been improved.

\begin{itemize}
\item The assessment procedure is a strightforward least square fit of
  experimental data to the same data calculated by the model.  In fact
  very little had to be done in the OC software for this purpose as we
  use as a free optimizing subroutine called LMDIF from the MINPACK
  software developed at Argonne National Lab 1980.  In OC a subroutine
  was added to allow this subroutine to change model parameters in the
  GTP package and to calculate the difference between the experimental
  data provided by the user and the same property calculated from the
  models of the phases.

  A number of new commands had to be added for the user to set up and
  control the assessment process, these are described in
  section~\ref{sc:ass}.

  The current assessment procedure requires skill and experience to be
  used and the ambition is to provide a lot of help for beginners
  using the OC assessment software.  We would also like to develop
  additional statistical analysis of the results like estimating
  uncertainties using the assessment results for extrapolations.

\item A full Fortran/C++ application software interface including the
  use of compatible data structures has been implemented using the
  isoC binding.  This means such software can directly access results
  form an Oc calculation without using calls to subroutines.

\item It took a very long time to implemenent calculations using a
  condition on the enthalpy, $H$, of a system.  This was partly due to
  some shortcuts made in the earlier versions of OC when calculating
  second derivatives of the Gibbs energy.  After implementing these
  derivatives correctly it is now possible to obtain with a single
  calculation for example the adiabatic flame temperature or the
  change in $T$ adding a certain amount of heat to a system.

\item Very little has been done for improving STEP and MAP and for
  many ot the other items in the list in section~\ref{sc:bugs}.  This
  will be the main task in the next update.

\end{itemize}

\section{Long term goals for the OC software}

The long term software goals with the OC initiative has now been
reached 
\begin{itemize}
\item A model package open for implementation of new models.
\item A minimizer calculating equilibria for a flexible set of conditions.
\item Calculation of property and phase diagrams with graphics output.
\item An assessment package for fitting model parameters.  
\item A software interface to applications packages.
\end{itemize}
All of this in an open software free for non-commercial applications.
But there are still many bugs and features missing or not fully
implemented.  This opens a rich field for testing new thermodynamic
models for anyone interested.  It will also be interesting to
implement several minimizers and optimizers and improve the graphics.

\section{Short description of all commands}

See the ochelp3 file for a more extensive documentation.  There is
also a descrciption of the procedure and commands needed to perform an
assessments in section~\ref{sc:ass}

\subsection{All top level commands}

\begin{tabular}{llll}
 ABOUT        &   ENTER        &   LIST        &     QUIT\\
 AMEND        &   EXIT         &   MACRO       &     READ\\
 BACK         &   FIN          &   MAP         &     SAVE\\
 CALCULATE    &   HELP         &   NEW         &     SELECT\\
 DEBUG        &   HPCALC       &   OPTIMIZE    &     SET\\
 DELETE       &   INFORMATION  &   PLOT        &     STEP\\
\end{tabular}

One restriction I have applied to commands and subcommands is that one
should not need to type more than 3 characters to have a unique
abbreviation.

Directly after the top level command the user can give some options
preceeded by a slash like /output=filename or /append=filename.  The
output that would normally appear on the screen will instead be listed
on this file with extention DAT.  The output option will delete any
previous content on the file but the append will add the new output at
the end of the file.  The output is reset to the screen after the
command.

A frequent Thermo-Calc user must disable his tendency to put hyphens
or underscore characters between the command and subcommand.

\subsection{Commands without subcommands}

\begin{itemize}
\item ABOUT the software
%%%%%%%%%%%%%%%%%%%%%%%%%%%%%%%%%%%%%%%%%%%%%%%%%%%%%%%%%5
\item BACK to calling software (or exit) after confirmation
%%%%%%%%%%%%%%%%%%%%%%%%%%%%%%%%%%%%%%%%%%%%%%%%%%%%%%%%%5
\item EXIT, terminate the software (in Swedish) after confirmation
%%%%%%%%%%%%%%%%%%%%%%%%%%%%%%%%%%%%%%%%%%%%%%%%%%%%%%%%%5
\item FIN, terminate the software (in French) without confirmation
%%%%%%%%%%%%%%%%%%%%%%%%%%%%%%%%%%%%%%%%%%%%%%%%%%%%%%%%%5
\item HELP gives explanations about a (few) commands from the user guide
%%%%%%%%%%%%%%%%%%%%%%%%%%%%%%%%%%%%%%%%%%%%%%%%%%%%%%%%%5
\item HPCALC starts the inverse polish calculator
%%%%%%%%%%%%%%%%%%%%%%%%%%%%%%%%%%%%%%%%%%%%%%%%%%%%%%%%%5
\item INFORMATION is not implemented yet
%%%%%%%%%%%%%%%%%%%%%%%%%%%%%%%%%%%%%%%%%%%%%%%%%%%%%%%%%5
\item MACRO asks for name of macro file and exectutes it.  A macro
  file can call another macro five levels deeply.
%%%%%%%%%%%%%%%%%%%%%%%%%%%%%%%%%%%%%%%%%%%%%%%%%%%%%%%%%5
\item MAP a phase diagram with 2 or more independent axis.
%%%%%%%%%%%%%%%%%%%%%%%%%%%%%%%%%%%%%%%%%%%%%%%%%%%%%%%%%5
\item NEW asks for confirmation and if so removes all data.
%%%%%%%%%%%%%%%%%%%%%%%%%%%%%%%%%%%%%%%%%%%%%%%%%%%%%%%%%5
\item OPTIMIZE asks for the maxmimum number of iterations and then use
  a least square routine to obtain the best fit to experimental data
  by varying the set of model parameters specified by the user.
%%%%%%%%%%%%%%%%%%%%%%%%%%%%%%%%%%%%%%%%%%%%%%%%%%%%%%%%%5
\item QUIT terminates the software (in Enlish) after confirmation
%%%%%%%%%%%%%%%%%%%%%%%%%%%%%%%%%%%%%%%%%%%%%%%%%%%%%%%%%5
\end{itemize}

\subsection{Commands with subcommands}

Many subcommands are not implemented, I have indicated some but not
all.

\begin{itemize}
\item AMEND should change something already entered or set.  But
  sometimes it creates something so one should be careful if it should
  be an ENTER or SET command.
  \begin{itemize}
  \item ALL\_OPTIM\_COEFF means you can rescale or recover previous
    set of the optimizing coefficients.
%============================================
  \item BIBLIOGRAPHY asks for a bibliographic id and amends its text.
%============================================
  \item COMPONENTS changes the set of components (not implemented).
%============================================
  \item CONSTITUTION asks for a phase and the user can amend the
    amount and current constitution of a phase.  I have put this here
    and not as part of AMEND PHASE to avoid confusion with adding a
    composition set.
%============================================
  \item CP\_MODEL not implemented and probably redundant.
%============================================
  \item ELEMENT amends data for an element (not implemented).
%============================================
  \item EQUILIBRIUM is not implemented yet.
%============================================
  \item GENERAL can name of current equilibrium, the user can specify
    if he is a beginner (the software can provide more help (not
    implemented)), or expert, if global gridminimizer should be used
    and if it can merge composition sets, if composition sets can be
    created automatically and if redundant composition set can be
    deleted after an equilibrium calculation. The latter questions are
    mainly interesting for debugging.
%============================================
  \item PARAMETER amends a parameter expression for a phase (not
    implemented, use ENTER PARAMETER to change it.
%============================================
  \item PHASE (default) asks for phase name and amends data for this
    phase like:
    \begin{itemize}
    \item COMPOSITION\_SET (default) adds or deletes a composition
      set.  Composition sets are needed for miscibility gaps when a
      phase can be stable with two or more compositions.  A
      composition set can be identified with a hash symbol ``\#''
      followed by a number or by a user specified prefix or suffix.
  
      Deleting a set will always be the one with the highest number.
      When adding a set the user can provide a prefix and suffix and
      the default constitution for this set.
%--------------------------------------------
    \item MAGNETIC\_CONTRIB adds an Inden-Hillert magnetic model.
%--------------------------------------------
    \item DISORDERED\_FRACS adds a disordered fraction set for an
      ordered phase.  This replaces AMEND ... DISORDERED\_PART in
      Thermo-Calc.
%--------------------------------------------
    \item GLAS\_TRANSITION adds a glas transition model (not
      implemented).
%--------------------------------------------
    \item QUIT You did not want to amend anything for the phase
%--------------------------------------------
    \item DEFAULT\_CONSTIT amends the default constitution for a
      composition set of the phase.  Same as SET PHASE ``name''
      DEFAULT\_CONSTITU.
%--------------------------------------------
    \item DEBYE\_CP\_MODEL adds a Debye Cp model (not implemented).
%--------------------------------------------
    \item EINSTEIN\_CP\_MDL adds a Einstein Cp model (not implemented).
%--------------------------------------------
    \item INDEN\_WEI\_MAGMOD adds a magnetic model according to Wei
      (not implemented).
%--------------------------------------------
    \item ELASTIC\_MODEL\_A adds an elastic model (not implemented).
      Note the LIST MODEL\_PARAM\_ID how to enter $T, P$ and composition
      dependent elasic constants and lattice parameters.
    \end{itemize}
%============================================
  \item QUIT you did not want to amend anything.
%============================================
  \item SPECIES amends data for a species (not implemented).
%============================================
  \item SYMBOL the user can specify if the symbol can only be
    calculated when explicitly named (usually all symbols are
    evaluated when any symbol is evaluated as they can depend on each
    other).  This is needed for symbols used as conditions.  The user
    can also specify that a symbol should be local to a specific
    equilibrium.  In this way one can store the value of a symbol from
    one quilibrium and calculate differences with respect to other
    equilibria.
%============================================
  \item TPFUN\_SYMBOL amends a TPfun expression.
  \end{itemize}
%%%%%%%%%%%%%%%%%%%%%%%%%%%%%%%%%%%%%%%%%%%%%%%%%%%%%%%%%5
\item CALCULATE various things like:
  \begin{itemize}
  \item ALL\_EQUILIBRIA all equilibria within the range set for
    experimental equilibria with non-zero weight are calculated.  This
    is useful for testing if there are any problems before optimizing.
%------------------------------
  \item EQUILIBRIUM (default) is the normal equilibrium calculation
    command which first calls the grid minimizer (if the conditions
    allow) and then the iterative minimizer.
%------------------------------
  \item GLOBAL\_GRIDMIN Only the grid minimizer is called to find the
    gridpoints that reperesent the lowest Gibbs energy.  These are
    normally used by the iterative minimizer to find the real
    equilibrium.  If followed by COMPUTE NO\_GRDMIN one will have the
    same result as COMPUTE EQUILIBRIUM.
%------------------------------
  \item NO\_GLOBAL calculates the equilibrium for the current set of
    conditions starting from the current set of stable phases and
    their constitutions.  No grid minimizer called.
%------------------------------
  \item PHASE ask for phase name, amount and constitution and at
    current $T$ and $P$ calculates either:
    \begin{itemize}
    \item ONLY\_G Gibbs energy and first and second derivatives with
      respect to $T$ and $P$.
%................
    \item G\_AND\_DGDY calculates also all first derivatives with
      respect to the phase constituents.
%................
    \item ALL\_DERIVATIVES Also all second derivatives with respect to
      the phase constituents.
%................
    \item CONSTITUTION\_ADJUST will calculate G and all derivatives
      after adjusting the constitution of the phase to have the
      minimum Gibbs energy for the same overall composition as the
      original constitution.  It is only interesting when one or more
      components are parts of several constituents (not implemented).
    \end{itemize}
%------------------------------
  \item QUIT if you did not really want to calculate anything.
%------------------------------
  \item SYMBOL Calculate the value of one or all symbols at the current
    equilibrium.
%------------------------------
  \item TPFUN\_SYMBOLS all TP functions values and their first and
    second derivatives with respect to T and P (6 values).
%------------------------------
  \item TRANSITION asks for a phase to be stable with zero amount and
    a condition to be released to calculates the equilibrium.  The
    phase must not have the FIX status.  After the calculation the
    phase is set to be entered and the released condition set to the
    calculated value.  If calculation fails the status is not reset
    (sorry I have not had time to do all).  No grid minimizer called.
  \end{itemize}
%%%%%%%%%%%%%%%%%%%%%%%%%%%%%%%%%%%%%%%%%%%%%%%%%%%%%%%%%5
\item DEBUG           Nothing of this works
  \begin{itemize}
  \item FREE\_LISTS      
%------------------------------
  \item STOP\_ON\_ERROR   
%------------------------------
  \item ELASTICITY 
  \end{itemize}
%%%%%%%%%%%%%%%%%%%%%%%%%%%%%%%%%%%%%%%%%%%%%%%%%%%%%%%%%5
\item DELETE Only composition sets and equilibria can be deleted.  To
  delete a parameter you can amend its expression to be zero.
  \begin{itemize}
  \item PHASE (default) but not allowed
%------------------------------
  \item ELEMENTS        not allowed
%------------------------------
  \item SPECIES         not allowed
%------------------------------
  \item QUIT            you did not want to delete anything
%------------------------------
  \item COMPOSITION\_SET The highest set is deleted (one cannot delete
    all).  Must be used with great care.
%------------------------------
  \item EQUILIBRIUM must be used with great care.
  \end{itemize}
%%%%%%%%%%%%%%%%%%%%%%%%%%%%%%%%%%%%%%%%%%%%%%%%%%%%%%%%%5
\item ENTER is the main command to enter data interactivly.  Note that
  in most cases data are read from a TDB file.
  \begin{itemize}
  \item BIBLIOGRAPHY enter a bibligraphic reference id and text.
%------------------------------
  \item CONSTITUTION to enter the constitution of a phase (same as AMEND
    CONSTITUTION).
%------------------------------
  \item COPY\_OF\_EQUILIB the current equilibrium is coped to a new one
    with a name specified by the user.
%------------------------------
  \item ELEMENT an element with data.
%------------------------------
  \item EQUILIBRIUM an equilibrium record with the specified name is
    created.  Each equilibrium record has an independent set of
    conditions.  Will be used for assessments and is already used to
    store node points during step and map.
%------------------------------
  \item EXPERIMENT data for assessments.
%------------------------------
  \item OPTIMIZE\_COEFF the coefficinets for use in assessments are
    entered.
%------------------------------
  \item PARAMETER the expression of a parameter of a phase.  The phase,
    the constituent array and degree must be specified.
%------------------------------
  \item PHASE a phase with sublattices, site ratios and constituents.
    The parameters are entered individually with ENTER PARAMETER.
%------------------------------
  \item QUIT you did not want to enter anything.
%------------------------------
  \item SPECIES a species with name and stochiometry.  Its name must be
    unique but one can have several with the same stoichiometry.
%------------------------------
  \item SYMBOL (default) name and expression of a state variable
    function.
%------------------------------
  \item TPFUN\_SYMBOL The name and expression of a function of T and P
    that can be used in parameters.
  \end{itemize}
%%%%%%%%%%%%%%%%%%%%%%%%%%%%%%%%%%%%%%%%%%%%%%%%%%%%%%%%%5
\item LIST of many things ...  

  Note the possibility to direct output to a file using
  /output=filename or /append=filename directly after the command, as
  mentioned in the beginning.
  \begin{itemize}
  \item AXIS lists current axis set by the user.
%=======================================================                
  \item BIBLIOGRAPHY lists bibliographic text for specific id or all.
%=======================================================                
  \item CONDITIONS lists all conditions in current equilibrium.
%=======================================================                
  \item DATA lists all parameters on different devices and ways:
    \begin{itemize}
    \item SCREEN (default) Writes all parameters for all phases on the
      screen including the bibliographic information.
%-------------------------------------------------------
    \item TDB Writes all parameters on file in TDB format.
%-------------------------------------------------------
    \item MACRO Writes all parameters on file as a macro file (not
      implemented).
%-------------------------------------------------------
    \item LATEX Writes all parameters on file as a LaTeX file (not
      implemented).
    \end{itemize}
%=======================================================
  \item EQUILIBRIA lists all entered equilibria with name and number (no
    results).
%=======================================================                
  \item LINE\_EQUILIBRIA lists all equilibra stored during STEP or MAP.
    With the SET ADVANCED command one can copy one of these to the
    current equilibrium.
%=======================================================                
  \item MODEL\_PARAM\_ID lists all implemented model parameter
    identifiers like G, TC, BMAGN, elastic constants etc. that can
    depend on $T, P$ and constitution of a phase.  The use of such
    parameters require implementation of the model in the software.
%=======================================================                
  \item OPTIMIZATION the result of an optimization is listed.
%=======================================================                
  \item PARAMETER lists the expression for a single parameter.
%=======================================================                
  \item PHASE asks for phase name and then lists for option
    \begin{itemize}
    \item CONSTITUTION (DEFAULT) lists constitution for this phase.
%-------------------------------------------------------
    \item DATA lists parameter for this phase (no bibliography).
%-------------------------------------------------------
    \item MODEL lists some model information for this phase.
    \end{itemize}
%=======================================================                
  \item QUIT You do not want to list anything.
%=======================================================                
  \item RESULTS (default) from an equilibrium calculation.  the program
    asks for a number how to format the phase information.  
    \begin{enumerate}
    \item means stable phases and composition in mole fractions in value order
    \item means stable phases and composition in mole fractions and constitution
    \item means stable phases and composition in mole fractions and
      alphabetical order??
    \item means stable phases and composition in mass fractions in value order
    \item means stable phases and composition in mass fractions in
      alphabetical order??
    \item means stable phases and composition in mole fractions and constitution
    \item means all phases and composition in mass fractions
    \item means all phases and composition in mole fractions and constitutions
    \item means all phases and composition in mole fractions and
      alphabetical order
    \end{enumerate}
    The conditions are listed first, then some global properties, then
    some data for each component and then the stable phases with
    amount and composition.
%=======================================================                
  \item SHORT
    \begin{itemize}
    \item A writes one line for all elements, species and
      phases.
    \item P writes one line for all phases sorted with stable first,
      then max 10 entered phases in decreasing stability, finally the
      dormant in decreasing stability.
    \end{itemize}
%=======================================================                
  \item STATE\_VARIABLES asks for state variable symbol and lists it
    value.  Can also be used for the current value of model parameters
    as Curie temperature, TC(BCC), MQ\&FE(FCC), etc.

    {\bf I will try to merge list state\_variables and compute symbol
      to a SHOW command}
%=======================================================                
  \item SYMBOLS lists all entered state variable symbols (same in all
    equilibria).
%=======================================================                
  \item TPFUN\_SYMBOLS lists one or all TP function symbols and
    expressions (same in all equilibria).
  \end{itemize}
%%%%%%%%%%%%%%%%%%%%%%%%%%%%%%%%%%%%%%%%%%%%%%%%%%%%%%%%%5
\item PLOT asks for state variables or symbols for x and y axis and
  after that the user can plot directly or change anything in the
  submenu below.  

  OC generates a command file for GNUPLOT and a data file with the
  values to plot and then executes this in a separate shell.  The user
  can edit the command file to add options and execute it again inside
  gnuplot.  But beware not to overwrite the files you want to edit.
  There are 10 colors for the lines to plot.  If more than 10 lines to
  plot the colors are repeated cyclically.

  OC keeps the previous values set of all options set (except the
  scaling of an axis with a new variable and the output file which is
  always reset to the default ``ocgnu'') unless changed explicitly.

  I have no idea how to overlay a calculated result with for example
  experimental data.  Hopefully some GNUPLOT expert will tell me how
  to do that.

  \begin{itemize}
  \item RENDER (default) finally plot when all options set.
%=======================================================                
  \item XRANGE set plot range on x axis.
%=======================================================                
  \item YRANGE set plot range on y axis.
%=======================================================                
  \item XTEXT set text on x axis.
%=======================================================                
  \item YTEXT set text on y axis.
%=======================================================                
  \item TITLE set title of plot.
%=======================================================                
  \item POSITION\_OF\_KEYS select position of the labels
    (identification) of the lines in the plot.  The lables can be
    placed inside/outside of the plot, to the left/center/right and
    top/bottom.  See the explanation of ``set key'' in GNUPLOT.
%=======================================================                
  \item GRAPHICS\_FORMAT set type of terminal (P for postscript, G for
    gif).  You will also be asked for output file.
%=======================================================
  \item OUTPUT\_FILE set name of plot file (default is ocgnu.dat).
%=======================================================
  \item GIBBS\_TRIANGLE set diagram to be a Gibbs triangle (not implemented).
%=======================================================                
  \item QUIT you do not want to plot.

    {\bf More options will be added when I understand GNUPLOT better}

  \end{itemize}
%%%%%%%%%%%%%%%%%%%%%%%%%%%%%%%%%%%%%%%%%%%%%%%%%%%%%%%%%5
\item READ At present only TDB and UNFORMATTED implemented.
  \begin{itemize}
  \item TDB (default) an unencrypted TDB file can be read.  Many
    TYPE\_DEFS are not handelled correctly and warning are given.  For
    partitioned phases you may have to edit the parameters.  The user
    can select which elements he wants to read.
%=======================================================                
  \item UNFORMATTED an unformatted file with model parameters and
    results for a single equilibrium calculation.  This is very fragile
    as any change in the data structure may make it impossible to read.
%=======================================================                
  \item QUIT you did not want to read anything.
%=======================================================                
  \item DIRECT will save results from STEP and MAP on a random acess
    file (not implemented).
  \end{itemize}
%%%%%%%%%%%%%%%%%%%%%%%%%%%%%%%%%%%%%%%%%%%%%%%%%%%%%%%%%5
\item SAVE The only save option (partially) implemented is
  unformatted.
  \begin{itemize}
  \item UNFORMATTED (default) A file is written with unformatted data
    for all thermodynamic data and conditions and results for a single
    equilibrium.  There is no guarantee an unformatted file will be
    readable in a later version of OC.
%=======================================================                
  \item DIRECT not implemeted yet (for STEP and MAP results).
%=======================================================                
  \item QUIT do not save anything.
  \end{itemize}
%%%%%%%%%%%%%%%%%%%%%%%%%%%%%%%%%%%%%%%%%%%%%%%%%%%%%%%%%5
\item SELECT a few things.
  \begin{itemize}
  \item EQUILIBRIUM (default) change the current equilibrium to the
    selected one (number or name or next or previous).
%=======================================================                
  \item MINIMIZER there is only one.       
%=======================================================                
  \item OPTIMIZER there is only one.       
%=======================================================                
  \item GRAPHICS there is only one.
%=======================================================                
  \item LANGUAGE there is only one (English).
\end{itemize}
%%%%%%%%%%%%%%%%%%%%%%%%%%%%%%%%%%%%%%%%%%%%%%%%%%%%%%%%%5
\item SET can be used for many things.  The most important is
  conditions.
  \begin{itemize}
  \item ADVANCED This command for very special things.
    \begin{itemize}
    \item EQUILIB\_TRANSF  transfer an equilibrium calculated along a line
      in STEP or MAP to current equilibrium.

      {\bf this is probably the most awkward command of all.  But I do not
        want to have a TRANSFER or COPY command on the top level as that
        will certainly be misunderstood and misused}

    \item QUIT you did not want to set anything advanced.
    \end{itemize}
%================================================
  \item AS\_START\_EQUILIB use current equilibrium as start for step
    or map.  Not necessary if there is only one start equilibrium.
%================================================
  \item AXIS axis ``number'' to an independent variable (must be a
    condition).
%================================================
  \item BIT some global bits can be set
%================================================
  \item CONDITION (default) the state variable and value of a condition.
    Only single values are allowed, expressions not yet implemented.
%================================================
  \item ECHO echo of the input from macro files on the screen.
%================================================
  \item FIXED\_COEFF to set an optimizing coefficient to a fixed value.
%================================================
  \item INPUT\_AMOUNTS amount of species.  These will be added
    together and used for conditions of the components.
%================================================
  \item INTERACTIVE at the end of macro files.
%================================================
  \item LEVEL I am not sure what this was intended for.
%================================================
  \item LOG\_FILE the name of a file with a copy of all input and
    defaults.
%================================================
  \item NUMERIC\_OPTIONS maximum number of iterations (default 500)
    and convergence limit (default 10$^{-6}$).
%================================================
  \item OPTIMIZING\_COND depending on the optimizer used for
    assessment some conditions can be set.
%================================================
  \item PHASE the user must specify a phase name and can then ...
    \begin{itemize}
    \item STATUS (default) the status of a single phase, or all using
      an asterisk ``*'', can be set.  See also SET STATUS PHASE
      with a more flexible way to specify phases.
%------------------------------------------------
    \item DEFAULT\_CONSTITU the default constitution of the phase can be
      set.  Same as AMEND PHASE name DEFAULT\_CONSTITU.
%------------------------------------------------
    \item AMOUNT the amount of the phase (redundant).
%------------------------------------------------
    \item QUIT nothing is set for the phase.
%------------------------------------------------
    \item BITS some special bits for a phase can be set.  At present
      there is no way to ``UNSET'' these bits ... so be careful.
      \begin{itemize}
      \item QUIT (default) no bit is changed.
%.....................................
      \item FCC\_PERMUTATIONS to indicate 4 sublattice fcc or hcp
        permutations.  Only one parameter stored for each unique
        permutatation.  Must be set before any parameters are entered.
%.....................................
      \item BCC\_PERMUTATIONS to indicate 4 sublattice bcc
        permutations.  Must be set before any parameters are entered
        (when implemented).
%.....................................
      \item IONIC\_LIQUID\_MDL to indicate ionic liquid model.  A
        phase with the ionic liquid model that is entered interactivly
        must currently have the name ionic\_liquid and this bit is
        automatically set.  If read from a TDB file the :Y after the
        phase name assignes this model.  Setting this bit interactivly
        has no function at present.
%.....................................
      \item AQUEOUS\_MODEL to indicate aqueus model (not impemented).
%.....................................
      \item QUASICHEMICAL to indicate quasichemical model (not implemented).
%.....................................
      \item FCC\_CVM\_TETRADRN to indicate CVM fcc tetrahedon model (not
        implemented).
%.....................................
      \item FACT\_QUASICHEMCL to indicate FACT quasichemicval model (not
        implemented).
%.....................................
      \item NO\_AUTO\_COMP\_SET to prevent automatic creations of
        composition sets for this phase.  One can forbid creating
        automatic composition sets (by the grid minimizer) for all
        phases with the AMEND GENERAL command.
      \end{itemize}
%================================================
  \item QUIT nothing is set.
%================================================
  \item RANGE\_EXP\_EQUIL the first and last equilibrium included in
    the assessment must be specified.
%================================================
  \item REFERENCE\_STATE the reference state of a component.  The phase,
    $T$ and $P$ must be specified.  The phase must exist with the
    component as its only component.  When the phase can exist with the
    component in different ways, like O in a gas can be O, O$_2$ or
    O$_3$ the most stable is selected.
%================================================
  \item SCALED\_COEFF can be used to set a coefficient to be optimized
    with a specified scaling factor and popssibly a min and max value.
%================================================
  \item STATUS          
    \begin{itemize}
    \item PHASE (default) one or more phases can be set as suspended,
      dormant, entered or fixed.  You can use * to mean all phases, *S
      for all suspended, *D for all dormant and *E for all entered and
      *U for all entered and unstable.  The list of phases is
      terminated by an equal sign ``='' or an empty line.

      If the new status is not already given after the equal sign it
      is asked for.  If the new status is entered or fixed the amount
      is asked for.
%------------------------------------------------
    \item ELEMENT an element can be entered or suspended.
%------------------------------------------------
    \item SPECIES a species can be entered or suspended.
%------------------------------------------------
    \item CONSTITUENT not implemented.
    \end{itemize}
%================================================
    \end{itemize}
%================================================
  \item UNITS like energy Joule/cal or mass kg/lb ... but not implemented yet.
%================================================
  \item VARIABLE\_COEFF an optimizing parameter is specfied.
%================================================
  \item VERBOSE the software will write extra output.
%================================================
  \item WEIGHT to be used for assessments.
  \end{itemize}
%%%%%%%%%%%%%%%%%%%%%%%%%%%%%%%%%%%%%%%%%%%%%%%%%%%%%%%%%5
\item STEP is used to calculate along a single independent axis
  variable.
  \begin{itemize}
  \item NORMAL follow the axis variable from low to high limit.
%================================================
  \item SEPARATE calculate each phase separately.
%================================================
  \item QUIT do nothing.
%================================================
  \item CONDITIONAL follow the axis variable and update s symbol after
    each step (to be used for Scheil-Gulliver simulations, not yet
    implemented).
  \end{itemize}
%%%%%%%%%%%%%%%%%%%%%%%%%%%%%%%%%%%%%%%%%%%%%%%%%%%%%%%%%5

\end{itemize}

\section{The assessments procedure in OC}\label{sc:ass}

The normal procedure for an assessment would roughly be performed
in the following steps.

\begin{enumerate}
\item On a  macro file file you have the commands to:
  \begin{itemize}
  \item ENTER elements, species, phases etc. from a macro file.
  \item ENTER OPT to enter the coefficients to be optimized, A00 to A99.
  \item ENTER PARAMETERS with coefficients to be optimized using the
    symbols A00 to A99.
    \end{itemize}

\item on the same or on another macro file you have the experimental
  data.
  \begin{itemize}
  \item The data is entered by ENTER EQUILIBRIA command for each
    experiment and in addition to the necessary conditions you use
    ENTER EXPERIMENT to specify the experimental data.
  \item With the command SET RANGE\_EXP\_EQUIL you specify the range of
    equilibria with experimental data.
  \end{itemize}

\item The rest of the assessment is done interactivly and the set of
  commands are used as needed.  Thses must normally be reiterated many
  many times using different weights of the equilibria and optimizing
  variables.
  \begin{itemize}
  \item You SET VARIABLE\_COEFF to specify a coefficient to be optimized.
  \item You OPTIMIZE to make a least square fit.  
  \item You LIST OPTIMIZATION to list the current result.
  \item You may SET WEIGHT on the different equilibria.
  \item You may AMEND ALL\_OPTIM\_COEFF to rescale or recover
    parameters and use many other commands.
  \end{itemize}

\item When you finished you SAVE TDB to create a TDB file with the
  results (need editing).
\end{enumerate}

\subsection{Special commands for an assessment}

The special commands for performing an assessment is described a
little more in detail here.  Many of the other commands in OC are also
needed.

\begin{description}
\item{\bf ENTER OPTIMIZE\_COEFF} enters up to 100 coefficients to be
  optimized.  They are called A00 to Aij (default A99).  This must be
  done before any assessment.  The symbols Aij can be used when
  entering TP functions or parameters.

\item{\bf ENTER EQUILIBRIUM ``name''} experimental data are related to
  an equilibrium.  This commands enters the necessary data structure
  to set to conditions for an experiment and to add experimental data.

\item{\bf ENTER EXPERIMENT} for each equilibria the
  experimental state variable, value and uncertainity is added with
  this command.

\item{\bf AMEND ALL\_OPTIM\_COEFF} to rescale or recover their values.

\item{\bf LIST OPTIMIZATION} lists results of the optimization in
  various ways.  Only the SHORT option implemented.

\item{\bf OPTIMIZE} calls the optimizer to obtain a least square fit
  of the values calculated from the model parameters to the
  experimental data.

\item{\bf READ EXPERIMENT\_DATA} reads a file with equilibria
  definitions together with experimental data (POP file).  Not
  implemented yet, such data can also be read from a macro file.

\item{\bf SET RANGE\_EXP\_EQUIL ``first'' ``last''} must be given to
  specify the equilibria with experimental data.  They must have been
  entered sequentially.  An AMEND RANGE will be added.

\item{\bf SET OPTIMIZING\_COND} to set some parameters for the
  optimizer.

\item{\bf SET VARIABLE\_COEFF ``index'' ``start value''} set a start
  vaule of a coefficient ij to be optimized.

\item{\bf SET SCALED\_COEFF} ``index'' ``start value'' additionally
  asks for scaling factor, a min and max value for a coefficient.  Not
  yet implemented.

\item{\bf SET FIXED\_COEFF ``index ``value''} to set an optimizing
  coefficent ij to a fixed value (usually its current value)

\item{\bf SET WEIGHT ``value'' ``range''} assigns a value to the
  weight (default unity) to equilibria with names fitting the
  ``range''.  If the weight is set to zero the experiments from these
  equilibria will be ignored.

\item{\bf SELECT EQUILIBRIUM ``name/number''} to select an equilibrium
  to perform some changes.

\end{description}

There is no special ``edit'' module as in Thermo-Calc for modifying
the equilibria with experiments, this can be done using the {\bf
  select\_equilibrium} command. The first equilibrium (DEFAULT) cannot
be used for experimental data.  Using {\bf step} and {\bf map} will
also create new equilibria for node points, they may interfere with
the experimental equilibria, I do not know.  I have to implement some
nice way to get rid of redundant equilibria, the {\bf delete
  equilibria} is a bit dangerous.  There is no way yet to save results
like on a PAR file in Thermo-Calc.

\begin{thebibliography}{77Zzz}
\bibitem[81Hil]{81Hil} M Hillert, Physica, {\bf 103B} (1981) 31
\bibitem[01Hil]{01Hil} M Hillert, J of Alloys and Comp {\bf 320} (2001) 161
\bibitem[07Luk]{07Luk} H L Lukas, S G Fries and B Sundman, {\em Computational
Thermodynamics}, Cambridge Univ Press (2007)
\bibitem[http://www.opencalphad.org]{ocweb} http://www.opencalphad.org
\bibitem[http://github.com]{github} opencalphad at http://github.com.  
\bibitem[15Sun1]{15Sun1} B Sundman, U Kattner, M Palumbo and S G
  Fries, OpenCalphad - a free thermodynamic software, Integrating
  Materials and Manufacturing Innovation, {\bf 4}:1 (2015), open
  access
\bibitem[15Sun2]{15Sun2} B Sundman, X-G Lu, H Ohtani, Comp Mat. Sci
  {\bf 101}(2015) 127
\end{thebibliography}

\end{document}
